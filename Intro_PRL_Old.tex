Since the earliest developments of thermodynamics, there have been attempts to cast the theory in geometric form \cite{gibbs1873method}. A geometric interpretation exists for equilibrium thermodynamics—for instance, the work performed by a heat engine corresponds to the area enclosed by its stroke in the $PV$ plane. However, an analogous geometric relation for excess dissipation in non-equilibrium processes remains elusive.

One approach that bridges this gap is thermodynamic geometry, introduced in the 1970s by Ruppeiner and Weinhold \cite{ruppeinerThermodynamicsRiemannianGeometric1979, weinholdMetricGeometryEquilibrium1975}. In their formulation, the space of thermodynamic parameters is endowed with a Riemannian metric derived from the second derivatives of a thermodynamic potential—such as the internal energy, entropy, or any of their Legendre transforms. Although the choice of potential changes the explicit form of the metric, the resulting manifold remains invariant \cite{salamonRelationEntropyEnergy1984}.

In the 1980s, Salamon \cite{salamonThermodynamicLengthDissipated1983} established a connection between excess dissipation and thermodynamic length in this Riemannian manifold, subject to two important limitations. First, the result holds only for endoreversible processes, where the system remains in internal equilibrium but may be out of equilibrium with its environment. Second, the dissipation depends explicitly on the mean response time of the system, restricting the formalism’s applicability to systems with varying relaxation times. Despite these limitations, the framework has proven valuable for identifying minimal-entropy-production protocols in processes such as chemical reactions and distillation \cite{andresenCurrentTrendsFiniteTime2011}.

Building on this line of work, Sivak and Crooks \cite{sivakThermodynamicMetricsOptimal2012, crooksMeasuringThermodynamicLength2007} derived a closely related metric using linear response theory. In their formulation, the system’s relaxation dynamics are incorporated directly into the metric, thereby removing the dependence on a separate response-time parameter and eliminating the assumption of endoreversibility. This generalized framework extends thermodynamic geometry to a much broader class of systems—including those with widely varying time scales and microscopic systems such as biomolecular or single-molecule systems—and has found applications in optimizing computational and physical processes \cite{rotskoffGeometricApproachOptimal2017}.

A particularly challenging class of systems exhibiting widely separated time scales are those undergoing phase transitions. The application of thermodynamic geometry to such systems has drawn significant attention \cite{janyszekRiemannianGeometryThermodynamics1989, PhysRevE.51.1006}, including efforts to identify optimal driving protocols across critical regions (e.g., the 2D ferromagnetic Ising model \cite{rotskoffDynamicRiemannianGeometry2015}). Yet, the divergence of thermodynamic length near phase transitions has received limited study. Notably, a divergence in the metric tensor does not necessarily imply a divergence in thermodynamic length—much as the divergence of the Schwarzschild metric at $r=r_s$ does not correspond to a physical singularity.

In systems with disconnected phases, any transformation from one phase to another must necessarily cross a phase transition. One such case, examined in this work, is the mean-field antiferromagnet, where the phase boundary entirely separates distinct thermodynamic states. Determining whether the thermodynamic length remains finite near such transitions is crucial for understanding the applicability of thermodynamic geometry in these regimes.

Both the Ruppeiner and Sivak metrics predict that the thermodynamic metric should diverge in the vicinity of a second-order phase transition (see Sec.~\ref{sec:framework}). A natural question, therefore, is whether this divergence of the metric translates into a divergence of the corresponding thermodynamic lengths. Note that even if the length remains finite, other geometric quantities such as the curvature may diverge, signaling critical behavior. Though these do not directly contribute to dissipation and are thus outside our present focus.

In this paper, we analyze the behavior of thermodynamic geometry near second-order phase transitions. We show that the Ruppeiner and Sivak manifolds exhibit qualitatively distinct behavior around criticality, primarily due to the phenomenon of critical slowing down. Interestingly, in both formalisms, there exist cases where the metric does not diverge at the transition.

Finally, we apply the Sivak metric to determine the optimal protocol connecting two points in different phases of the mean-field antiferromagnet. Because the two phases are entirely separated by a phase boundary, any such protocol necessarily traverses a phase transition. Our results show that, while the thermodynamic length can diverge for the Sivak metric in certain models, it remains finite for all cases examined under the Ruppeiner metric.
