% \documentclass[%
%  reprint,
% %superscriptaddress,
% %groupedaddress,
% %unsortedaddress,
% %runinaddress,
% %frontmatterverbose, 
% %preprint,
% %preprintnumbers,
% %nofootinbib,
% %nobibnotes,
% %bibnotes,
%  amsmath,amssymb,
%  aps,
% %pra,
% %prb,
% %rmp,
% %prstab,
% %prstper,
% %floatfix,
% ]{revtex4-2}

% For Thesis
\documentclass[pra,preprint, ,superscriptaddress,notitlepage,longbibliography]{revtex4-2}
\usepackage{anyfontsize}
\renewcommand{\normalsize}{\fontsize{12}{14}\selectfont}
\linespread{1.5}
\usepackage[margin=1.5cm]{geometry}
\usepackage{newtxtext,newtxmath}

\usepackage[normalem]{ulem}
\usepackage{graphicx}% Include figure files
\usepackage{dcolumn}% Align table columns on decimal point
\usepackage{bm}% bold math
\usepackage{cancel}
\usepackage{nicefrac}
\usepackage{multirow}
\usepackage[dvipsnames]{xcolor} % for colors 
\usepackage{xspace}
\usepackage[hidelinks]{hyperref}
\usepackage{caption}
\usepackage{subcaption}

%\newcommand{\oren}[1]{{\color{red} [Oren:]\xspace}}%\usepackage{hyperref}% add hypertext capabilities
\newcommand{\oren}[1]{{\color{red} [Oren: #1]\xspace}}
\newcommand{\omer}[1]{{\color{ForestGreen} [Omer: #1]\xspace}}
\newcommand{\arctanh}[0]{\text{arctanh}}
\newcommand{\red}[1]{{\color{red}{#1}}}
\newcommand{\done}[0]{{\color{ForestGreen} \checkmark \quad}}

%\usepackage[mathlines]{lineno}% Enable numbering of text and display math
%\linenumbers\relax % Commence numbering lines

%\usepackage[showframe,%Uncomment any one of the following lines to test 
%%scale=0.7, marginratio={1:1, 2:3}, ignoreall,% default settings
%%text={7in,10in},centering,
%%margin=1.5in,
%%total={6.5in,8.75in}, top=1.2in, left=0.9in, includefoot,
%%height=10in,a5paper,hmargin={3cm,0.8in},
%]{geometry}

\begin{document}

\preprint{APS/123-QED}

\title{Thermodynamic Geometry Through Second Order Phase Transitions}

\author{Omer Michael Basri}
 \email{omer.basri@weizmann.ac.il}
\author{Oren Raz}
 \email{oren.raz@weizmann.ac.il}
\affiliation{%
 Department of Physics of Complex Systems,\\Weizmann Institute of Science, Rehovot 76100, Israel. 
}%

\date{\today}

\begin{abstract}

Thermodynamic geometry provides a powerful framework to analyze the dissipation of physical systems driven out of equilibrium, but near equilibrium.
In its original formulation, the system's relaxation timescales had to be, to good approximation, constant throughout the control parameter space.
In its current form, the formalism requires that linear response theory holds.
We study the thermodynamic geometry of systems that undergo second-order phase transitions. 
Using the scaling hypothesis, we show that while the Ruppeiner metric diverges at the critical point, the thermodynamic length may remain finite.
However, for the Sivak-Crooks formulation of the metric, we show by use of the dynamical scaling hypothesis that the thermodynamic length can diverge in some systems.
We argue, that the length likely converges for higher dimensional systems.
Finally, we compute the Sivak-Crooks metric of the mean-field antiferromagnetic Ising model, and use it to study optimal protocols connecting points of interest in the system.
We show analytically that in the disordered phase, the metric effectively reduces to a 1D manifold, allowing for simple determination of optimal protocols.
In the ordered phase, we numerically find optimal protocols using the Fast-Marching method.

\end{abstract}

\maketitle

\section{Introduction}

Since the earliest developments of thermodynamics, there have been attempts to cast the theory in geometric form \cite{gibbs1873method}. A geometric interpretation exists for equilibrium thermodynamics—for instance, the work performed by a heat engine corresponds to the area enclosed by its stroke in the $PV$ plane. However, an analogous geometric relation for excess dissipation in non-equilibrium processes remains elusive.

One approach that bridges this gap is thermodynamic geometry, introduced in the 1970s by Ruppeiner and Weinhold \cite{ruppeinerThermodynamicsRiemannianGeometric1979, weinholdMetricGeometryEquilibrium1975}. In their formulation, the space of thermodynamic parameters is endowed with a Riemannian metric derived from the second derivatives of a thermodynamic potential—such as the internal energy, entropy, or any of their Legendre transforms. Although the choice of potential changes the explicit form of the metric, the resulting manifold remains invariant \cite{salamonRelationEntropyEnergy1984}.

In the 1980s, Salamon \cite{salamonThermodynamicLengthDissipated1983} established a connection between excess dissipation and thermodynamic length in this Riemannian manifold, subject to two important limitations. First, the result holds only for endoreversible processes, where the system remains in internal equilibrium but may be out of equilibrium with its environment. Second, the dissipation depends explicitly on the mean response time of the system, restricting the formalism’s applicability to systems with varying relaxation times. Despite these limitations, the framework has proven valuable for identifying minimal-entropy-production protocols in processes such as chemical reactions and distillation \cite{andresenCurrentTrendsFiniteTime2011}.

Building on this line of work, Sivak and Crooks \cite{sivakThermodynamicMetricsOptimal2012, crooksMeasuringThermodynamicLength2007} derived a closely related metric using linear response theory. In their formulation, the system’s relaxation dynamics are incorporated directly into the metric, thereby removing the dependence on a separate response-time parameter and eliminating the assumption of endoreversibility. This generalized framework extends thermodynamic geometry to a much broader class of systems—including those with widely varying time scales and microscopic systems such as biomolecular or single-molecule systems—and has found applications in optimizing computational and physical processes \cite{rotskoffGeometricApproachOptimal2017}.

A particularly challenging class of systems exhibiting widely separated time scales are those undergoing phase transitions. The application of thermodynamic geometry to such systems has drawn significant attention \cite{janyszekRiemannianGeometryThermodynamics1989, PhysRevE.51.1006}, including efforts to identify optimal driving protocols across critical regions (e.g., the 2D ferromagnetic Ising model \cite{rotskoffDynamicRiemannianGeometry2015}). Yet, the divergence of thermodynamic length near phase transitions has received limited study. Notably, a divergence in the metric tensor does not necessarily imply a divergence in thermodynamic length—much as the divergence of the Schwarzschild metric at $r=r_s$ does not correspond to a physical singularity.

In systems with disconnected phases, any transformation from one phase to another must necessarily cross a phase transition. One such case, examined in this work, is the mean-field antiferromagnet, where the phase boundary entirely separates distinct thermodynamic states. Determining whether the thermodynamic length remains finite near such transitions is crucial for understanding the applicability of thermodynamic geometry in these regimes.

Both the Ruppeiner and Sivak metrics predict that the thermodynamic metric should diverge in the vicinity of a second-order phase transition (see Sec.~\ref{sec:framework}). A natural question, therefore, is whether this divergence of the metric translates into a divergence of the corresponding thermodynamic lengths. Note that even if the length remains finite, other geometric quantities such as the curvature may diverge, signaling critical behavior. Though these do not directly contribute to dissipation and are thus outside our present focus.

In this paper, we analyze the behavior of thermodynamic geometry near second-order phase transitions. We show that the Ruppeiner and Sivak manifolds exhibit qualitatively distinct behavior around criticality, primarily due to the phenomenon of critical slowing down. Interestingly, in both formalisms, there exist cases where the metric does not diverge at the transition.

Finally, we apply the Sivak metric to determine the optimal protocol connecting two points in different phases of the mean-field antiferromagnet. Because the two phases are entirely separated by a phase boundary, any such protocol necessarily traverses a phase transition. Our results show that, while the thermodynamic length can diverge for the Sivak metric in certain models, it remains finite for all cases examined under the Ruppeiner metric.


\section{Framework} \label{sec:framework}
Before analyzing thermodynamic geometry near second-order phase transitions, we briefly define the key metrics and scaling tools used in our study.

\subsection{Ruppeiner and Weinhold Metric}
The Ruppeiner–Weinhold metric \cite{ruppeinerThermodynamicsRiemannianGeometric1979, weinholdMetricGeometryEquilibrium1975} is defined as the Hessian of a thermodynamic potential or generalized entropy:
\begin{equation}
    g_{\mu\nu}^R = \pm \frac{\partial^2 \varphi}{\partial \lambda^\mu \partial \lambda^\nu}
\end{equation}
where $\varphi$ is a thermodynamic potential (e.g., internal energy, entropy, or a Legendre transform), and $\lambda^\mu$ are its natural variables (e.g., $\{S, M\}$ for $E$).
The sign ensures positive definiteness, consistent with the convexity or concavity of $\varphi$ \cite{Callen:450289}.

Salamon \cite{salamonThermodynamicLengthDissipated1983} connected this metric to excess dissipation through the dissipated availability:
\begin{equation} \label{SalomonDissipatedAvail}
    \mathcal{A} = \bar{\epsilon} \int_{\boldsymbol{\lambda}(s)} g^R_{\mu\nu}\, \dot{\lambda}^\mu \dot{\lambda}^\nu\, ds
\end{equation}
where $\bar{\epsilon}$ is the average response time. The corresponding thermodynamic length is
\begin{equation}
    \mathcal{L} = \int_{\boldsymbol{\lambda}(s)} \sqrt{g_{\mu\nu}\, \dot{\lambda}^\mu \dot{\lambda}^\nu}\, ds
\end{equation}
Minimizing $\mathcal{L}$ minimizes excess dissipation for a fixed-duration protocol, a result of the Cauchy–Schwarz inequality:
\begin{equation}
    \mathcal{A} \ge \frac{\bar{\epsilon}}{\tau}\mathcal{L}^2
\end{equation}
with equality for constant-speed protocols (i.e $\sqrt{g_{\mu\nu}\, \dot{\lambda}^\mu \dot{\lambda}^\nu}=\text{const.}$).

\subsection{Sivak and Crooks Metric}
The Ruppeiner metric assumes endoreversibility and uniform response times, which limit its scope. Sivak and Crooks \cite{sivakThermodynamicMetricsOptimal2012} removed these restrictions using linear-response theory, defining
\begin{equation}
    g_{\mu\nu}^S = \int_0^\infty \langle \delta X_\mu(t)\, \delta X_\nu(0) \rangle\, dt =
    \mathcal{T}_{\mu\nu} \frac{\partial^2 \ln Z}{\partial \lambda^\mu \partial \lambda^\nu}
\end{equation}
where $\mathcal{T}_{\mu\nu}$ is the time-response matrix, $Z$ the partition function, and $X_\mu$ the conjugate variable to $\lambda^\mu$.
The response time is thus encoded directly in the metric, and the dissipated availability becomes
\begin{equation}
    \mathcal{A} = \int_{\boldsymbol{\lambda}(t)} g^S_{\mu\nu}\, \dot{\lambda}^\mu \dot{\lambda}^\nu\, dt
\end{equation}
with the inequality
\begin{equation}
    \mathcal{A} \ge \frac{\mathcal{L}^2}{\tau}
\end{equation}
again saturated by constant-speed protocols.

\subsection{Scaling Hypothesis} \label{subsec:ScalingHypothesisFramework}
Near second order phase transitions, derivatives of the free energy -- and hence the Ruppeiner and Sivak metrics often diverge (for example, via diverging magnetic susceptibility). 
Although exact partition functions are rarely tractable, the asymptotic form of these divergences is determined by critical exponents defining universality classes.

The scaling hypothesis relates the free energy near the critical point $(T_c, H_c)$ to reduced variables $t = (T - T_c)/T_c$ and $h = (H - H_c)/H_c$:
\begin{equation}
    \varphi \propto t^b f\!\left(\frac{h}{t^\Delta}\right),
\end{equation}
where $f(x)$ is continuous and $f(x) \sim x^p$ as $x \to \infty$ \cite{Kardar_2007}.
The exponents $b$, $\Delta$, and $p$ can be determined by standard critical exponents, e.g., $b = 2 - \alpha$ from the heat capacity exponent $\alpha$.
This approach captures metric divergences across universality classes but does not address relaxation times, which are critical the dissipation in the Ruppiner formalism, and for the definition of the Sivak metric.

\subsubsection*{Dynamic Scaling Hypothesis}
Critical slowing down near second-order transitions leads to divergent relaxation times. The dynamic scaling hypothesis relates the relaxation time $\tau$ and correlation length $\xi$ as
\begin{equation}
    \tau \propto \xi^z
\end{equation}
and the free energy scales as $\varphi \propto \xi^{-d}$ \cite{Kardar_2007, tauberCriticalDynamicsField2014}.
Combining these yields
\begin{equation}
    \tau \propto \varphi^{-z/d} \propto t^{-bz/d}\, f^{-z/d}\!\left(\frac{h}{t^\Delta}\right)
\end{equation}
linking dynamical and thermodynamic critical behavior.

\subsection{Divergence of Lengths}
A diverging metric does not necessarily imply a divergent thermodynamic length -- analogous to how the Schwarzschild metric diverges at $r = r_s$ without a corresponding geometric singularity. 
For constant-speed protocols of fixed duration,
\begin{equation}
    \mathcal{A} \propto \mathcal{L}^2
\end{equation}
Thus, a finite thermodynamic length guarantees the existence of a finite-dissipation protocol. As opposed to that, from the Cauchy-Schwarz inequality $\mathcal{L} \to \infty$ implies $\mathcal{A} \to \infty$.

Consequently, the convergence of thermodynamic length determines whether finite-dissipation protocols exist. 
If $\ell(s) = \sqrt{g_{\mu\nu}\dot{\lambda}^\mu\!\left(s\right)\dot{\lambda}^\nu\!\left(s\right)} \propto s^{-p}$ near criticality (as $s \rightarrow 0$), the total length diverges only for $p > 1$. 

Note that this observation does not necessarily inform  us about the metric itself, or intrinsic geometry of the manifold. The metric might diverge for flat space \cite{PhysRev.119.1743,Szekeres:1960gm}, and the manifold might have divergent curvature despite having finite lengths and a non-divergent metric. 


\section{Results}

\subsection{Scaling Hypothesis} \label{subsec:ScalingHypothesisResults}

\subsubsection{Scaling Hypothesis for the Free Energy}
We employ the scaling hypothesis for the free energy to analyze the divergence of the Ruppeiner 
metric and determine whether the thermodynamic length diverges along paths crossing a phase transition.

Assuming the singular part of the thermodynamic potential potential scales as
\begin{equation}
    \varphi \propto t^b f\!\left(\frac{h}{t^\Delta}\right)
\end{equation}
the metric near criticality ($t,h \to 0$) scales as
\begin{align}
    \left(g_{\mu\nu}\right) \Big|_{\frac{h}{t^\Delta} \to \text{const.}} &\propto 
    \begin{pmatrix}
        t^{b-2} & t^{b-\Delta-1} \\
        t^{b-\Delta-1} & t^{b-2\Delta}
    \end{pmatrix}, \\[4pt]
    \left(g_{\mu\nu}\right) \Big|_{\frac{h}{t^\Delta} \to \infty} &\propto
    \begin{pmatrix}
        t^{-2}h^p & t^{-1}h^{p-1} \\
        t^{-1}h^{p-1} & h^{p-2}
    \end{pmatrix}.
\end{align}
Here, the two limits correspond to different asymptotic behaviors of $f$ as $\frac{h}{t^\Delta}$ tends to a constant or diverges.

To study length divergences, we must specify the path in parameter space, which determines the scaling of $\frac{h}{t^\Delta}$.  
We consider two representative cases.

\paragraph{Case 1: $\frac{h}{t^\Delta} \to \text{const.}$ \label{par:scalingCase_1}}
\begin{equation}
    \begin{pmatrix}
        t \\ h
    \end{pmatrix} =
    \begin{pmatrix}
        s \\ s^k
    \end{pmatrix}, \quad k \ge \Delta
\end{equation}
for which
\begin{equation*}
    \frac{h}{t^\Delta} \xrightarrow[s \to 0]{} \text{const.}
\end{equation*}
The metric contributions then scale as
\begin{align*}
    g_{tt} \dot{t}^2 &\propto s^{b-2} \\
    g_{th} \dot{t}\dot{h} &\propto s^{b-2+(k-\Delta)} \\
    g_{hh} \dot{h}^2 &\propto s^{b-2+2(k-\Delta)}
\end{align*}
Since $k \ge \Delta$, the sufficient condition for convergence of the Ruppeiner length $\mathcal{L}$ is
\begin{equation}
    b > 0
\end{equation}
As established earlier, convergence of $\mathcal{L}$ is necessary and sufficient for convergence of the Salamon dissipated availability $\mathcal{A}$ in finite-time protocols, provided $\bar{\epsilon}$ remains finite.

\paragraph{Case 2: $\frac{h}{t^\Delta} \to \infty$ \label{par:scalingCase_2}}
\begin{equation}
    \begin{pmatrix}
        t \\ h
    \end{pmatrix} =
    \begin{pmatrix}
        s^{1/k} \\ s
    \end{pmatrix}, \quad k < \Delta
\end{equation}
implying
\begin{equation*}
    \frac{h}{t^\Delta} \xrightarrow[s \to 0]{} \infty
\end{equation*}
In this case,
\begin{align*}
    g_{tt} \,\dot{t}^2 &\propto s^{p-2} &
    g_{th} \,\dot{t}\,\dot{h} &\propto s^{p-2} &
    g_{hh} \,\dot{h}^2 &\propto s^{p-2}
\end{align*}
Hence, convergence requires
\begin{equation}
    p > 0
\end{equation}
Both conditions ($b > 0$ and $p > 0$) hold for several universality classes, as shown in Table~\ref{tab:scaling-divergences}.

\begin{table}[h]
    \centering
    \begin{tabular}{c||c|c}
        & $b$ & $p$ \\
        \hline \hline
        Mean Field & $2$ & $\tfrac{4}{3}$ \\
        2D Ising & $2$ & $\tfrac{16}{15}$ \\
        3D Ising & $1.8899(3)$ & $1.2088(1)$ \\
        2D 3-state Potts & $\tfrac{5}{3}$ & $\tfrac{15}{14}$ \\
        2D 4-state Potts & $\tfrac{4}{3}$ & $\tfrac{16}{15}$ \\
        % 3D XY Model & $ 2.0153(3) $ & $ 1.20923(6) $
    \end{tabular}
    \caption{Scaling parameters for representative universality classes.}
    \label{tab:scaling-divergences}
\end{table}

\subsubsection{Dynamical Scaling Hypothesis}
In both Ruppeiner and Sivak formalisms, dissipation depends on the system’s response time—appearing as $\bar{\epsilon}$ in the former and directly in the metric for the latter.  
Using the dynamical scaling hypothesis, we can study how the relaxation time $\tau$ diverges.

From $\tau \propto \psi^{-z/d}$ and the static scaling relation, we obtain
\begin{equation}
    \tau \propto t^{-bz/d}\, w\!\left(\frac{h}{t^\Delta}\right)
\end{equation}
where $w(x)$ is continuous and satisfies
\begin{equation}
    w(x) \xrightarrow[x\to\infty]{} x^{-pz/d}
\end{equation}
Hence, the divergence of $\tau$ follows
\begin{equation}
    \tau \propto
    \begin{cases}
        t^{-bz/d}, & \frac{h}{t^\Delta} \to \text{const.} \\[4pt]
        h^{-pz/d}, & \frac{h}{t^\Delta} \to \infty
    \end{cases}
\end{equation}

Including this factor modifies the scaling of the metric elements in \hyperref[par:scalingCase_1]{Case 1}
\begin{align}
    g_{tt}\,\dot{t}^2 &\propto s^{b(1-z/d)-2} \\
    g_{th}\,\dot{t}\,\dot{h} &\propto s^{b(1-z/d)-2+(k-\Delta)} \\
    g_{hh}\,\dot{h}^2 &\propto s^{b(1-z/d)-2+2(k-\Delta)}
\end{align}
Thus, the convergence condition becomes
\begin{equation}
    b\!\left(1 - \frac{z}{d}\right) > 0
\end{equation}
Similarly, in \hyperref[par:scalingCase_2]{Case 2}
\begin{equation}
    g_{ij}\dot{x}^i\dot{x}^j \propto s^{p(1-z/d)-2}
\end{equation}
so convergence requires
\begin{equation}
    p\!\left(1 - \frac{z}{d}\right) > 0
\end{equation}

As summarized in Table~\ref{tab:dynamical-divergences}, the thermodynamic lengths diverge for the 2D Ising and 2D Potts models but remain finite for the 3D Ising model.

\begin{table}[h]
    \centering
    \begin{tabular}{c|c||c|c|c}
        \multicolumn{2}{c||}{} & $d$ & $z$ & $z/d$ \\
        \hline \hline
        \multirow{2}{*}{2D Ising} & Class A & $2$ & $2.165(10)$ & $1.083(5)$ \\
                                  & Class B & $2$ & $2.235(10)$ & $1.117(5)$ \\
        \hline
        \multicolumn{2}{c||}{3D Ising} & $3$ & $2.032(4)$ & $0.677(1)$ \\
        \hline
        \multicolumn{2}{c||}{2D 3-state Potts} & $2$ & $2.198(2)$ & $1.099(1)$ \\
        \hline
        \multicolumn{2}{c||}{2D 4-state Potts} & $2$ & $2.290(3)$ & $1.095(2)$ \\
    \end{tabular}
    \caption{Dynamical scaling parameters for representative universality classes \cite{odor_universality_2004}.}
    \label{tab:dynamical-divergences}
\end{table}


\subsection{Antiferromagnetic Mean-Field Model - Ruppiner metric}
As was mentioned before, in the Antiferromagnet it is impossible to build a path that crosses between the antiferromagnetic and disordered phases without passing through a second order phase transition. Therefore, it is a prime example of a system where our analysis is necessary.

We demonstrate finding the path of least dissipation between two points $\left(T_1, h_1\right) \rightarrow \left(T_2, h_2\right)$, where $\left(T_1, h_1\right)$ is in the antiferromagnetic phase, and $\left(T_2, h_2\right)$ is in the disordered phase. We use the Ruppiner metric, as it is analytically solvable.

\subsubsection{The Model}
The model we use is the Antiferromagnetic Mean-Field model, as laid out in \cite{vivesUnifiedMeanfieldStudy1997}. It consists of spins on a square lattice, divided into two sub-lattices, such that spins interact only with spins from the other sub-lattice. Denoting the magnetization of each sub-lattice by $m_A,m_B$, the Gibbs Free energy is:
\begin{equation}
    f = \frac{1}{2} \left[ K m_A m_B - h \left( m_A + m_B \right) + \frac{1}{2} T \left( \vartheta \left(m_A\right) +\vartheta\left(m_B\right) \right) \right]
\end{equation}
With 
\begin{equation}
    \vartheta \left(x\right) = \frac{1}{2} \left[ \left(1+x\right) \log \left(1+x\right) + \left(1-x\right) \log \left(1-x\right) \right]
\end{equation}
We define $K$ to be the interaction strength $J$ times the coordination number.

In thermodynamic equilibrium $m_A,m_B$ are the minimizers of $f$, and solve the self consistent equations
\begin{align}
    \label{eq:SCExy}
    m_A = & \tanh\left(\frac{h-Km_B}{T}\right) \\
    m_B = & \tanh\left(\frac{h-Km_A}{T}\right)
\end{align}

Taking the derivative gives the Ruppiner metric, since this is the gibbs free energy, is done w.r.t $T, h$, and gives:
\begin{widetext}

    \begin{equation}
        g_{hh} = \frac{T \left(2-m_A^2-m_B^2\right) - 2 K \left(1-m_A^2\right) \left(1-m_B^2\right)}
        {2 \left[ T^2- \left(1-m_A^2\right) \left(1-m_B^2\right) K^2\right]}
    \end{equation}
    \begin{equation}
        g_{hT} = - \frac{\left(1-m_A^2\right) \arctanh(m_A) \left(T-\left(1-m_B^2\right) K\right)+\left(1-m_B^2\right) \arctanh (m_B) \left(T-\left(1-m_A^2\right) K\right)}{2 \left[ T^2- \left(1-m_A^2\right) \left(1-m_B^2\right) K^2\right]}
    \end{equation}
    \begin{equation}
        g_{TT} = \frac{T \left[ \arctanh^2( m_B ) \left(1-m_B^2\right) + \arctanh^2( m_A ) \left(1-m_A^2\right)\right] - K \left( 1-m_A^2 \right) \left(1-m_B^2\right) \arctanh(m_B) \arctanh(m_A) 
        }{2 \left[T^2- \left(1-m_A^2\right) \left(1-m_B^2\right) K^2\right]}
    \end{equation}
\end{widetext}
% With the determinant of the metric being
% % \begin{widetext}
%     \begin{equation}
%         g = \frac{\left(1-m_A^2\right) \left(1-m_B^2\right) \left(\arctanh(m_A)-\arctanh(m_B)\right)^2}{4 \left(T^2-\left(1-m_A^2\right) \left(1-m_B^2\right) (K)^2\right)}
%     \end{equation}
% % \end{widetext}

\subsubsection{Disordered Phase}
One interesting property that can be readily observed is that in the disordered phase, where $m_A=m_B\equiv m$, one gets that $g=0$, i.e the determinant of the metric vanishes. This has been observed before in the Ferromagnetic mean-field model \cite{janyszekRiemannianGeometryThermodynamics1989}. It can be shown that the 0 eigenvectors in the disordered phase point along lines of constant magnetization (or equivalently lines of constant $m_A$). As a result, the problem becomes a 1D minimization problem - at what distance from the phase transition line will the length be minimal? As we show below, the answer is simply at the phase transition line itself.

In the following section we will assume when writing $T_c$ that it is the critical temperature for a specific magnetization. Two central results that can be easily proven by derivation of the free energy are that on the phase transition line
\begin{equation}
    m_A=m_B=\pm \sqrt{1-\frac{T_c}{K}}
\end{equation}
And
\begin{equation} \label{eq:h_c(T_c)}
    h_c (T_c) = \pm\left(K\sqrt{1-\frac{T_c}{K}} + T_c \arctanh\left( \sqrt{1-\frac{T_c}{K}} \right) \right)
\end{equation}


The metric becomes in the disordered phase ($m_A=m_B$)
\begin{equation}
    g_{ij} = \frac{T_c}{K \left(T+T_c\right)}
    \left( \begin{array}{cc}
         \arctanh(m)^2 & -\arctanh(m) \\
         -\arctanh(m) & 1 \\
    \end{array} \right)
\end{equation}
The eigenvalues and eigenvectors of this metric are
\begin{align}
    \lambda_1 & = 0,\quad \boldsymbol{v}_1 = \left( \begin{array}{c}
         1  \\
        \arctanh(m_A)
    \end{array} \right) \\
    \lambda_2 & = \arctanh^2(m_A)+1,\quad \boldsymbol{v}_2 = \left( \begin{array}{c}
        \arctanh(m_A) \\
         -1 
    \end{array} \right) 
\end{align}
One can see an illustration of the "0-length" lines in Fig. (\ref{fig:0length}).

\begin{figure}
    \centering
    \includegraphics[width=1\linewidth]{figures/disorderedPhaseLines.jpg}
    \caption{lines of "0-length" in disordered phase}
    \label{fig:0length}
\end{figure}

And so we need to minimize the expression
$ \sqrt{\dot{\lambda}^\mu g_{\mu\nu} \dot{\lambda}^\nu} \,dt$
along the $\boldsymbol{v}_1$ axis. We know $g_{\mu\nu}$ scales as $\left(T+T_c\right)^{-1}$. We will argue that the rest of the expression scales as $T-T_0$ for some $T_0$.

\omer{It feels to me like the next part could be made more precise and clear with some statement of curvature, this is just how I  calculated it originally, but if you can formalize it better I would be happy.}
The rest of the expression is simply euclidean distance projected onto the normal to the lines of constant magnetization. Hence, it should scale as the radial distance from the point of meeting of two adjacent lines of constant magnetization.
The radial distance, since the lines aren't vertical or horizontal (except for one) is proportional to the distance along the $T$-axis.

Taking some point $\left(T_c,h_c\right)$ and an adjacent point on the phase transition line
\begin{equation*}
    (T_c+\delta_T, h_c + \frac{\partial h_c}{\partial T_c}\delta_T)
\end{equation*}
We want the intersection of the lines (i.e, to solve for $T_0$):
\begin{align} \label{eq:T0,h0}
    \left( \begin{array}{c}
         T_0 \\
         h_0
    \end{array}\right) & = a \, \boldsymbol{v}_1\left(T_c\right) + \left( \begin{array}{c}
         T_c \\
         h_c
    \end{array}\right) \\
    \left( \begin{array}{c}
         T_0 \\
         h_0
    \end{array}\right) & = b \, \boldsymbol{v}_1\left(T_c + \delta_T \right) + \left( \begin{array}{c}
         T_c + \delta_T \\
         h_c + \frac{\partial h_c}{\partial T_c}\delta_T
    \end{array}\right)
\end{align}

We first subtract the equations to get
\begin{equation}
    \delta_T \left( \begin{array}{c}
         1  \\
         \frac{\partial h_c}{\partial T_c}
    \end{array}\right) = \left( \begin{array}{cc}
        1 & 1 \\
        \arctanh(m(T_c+ \delta_T)) & \arctanh(m(T_c))
    \end{array} \right)
    \left( \begin{array}{c}
         -b  \\
         a
    \end{array}\right)
\end{equation}
Inverting the matrix gives:
\begin{widetext}
\begin{equation}
    \begin{array}{rl}
        \left( \begin{array}{cc}
            1 & 1 \\
            \arctanh(m(T_c+\delta_T)) & \arctanh(m(T_c))
        \end{array} \right)^{-1} & = \frac{1}{\arctanh\left(m(T_c)\right) - \arctanh\left(m(T_c + \delta_T)\right)}\left( \begin{array}{cc}
            \arctanh(m(T_c)) & -1 \\
            -\arctanh(m(T_c + \delta_T)) & 1
        \end{array} \right) \\ & = -\frac{2T_c \sqrt{1-\frac{T_c}{K }}}{K  \, \delta_T}\left( \begin{array}{cc}
            \arctanh\left(\sqrt{1-\frac{T_c}{K }}\right) & -1 \\
            -\arctanh\left(\sqrt{1-\frac{T_c + \delta_T}{K }}\right) & 1
        \end{array} \right)
    \end{array}
\end{equation}
giving
\begin{equation} \label{eq:a}
    a = \frac{2}{K }\arctanh\left(\sqrt{1-\frac{T_c}{K }}\right) T_c \sqrt{1-\frac{T_c}{K }} - 
    \frac{2}{K } T_c \sqrt{1-\frac{T_c}{K }} \frac{\partial h_c}{\partial T_c}
\end{equation}
\end{widetext}

Deriving equation \ref{eq:h_c(T_c)} one gets
\begin{equation}
    \frac{\partial h_c}{\partial T_c} = -\frac{1}{\sqrt{1-\frac{T_c}{K}}} + \arctanh\sqrt{1-\frac{T_c}{K }}
\end{equation}

And substituting this into equation \ref{eq:a} gives
\begin{equation}
    a = \frac{2 T_c}{ K }
\end{equation}

By substituting this result into \ref{eq:T0,h0} we get
\begin{equation}
    T_0 = T_c + \frac{2}{K} T_c = \frac{K + 2}{K} T_c
\end{equation}

Accordingly, We get that the differential length between two different points in the disordered phase with different magnetization scales as
\begin{equation}
    \frac{\left(T + \frac{K + 2}{K} T_c\right)^2}{T+T_c}
\end{equation}
the minimum in the disordered phase will be attained at $T=T_c$.
% \subsubsection{Antiferromagnetic Phase}

As previously mentioned, in the anti-ferromagnetic phase the metric is non-degenerate. Hence, the problem of finding shortest paths is 2D. We used numeric methods to find the shortest paths from a given point to different points on the phase transition line.

Calculation of the shortest path to the phase transition line from the inner point was done using the fast marching method\cite{kimmelComputingGeodesicPaths1998}.
% Some more text and graphs once I finish the calculations.
 
\subsection{Antiferromagnetic Mean-Field Model - Sivak metric}
\omer{I wrote this assuming we will only publish this, and not the Ruppiner metric}


We now determine the least-dissipation path between two states 
$\left(T_1,h_1\right)\!\to\!\left(T_2,h_2\right)$,
with $\left(T_1,h_1\right)$ in the antiferromagnetic phase and $\left(T_2,h_2\right)$ in the disordered phase. 
In particular, we compute the Sivak metric for the MF antiferromagnetic model of \cite{vivesUnifiedMeanfieldStudy1997}.

\subsubsection{The Model}
The Massieu potential is
\begin{widetext}
\begin{equation}
    \psi \;=\ln Z=\; -\frac{1}{2}\!\left[ \beta K\, m_A m_B \;-\; \beta h\, (m_A+m_B)
    \;+\; \frac{1}{2}\,\big(\vartheta(m_A)+\vartheta(m_B)\big) \right]
\end{equation}
\end{widetext}
where $\beta=1/T$.

\subsubsection{Calculation of the Metric}
The computation proceeds in three steps:
\begin{enumerate}
    \item \textbf{Equilibrium covariances:} evaluate 
    $\langle \delta m_i(0)\,\delta m_j(0)\rangle$ ($i,j\in\{A,B\}$).
    \item \textbf{Dynamics and time integral:} incorporate linearized relaxation to obtain the integrated correlation.
    \item \textbf{Variable transform:} convert to the $\boldsymbol{\lambda}=(\beta,\beta h)\equiv(\beta,\alpha)$ representation, which is the representation of our control parameters, via the Jacobian from $(m_A,m_B)$ to $(-E,m)$.
\end{enumerate}
Note we set $\alpha \equiv \beta h$ for brevity.

Let $\zeta_i\equiv 1-m_i^2$ for $i\in\{A,B\}$. 
The inverse covariance is the Hessian of $\psi$ with respect to $(m_A,m_B)$:
\begin{equation}
    -\Sigma^{-1} \;=\; \frac{1}{2}\!
    \begin{pmatrix}
        \dfrac{1}{\zeta_A} & \beta K \\
        \beta K & \dfrac{1}{\zeta_B}
    \end{pmatrix}.
\end{equation}
Its inverse is
\begin{equation}
    \Sigma \;=\; \frac{2}{\,1-\beta^2 K^2 \zeta_A \zeta_B\,}
    \begin{pmatrix}
        \zeta_A & -\beta K\,\zeta_A\zeta_B \\
        -\beta K\,\zeta_A\zeta_B & \zeta_B
    \end{pmatrix}
\end{equation}

The relaxation dynamics \cite{klichMpembaIndexAnomalous2019} are
\begin{align}
    \dot m_A &= \Gamma\!\left[\tanh\!\left(\alpha - \beta K\, m_B\right) - m_A\right]\\
    \dot m_B &= \Gamma\!\left[\tanh\!\left(\alpha - \beta K\, m_A\right) - m_B\right]
\end{align}
with microscopic attempt rate $\Gamma$. 
Linearizing about equilibrium gives
\begin{align}
    \delta \dot m_A &= \Gamma\left(-\,\delta m_A \;-\; \beta K\,\zeta_A\,\delta m_B\right)\\
    \delta \dot m_B &= \Gamma\left(-\,\beta K\,\zeta_B\,\delta m_A \;-\; \delta m_B\right)
\end{align}
In vector form, with $y=(\delta m_A,\delta m_B)^{\!\top}$
\begin{equation}
    \dot y \;=\; -\Gamma 
    \begin{pmatrix}
        1 & \beta K\,\zeta_A \\
        \beta K\,\zeta_B & 1
    \end{pmatrix} y \;=\; -M y
\end{equation}

We require
\begin{equation}
    \beta \int_{0}^{\infty}\!\langle \delta m_i(0)\,\delta m_j(t)\rangle\,dt
    \;=\; \beta \int_0^\infty \!\langle y(0)\,y^{\!\top}(t)\rangle\,dt
\end{equation}
Since $\langle \delta m_i(0)\,\eta(t)\rangle=0$ for $t>0$, substituting $y(t)=e^{-Mt}y(0)$ yields
\begin{widetext}
\begin{equation}
\begin{aligned}
    \beta \int_0^\infty \!\langle y(0)\,y^{\!\top}(t)\rangle\,dt
    &= \beta\,\langle y(0)\,y^{\!\top}(0)\rangle \int_0^\infty \!e^{-M^{\!\top}t}\,dt
     \;=\; \beta\,\Sigma\,(M^{\!\top})^{-1} \\[3pt]
    &= \frac{2\beta}{\Gamma\big(1-\beta^2 K^2 \zeta_A \zeta_B\big)^{\!2}}
    \begin{pmatrix}
        \zeta_A\big(1+\beta^2 K^2 \zeta_A \zeta_B\big) & -\,2\beta K\,\zeta_A\zeta_B\\
        -\,2\beta K\,\zeta_A\zeta_B & \zeta_B\big(1+\beta^2 K^2 \zeta_A \zeta_B\big)
    \end{pmatrix}\!
\end{aligned}
\end{equation}
\end{widetext}

To express the Sivak metric in the control variables $\boldsymbol{\lambda}=(\beta,\alpha)$, we transform fluctuations to $(-E,m)$ via
\begin{align}
    E &= \tfrac{1}{2} K\, m_A m_B\\
    m &= \tfrac{1}{2}(m_A+m_B)
\end{align}
so the Jacobian is
\begin{equation}
    P \;=\; \frac{\partial(-E,m)}{\partial(m_A,m_B)} 
    \;=\; \frac{1}{2}
    \begin{pmatrix}
        -K m_B & -K m_A \\
        \;1 & \;1
    \end{pmatrix}
\end{equation}
The Sivak metric then reads
\begin{widetext}
\begin{equation}
\resizebox{\columnwidth}{!}{$
\begin{aligned}
    g^{\text{Sivak}}_{\mu\nu}
    \;= & \; \beta\, P\,\Sigma\,(M^{\!\top})^{-1} P^{\!\top} \\
    \;= & \; \frac{\beta}{2\Gamma\big(1-\beta^2 K^2 \zeta_A \zeta_B\big)^{\!2}}
    \begin{pmatrix}
    \scriptstyle K^2 \!\left[\!\big(1+\beta^2 K^2 \zeta_A \zeta_B\big)\big(m_A^2 \zeta_B + m_B^2 \zeta_A\big) - 4\beta K\, m_A m_B \zeta_A \zeta_B \!\right]
    &
    \scriptstyle -K \!\left[\!\big(1+\beta^2 K^2 \zeta_A \zeta_B\big)\big(m_A \zeta_B+m_B \zeta_A\big) - 2\beta K (m_A+m_B)\zeta_A\zeta_B \!\right]
    \\[6pt]
    \scriptstyle -K \!\left[\!\big(1+\beta^2 K^2 \zeta_A \zeta_B\big)\big(m_A \zeta_B+m_B \zeta_A\big) - 2\beta K (m_A+m_B)\zeta_A\zeta_B \!\right]
    &
    \scriptstyle \big(1+\beta^2 K^2 \zeta_A \zeta_B\big)(\zeta_A+\zeta_B) - 4\beta K\,\zeta_A\zeta_B
    \end{pmatrix}\!
\end{aligned}
$}
\end{equation}
\end{widetext}

\begin{figure}[h]
    \centering
    \begin{subfigure}[b]{0.45\textwidth}
        \centering
        \includegraphics[width=0.9\textwidth]{figures/sivakmetric_bb.png}
        \caption{Caloric element $g_{\beta\beta}$}
    \end{subfigure}%\hfill
    
    \begin{subfigure}[b]{0.45\textwidth}
        \centering
        \includegraphics[width=0.9\textwidth]{figures/sivakmetric_ab.png}
        \caption{Magneto–caloric element $g_{\alpha\beta}$}
    \end{subfigure}
    % \vspace{0.75ex}
    \begin{subfigure}[b]{0.45\textwidth}
        \centering
        \includegraphics[width=0.9\textwidth]{figures/sivakmetric_aa.png}
        \caption{Magnetic element $g_{\alpha\alpha}$}
    \end{subfigure}
    \caption{Elements of the Sivak metric for the MF antiferromagnet}
    \label{fig:sivakmetric_antiferro}
\end{figure}

Figures~\ref{fig:sivakmetric_antiferro} show the resulting metric components.


\subsubsection{Disordered phase} \label{subsec:SivakDisorderedPhase}
It is convenient to define the total and staggered magnetizations
\begin{align}
    m &= \frac{m_A+m_B}{2}\\
    s &= \frac{m_A-m_B}{2}
\end{align}

For each $m\in[-1,1]$, the corresponding critical parameters $(\beta_c,\alpha_c)$ on the phase boundary satisfy
\begin{align}
    m &= \pm\sqrt{1-\frac{1}{\beta_c K}} \label{m(beta)}\\
    \alpha_c &= \pm\Big(\beta_c K\, m + \arctanh m\Big) \label{alpha(beta,m)}
\end{align}

In the disordered phase, $m_A=m_B\equiv m$ (hence $\zeta_A=\zeta_B\equiv \zeta$). 
The metric in this case reduces to
\begin{equation} \label{eq:disorderedmetric}
    g_{\mu\nu} \;=\; \frac{\beta\,\zeta}{\Gamma\bigl(1+\beta K \zeta\bigr)^2}
    \begin{pmatrix}
        K^2 m^2 & -K m \\[2pt]
        -K m & 1
    \end{pmatrix}.
\end{equation}
As in the ruppiner case, the metric is rank-one, with the null eigenvector aligning with directions of constant $m$.
The problem of minimizing distances again reduces to a 1D minimization: at what distance from the phase boundary does the length become minimal? 
This will be calculated in the following paragraphs:

For each $m\in[-1,1]$, the corresponding critical parameters $(\beta_c,\alpha_c)$ on the phase boundary satisfy
\begin{align}
    m &= \pm\sqrt{1-\frac{1}{\beta_c K}} \label{m(beta)}\\
    \alpha_c &= \pm\Big(\beta_c K\, m + \arctanh m\Big) \label{alpha(beta,m)}
\end{align}
The eigenpairs of the metric \ref{eq:disorderedmetric} are
\begin{align}
    \lambda_1 &= 0 \quad :
    & \boldsymbol{v}_1 &= \begin{pmatrix} 1 \\ K m \end{pmatrix}
    \\
    \lambda_2 &= 1+K^2 m^2 \quad:
    & \boldsymbol{v}_2 &= \begin{pmatrix} K m \\ -1 \end{pmatrix}
\end{align}
Thus the zero mode $\boldsymbol{v}_1$ follows lines of constant $m$ in $(\beta,\alpha)$.

\paragraph{Scaling of the metric prefactor.}
From \eqref{m(beta)} we have $\zeta=1-m^2=1/(\beta_c K)$ along the phase boundary labeled by magnetization $m$, with critical parameter $\beta_c$.
Using this in the prefactor gives
\begin{equation}
    g_{\mu\nu} \;\propto\; 
    \frac{\beta}{\bigl(\beta_c+\beta\bigr)^2}\,\bigl(1+K^2 m^2\bigr).
\end{equation}

\paragraph{Scaling of the displacement along the zero mode.}
Consider two neighboring constant-$m$ lines labeled by 
$(\beta_{c1},\alpha_{c1})$ and $(\beta_{c2},\alpha_{c2})=(\beta_{c1}+\delta_\beta,\alpha_{c1}+\delta_\alpha)$.
We find their intersection with a straight line parallel to the zero mode at each point:
\begin{align}
    \begin{pmatrix}\beta\\ \alpha\end{pmatrix}
    &= a\,\boldsymbol{v}_1(\beta_{c1}) + \begin{pmatrix}\beta_{c1}\\ \alpha_{c1}\end{pmatrix}
    \label{eq:b0a0-1}\\[2pt]
    \begin{pmatrix}\beta\\ \alpha\end{pmatrix}
    &= b\,\boldsymbol{v}_1(\beta_{c2}) + \begin{pmatrix}\beta_{c1}+\delta_\beta\\ \alpha_{c1} + \dfrac{\partial \alpha_c}{\partial \beta_c}\,\delta_\beta \end{pmatrix}
    \label{eq:b0a0-2}
\end{align}
Subtracting \eqref{eq:b0a0-1} from \eqref{eq:b0a0-2} yields
\begin{equation}
    \delta_\beta 
    \begin{pmatrix} 1 \\[2pt] \dfrac{\partial \alpha_c}{\partial \beta_c} \end{pmatrix}
    =
    \begin{pmatrix}
        1 & 1 \\
        K m_2 & K m_1
    \end{pmatrix}
    \begin{pmatrix} -b \\[2pt] a \end{pmatrix}
\end{equation}
where $m_1=m(\beta_{c1})$, $m_2=m(\beta_{c2})$.
From \eqref{alpha(beta,m)} we have 
\(
\dfrac{\partial \alpha_c}{\partial \beta_c} = \dfrac{1}{m}
\)
Inverting gives
\begin{equation}
\begin{aligned}
    a \;&=\; \frac{1-m^2}{K m}\,\frac{\partial m}{\partial \beta_c}
      \;=\; \frac{1-m^2}{K m}\,\frac{2m}{(1-m^2)^2}\\
      \;&=\; \frac{2}{K(1-m^2)}
      \;=\; 2\beta_c
\end{aligned}
\end{equation}
where we used $1-m^2=1/(\beta_c K)$ and $\dfrac{\partial m}{\partial \beta_c} = \dfrac{2m}{(1-m^2)^2}$ from \eqref{m(beta)}.
Hence the displacement along the zero mode scales as 
\(
\delta\boldsymbol{\lambda}\sim (\beta+\beta_c).
\)

\paragraph{Conclusion: minimum at $\beta=0$.}
Combining the prefactor and displacement scalings,
\[
\sqrt{\delta\lambda^\mu\, g_{\mu\nu}\, \delta\lambda^\nu}
\;\propto\;
\sqrt{\beta}
\]
so the infinitesimal thermodynamic length is minimized at $\beta=0$ (infinite temperature).
Therefore, the optimal trajectory between two points in this phase follows the null direction (of constant magnetization) from the initial point to $\beta=0$, then traverse along $\beta=0$, then return along a null trajectory to the final point. 

This is opposite to our result for the Ruppiner metric, where the optimal trajectory is along the phase transition line. Meaning, taking into account the divergence in response time substantially changes the optimal trajectory in this model.
\subsubsection{Divergence at the Phase transition}

In the ordered phase, near the critical line, the sublattice magnetizations can be expanded as shown in \cite{vivesUnifiedMeanfieldStudy1997}:
\begin{align}
    m &\approx \pm\sqrt{1-\frac{1}{\beta_c K}} + a_m\,\Delta\alpha - b_m\,\Delta\beta \\
    s &\approx \sqrt{a_s\,\Delta\alpha + b_s\,\Delta\beta}
\end{align}
where $a_m,b_m,a_s,b_s>0$, and
\begin{align*}
    \Delta\alpha &= \alpha - \alpha_c &
    \Delta\beta &= \beta - \beta_c
\end{align*}
For brevity we define
\begin{align*}
    \Delta_m &= a_m\,\Delta\alpha - b_m\,\Delta\beta &
    \Delta_s &= a_s\,\Delta\alpha + b_s\,\Delta\beta
\end{align*}

To lowest order in these small deviations, the sublattice magnetizations and corresponding susceptibilities satisfy
\begin{align}
    m_{A,B} &\approx \pm\sqrt{1-\frac{1}{\beta_c K}} \;\pm\; \sqrt{\Delta_s} \\
    \zeta_{A,B} &\approx \frac{1}{\beta_c K}\bigl(1 \pm 2\sqrt{\Delta_s}\bigr)
\end{align}

Substituting these expressions into the Sivak metric yields, to leading order,
\[
g_{\mu\nu}\;\propto\;\Delta_s^{-1}
\]
Hence, while the metric diverges as $\Delta_s\!\to\!0$, the associated thermodynamic length remains finite. 

\subsubsection{Minimal Paths}

We identify three distinct classes of minimal (shortest) paths, two of which cross the phase transition line.

\paragraph{Type A: Crossing between phases.}
In the first case, the initial and final states lie in different phases. 
The optimal trajectory proceeds from the initial point in the antiferromagnetic phase to the nearest point on the phase transition line, crosses into the disordered phase, and then ascends to infinite temperature ($\beta = 0$). 
Once at $\beta = 0$, the system moves horizontally along this line—where the thermodynamic length vanishes—until reaching the final magnetization, and finally descends along the constant-magnetization line to the destination point.

\paragraph{Type B: Crossing and returning.}
The second case occurs when both endpoints are in the antiferromagnetic phase, but it is shorter to cross the phase transition than to remain entirely within the ordered region. 
This trajectory can be viewed as a composition of two Type~I paths: the system transitions from the ordered phase to the disordered phase and then back again. 
Because, as shown in Sec.~\ref{subsec:SivakDisorderedPhase}, paths within the disordered phase have zero thermodynamic length, the total length of such a trajectory is dominated by the two crossings.

\paragraph{Type C: Entirely within one phase.}
In the third case, both points are again in the antiferromagnetic phase, but the direct geodesic connecting them is shorter than any path that detours through the disordered region. 
Here, the minimal path lies entirely within the ordered manifold.

An illustration of these three classes of paths is shown in Fig.~\ref{fig:shortestPaths}.
\begin{figure}[h]
    \centering
    \includegraphics[width=0.95\linewidth]{figures/ShortestPaths.png}
    \caption{Illustration of the three types of minimal paths in the mean-field antiferromagnet using the Sivak metric.}
    \label{fig:shortestPaths}
\end{figure}

The thermodynamic lengths within the antiferromagnetic phase were computed using the Fast Marching Method \cite{kimmelComputingGeodesicPaths1998}, following the approach of \cite{rotskoffDynamicRiemannianGeometry2015}. 
Minimal paths were obtained by performing gradient descent along the numerically computed distance function.

\subsection{2D Ising model}
As an example for a system with a diverging Ruppiener metric but  nondiverging distance through the phase transition, let us consider the 2D Ising model on a triangular lattice with nearest neighbors interactions. This model has an exact solution due to Onsager, therefore it is possible to analytically calculate the thermodynamic distance for a trajectory as long as the magnetic field stays zero. This model served to numerically demonstrate trajectories with minimal thermodynamic length in \cite{rotskoffDynamicRiemannianGeometry2015}. Since we are considering  trajectories with a constant external field $h=0$, the distance between two states $T_1$ and $T_2$ is given by
\begin{equation}
    \mathcal{L} = \int_{T_1}^{T_2}\sqrt{g_{TT}}dT
\end{equation}
For the Ruppiener metric, 
\begin{eqnarray}
    g_{TT}&=& \frac{c(T)}{k_{\mathrm B}T^2}\nonumber \\
		&=& \frac{4}{T^2\pi}\big(K\coth2K\big)^2\!\left[K_1(q)-E_1(q)\right]\nonumber\\
		& &-\frac{1-\tanh^2\!2K}{T^2}\left[\frac{\pi}{2}+\frac{2}{\pi}\big(2\tanh^2\!2K-1\big)K_1(q)\right]\nonumber
\end{eqnarray}
where $K=\beta J$, $K_1(\cdot)$ and $E_1(\cdot)$ denote the complete elliptic integrals of the first and second kinds respectively, and $q(K)=\frac{2\sinh(2K)}{\cosh^2(2K)}$ \cite{}.

Near the critical point, the metric can be expand as
\begin{equation}
    g_{TT}(T) \;\sim\; \frac{1}{T_c^{2}}\;\frac{2}{\pi}\left(\frac{2J}{k_{\mathrm B}T_c}\right)^{\!2}\,
		\ln\!\left|1-\frac{T}{T_c}\right|, 
		\qquad T\to T_c^\pm.
\end{equation}
and therefore the integral converges even when $T_c$ is in between $T_1$ and $T_2$. Note that $g_{TT}$ diverges as $\ln|1-T/T_c|$, which implies that the integrals in both $\mathcal{L}$ (Eq. \ref{eq:L-def}) and $\mathcal{A}$ (Eq. \ref{SalomonDissipatedAvail}) converge. Moreover, it is a (local) minimal distance  trajectory. To show that, consider a trajectory that ```bypass'' the critical point from $(t,h)=(-\epsilon,0)$ to $(t,h) = (+\epsilon,0)$ along some trajectory $(t(s), h(s))$. The length along this trajectory is
\begin{eqnarray}
    |\dot{\boldsymbol{\lambda}}|=\sqrt{g_{tt}\dot t^2 + g_{hh}(\dot h + \frac{g_{th}}{g_{hh}}\dot t)^2-\frac{g_{th}^2}{g_{hh}}\dot t^2}
\end{eqnarray}
For $h=0$, $g_{ht}=0$ (from symmetry with respect to $h$
 around $h=0$), so at fixed $\dot t$, the minimum over $\dot h$ is at $\dot h=0$. Any deviation from $\dot h=0$ increases the integral by 
 \begin{eqnarray}
     \Delta|\dot{\boldsymbol{\lambda}}|\approx \frac{g_{hh}\dot h^2}{2\sqrt{g_{tt}}}.
 \end{eqnarray}
This implies that the trajectory along the $h=0$ is a local minimum of the distance function, even though it is not a solution of the geodesic equation due to the singularity of the metric.

%Note, however, that the integral over the Sivak metric does not converge due to the critical slow down.




\bibliography{ThermodynamicGeometry}
\end{document}