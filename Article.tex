\documentclass[%
 reprint,
%superscriptaddress,
%groupedaddress,
%unsortedaddress,
%runinaddress,
%frontmatterverbose, 
%preprint,
%preprintnumbers,
%nofootinbib,
%nobibnotes,
%bibnotes,
 amsmath,amssymb,
 aps,
%pra,
%prb,
%rmp,
%prstab,
%prstper,
%floatfix,
]{revtex4-2}
\usepackage[normalem]{ulem}
\usepackage{graphicx}% Include figure files
\usepackage{dcolumn}% Align table columns on decimal point
\usepackage{bm}% bold math
\usepackage{cancel}
\usepackage{nicefrac}
\usepackage{multirow}
\usepackage[dvipsnames]{xcolor} % for colors 
\usepackage{xspace}
\usepackage[hidelinks]{hyperref}
\usepackage{caption}
\usepackage{subcaption}

%\newcommand{\oren}[1]{{\color{red} [Oren:]\xspace}}%\usepackage{hyperref}% add hypertext capabilities
\newcommand{\oren}[1]{{\color{red} [Oren: #1]\xspace}}
\newcommand{\omer}[1]{{\color{ForestGreen} [Omer: #1]\xspace}}
\newcommand{\arctanh}[0]{\text{arctanh}}
\newcommand{\red}[1]{{\color{red}{#1}}}
\newcommand{\done}[0]{{\color{ForestGreen} \checkmark \quad}}

%\usepackage[mathlines]{lineno}% Enable numbering of text and display math
%\linenumbers\relax % Commence numbering lines

%\usepackage[showframe,%Uncomment any one of the following lines to test 
%%scale=0.7, marginratio={1:1, 2:3}, ignoreall,% default settings
%%text={7in,10in},centering,
%%margin=1.5in,
%%total={6.5in,8.75in}, top=1.2in, left=0.9in, includefoot,
%%height=10in,a5paper,hmargin={3cm,0.8in},
%]{geometry}

\begin{document}

\preprint{APS/123-QED}

\title{Thermodynamic Geometry Through Second Order Phase Transitions}

\author{Omer Michael Basri}
 \email{omer.basri@weizmann.ac.il}
\author{Oren Raz}
 \email{oren.raz@weizmann.ac.il}
\affiliation{%
 Department of Physics of Complex Systems,\\Weizmann Institute of Science, Rehovot 76100, Israel. 
}%

\date{\today}

\begin{abstract}

\end{abstract}

\maketitle

\section{Introduction}

Attempts to geometrize thermodynamic parameter space have existed since the theory's early days \cite{gibbs1873method}. While we have a geometric interpretation for equilibrium thermodynamics, for example the  work performed by a heat engine can be related to the area of a stroke in the PV plane,
% entropy/heat generation in equilibrium thermodynamics - the first law, 
there does not exist an analogous relation for excess, non-equilibrium dissipation. 
One similar relation for excess dissipation in non-equilibrium processes, proposed in the 1970-s, is called thermodynamic geometry. 
Laid out by Ruppiner and Weinhold \cite{ruppeinerThermodynamicsRiemannianGeometric1979, weinholdMetricGeometryEquilibrium1975}, this formalism defines a Riemannian metric on thermodynamic parameter space.
The Riemannian metric, which defines this manifold, is simply the second derivatives of the energy, entropy or their Legendre transforms. Use of different thermodynamic potentials to derive the metric, evidently, still gives the same manifold \cite{salamonRelationEntropyEnergy1984}. 

Salomon \cite{salamonThermodynamicLengthDissipated1983} showed in the 1980s that excess dissipation is related to lengths in this metric space, subject to two major caveats.
Firstly, it was proved only for endoreversible processes. Here, endoreversible denotes processes where the system is in internal thermodynamic equilibrium, but may not be in equilibrium with the surroundings. Secondly, the dissipation contains a term of the mean response time of the system. Meaning, that the formalism does not work well in systems with large variations in response time.
This formalism has been used to calculate minimal entropy production protocols in non-equilibrium thermodynamic systems, for example, in distillation and in chemical reactions \cite{andresenCurrentTrendsFiniteTime2011}. 

In 2012 Sivak and Crooks \cite{sivakThermodynamicMetricsOptimal2012} defined a similar metric using linear response theory. This was a continuation of earlier work by Crooks \cite{crooksMeasuringThermodynamicLength2007}. In their definition, the response time was integrated into the metric itself. Therefore, this formalism removed the need for both the response time, and the assumption of endoreversibility, making it applicable to a much wider variety of systems. This includes systems with a widely varying timescales, as well as microscopic systems (e.g biological or single molecule systems). This framework served as an optimization tool for computing systems \cite{rotskoffGeometricApproachOptimal2017}.

One example of a class of systems with widely varying timescales is systems with phase transitions. Several researchers have shown interest in applying thermodynamic geometry to such systems \cite{janyszekRiemannianGeometryThermodynamics1989, PhysRevE.51.1006}. This includes searching for optimal paths in such systems (e.g the 2D ferromagnetic Ising model \cite{rotskoffDynamicRiemannianGeometry2015}). However, the divergence of the thermodynamic length near phase transitions attracted only limited attention. 
Note that divergence of the metric does not necessarily imply divergence of the thermodynamic length. This is evident from many examples, such as the divergence of the Schwarzschild metric at $r=r_s$. 
% {there has yet to be an exhaustive study of the divergence of the thermodynamic length around phase transitions}
Some systems have entirely disconnected phases, such that changing the system from one state to the other necessitates crossing a phase transition. One such example that we will study in this paper is the Anti-Ferromagnet. 
For these reasons, it is important to know wether the thermodynamic length converges or not in the vicinity of phase transitions to understand the applicability of thermodynamic geometry for systems with phase transitions.

One property that is immediately evident from Ruppiner, and Sivak`s definitions, and is presented in the following part (\ref{sec:framework}), is that the metric should diverge in the vicinity of the phase transition. A question that arises is whether a divergence of the metric translates to divergence of lengths.

Note that if lengths don't diverge, there might still be other signs of divergence in the manifold, such as curvature. However these don't affect the dissipation so we are less interested in them for this study.

In this article we explore thermodynamic geometry in the vicinity of second order phase transitions. We show markedly different behavior of the two manifolds (Ruppiner and Sivak metrics) around phase transitions, stemming from critical slowdown. 
One interesting finding is that in both the Ruppiner and Sivak definitions there are cases where the metric doesn't diverge.

We then apply the framework to the Sivak metric to find the shortest protocol between 2 points that belong to different phases of the Mean-Field Anti-Ferromagnet. 
In this model, the phase transition line completely separates the two phases (unlike, for example, the ferromagnetic Ising model), such that any protocol crossing the phase transition must pass through a phase transition.

We see in our work that the thermodynamic length can diverge in some models with the Sivak metric. In all the models we have checked, the lengths converge for the Rupinner metric.

\section{Framework\label{sec:framework}}
Before discussing the implementation of thermodynamic geometry on second order phase transitions, we first define clearly the metrics laid out before, as well as the tools we  use as part of our analysis.
\subsection{Ruppiner and Weinhold Metric}
The Ruppiner and Weinhold metric \cite{ruppeinerThermodynamicsRiemannianGeometric1979, weinholdMetricGeometryEquilibrium1975} is defined by the second derivatives of the thermodynamic potentials or generalized entropies:
\begin{equation}
    g_{\mu\nu}^{\text{Ruppiner}} \equiv g_{\mu\nu}^R= \pm \frac{\partial^2 \varphi}{\partial \lambda ^\mu \partial\lambda^\nu}, 
\end{equation}
where $ \varphi $ can be the energy, entropy or their Legendre transforms (i.e the Gibbs free energy, Masseiu potential and so on), and $ \lambda ^ \mu $ are the corresponding natural variables of the chosen thermodynamic potential (e.g $\{S, M\}$ - the entropy and total magnetization - for the internal energy $E$ of a magnetic system). 
The sign is chosen such that the metric is positive-definite, which is possible for a proper choice of $\varphi$ and $\lambda^\mu$, based on the convexity/concavity of the thermodynamic potentials and entropies \cite{Callen:450289}. 
% The definition is symmetric ($g_{\mu\nu}=g_{\nu\mu}$).
Salamon \cite{salamonThermodynamicLengthDissipated1983} showed that this metric is connected to non-equilibrium dissipation,
via the ``dissipated availability" 
\begin{equation} \label{SalomonDissipatedAvail}
    \mathcal{A} = \bar{\epsilon} \intop_{\boldsymbol{\lambda} \left(s\right)} g^R_{\mu\nu} \partial_{s^\prime}\lambda^{\mu} \partial_{s^\prime}\lambda^{\nu} ds^\prime
\end{equation}
Here $\boldsymbol{\lambda} \left( s \right)$ is the path, parameterized by $s$. $ \bar{\epsilon}$ is the averaged response time of the system (sometimes also referred to as "lag time"). The dissipated availability is the excess heat (or excess entropy, if using a generalized entropy) generated by the system due to finite time effects. Accordingly, one can see that in the quasi-static limie ($ \left| \dot{\boldsymbol{\lambda}} \right| \rightarrow 0 $), the dissipated availability approaches 0. The ``thermodynamic length" associated with the metric, is given by:
\begin{equation}\label{eq:L-def}    
    \mathcal{L} = \intop_{\boldsymbol{\lambda} \left(t\right)} \sqrt{g_{\mu\nu} \partial_s\lambda^{\mu} \partial_s\lambda^{\nu} } ds,
\end{equation}
It is an entirely mathematical construction, but minimization of the thermodynamic length, $\mathcal{L}$, is identical to minimization of the dissipated availability, $\mathcal{A}$, when fixing the time of the protocol. The length $\mathcal{L}$ gives a lower bound on the dissipated availability by the Cauchy-Schwartz inequality:
\begin{equation}
    \mathcal{A} \ge \frac{\bar{\epsilon}}{\tau} \mathcal{L}^2
\end{equation}
With $\tau$ being the time of the protocol. Equality is attained when $ \left| \dot{\boldsymbol{\lambda}} \right| = \sqrt{\dot{\lambda}^{\mu} g_{\mu\nu} \dot{\lambda}^{\nu}} $ is constant \cite{salamonThermodynamicLengthDissipated1983}.

\subsection{Sivak and Crooks Metric}
The endoreversible assumption, or a similar assumption about the relaxation time, essential for using the above metric, is severely limiting. Sivak and Crooks \cite{sivakThermodynamicMetricsOptimal2012} developed a similar formalism that does not require this assumption, and is thus usable across a wider range of applications.
This came at the price of a change in the definition of the metric:
\begin{equation} \label{eq:SivakMetric}
    g_{\mu\nu}^{\text{Sivak}} \equiv g_{\mu\nu}^S = \intop_0^\infty \left\langle \delta X_\mu (t) \delta X_\nu (0) \right\rangle dt = \mathcal{T}_{\mu\nu} \frac{\partial^2 \ln Z}{\partial \lambda^{\mu} \partial \lambda^{\nu}}
\end{equation}
Where as before $\lambda^\mu$ are the natural variables of the Masseiu potential, e.g $\{\beta, \beta h\}$ for magnetic systems, and $\boldsymbol{X}$ are the conjugate variables to $ \boldsymbol{\lambda} $ (in equilibrium, $X_\mu = \frac{\partial \ln Z}{\partial \lambda^\mu}$), $\mathcal{T}$ is the time response matrix, and $Z$ is the partition function. The the last equality is generically true in the thermodynamic limit,  \cite{sivakThermodynamicMetricsOptimal2012}, but for general, microscopic systems there is an added expression in the definition of the metric.

In the Sivak formalism the response time is directly encoded in the metric through $\mathcal{T}
$. Accordingly, we redefine $ \mathcal{A} $ to be
\begin{equation} \label{eqn:SivakDissAv}
    \mathcal{A} = \intop_{\boldsymbol\lambda \left(t\right)} g^S_{\mu\nu} \partial_t\lambda^{\mu} \partial_t\lambda^{\nu} dt,
\end{equation}
 

The definition of $\mathcal{L}$ stays identical, apart for the use of the Sivak metric. The Cauchy-Schwartz inequality gives in this case:
\begin{equation} \label{eq:CSineq}
    \mathcal{A} \ge \frac{\mathcal{L}^2}{\tau}
\end{equation}
With $\tau$ being the time of the protocol. As before, equality is attained when $\left| \dot{\boldsymbol{\lambda}} \right| = \sqrt{\dot{\lambda}^{\mu} g_{\mu\nu} \dot{\lambda}^{\nu}}$ is constant \cite{sivakThermodynamicMetricsOptimal2012}.

\subsection{Scaling Hypothesis \label{subsec:ScalingHypothesisFramework}}
% \oren{I think that it is better to first state the motivation and approach, something along the line of: (i) We expect the metric to be singular near a phase transition; (ii) There are very few cases where exact calculation of this singularity is possible; (iii) The singularity is governed by a set of critical exponents and scaling functions, that are identical for all models that belong to a given universality class. Therefore, we can learn a lot on the behavior of the metric near the phase transition by using the scaling hypothesis approach.}

The second derivatives of the free energy often diverges around second order phase transitions (e.g. the diverging magnetic susceptibility in a ferromagnet). Exact calculation of a system's partition function is many times hard or even impossible. 
However, the divergence of the partition function, and therefore its derivatives (e.g the Ruppiner metric), is dominated by the critical exponents. Those, in turn, will dictate whether the thermodynamic length diverges through the phase transition or not.
%Therefore there exist tools such as critical exponents that enable us to characterize these divergences without calculating them directly. 
Moreover, phase transitions have been organized into universality classes which are distinct from one another by the critical exponents.

The scaling hypothesis is a general relation between the critical exponents and the scaling of the free energy, around the phase transition \cite{Kardar_2007}. It allows us to check, for many different universality classes simultaneously, whether the thermodynamic length in the Ruppiner formalism diverges or not.
However, this does not tell us anything about the relaxation time, or its behavior around the phase transition. Therefore, this tool is not enough to understand divergences in the Sivak metric. 

The statement underpinning the scaling hypothesis is that around the phase transition point $ (T_c, h_c) $, the thermodynamic potentials scale as \cite{Kardar_2007}:
\begin{equation} \label{eq:scalingHypFreeEnergy}
    \varphi \propto t^b f \left( \frac{h}{t^\Delta} \right)
\end{equation}
where we have defined the reduced parameters around the phase transition point
\begin{equation}
    t = \frac{T-T_c}{T_c}
\end{equation}
\begin{equation}
    h = \frac{H-H_c}{H_c}
\end{equation}
meaning as we approach $ (T_c, h_c) $ we expect $t,h\rightarrow 0 $.

The hypothesis states that $f$ is continuous, and $ f \left(x\right) \xrightarrow[x\rightarrow\infty]{} x^p $. 

The variables $b, \Delta$ and $p$ can be determined from the critical exponents. For example, one can observe the critical exponent of the heat capacity at $h=0$, $\alpha$, is related to $b$ by $b=2-\alpha$.

The scaling hypothesis has been used before to quantify divergence of the metric, e.g \cite{PhysRevE.51.1006}. However, its impact on the thermodynamic length and dissipation has not been considered. 

\subsubsection*{Dynamic Scaling Hypothesis}
The scaling hypothesis can provide useful information about the divergence of the Ruppiner and Weinhold metric near a second order phase transition. However, commonly a second order phase transition is accompanied by a critical slow down, where the response timescale diverges. This can manifest itself in $ \bar{\epsilon} $, the mean response time, which may depend on the position, and even diverges at the critical point.
% change for different paths that get closer or further away from the phase transition. 
Gaining information about the critical slowdown can help us to quantify the divergence of $ \bar{\epsilon} $ in the Salomon dissipated availability (eq. \ref{SalomonDissipatedAvail}), 
and also the divergences of the Sivak metric (eq. \ref{eq:SivakMetric}).

The dynamical scaling hypothesis provides a relation between the relaxation time $\tau$ and the correlation length $\xi$ \cite{tauberCriticalDynamicsField2014}:
\begin{equation}
    \tau \propto \xi ^ z.
\end{equation}
This can be related to the scaling hypothesis of the free energy by the relation between free energies and the correlation length \cite{Kardar_2007}:
\begin{equation}
    \varphi \propto \xi ^ {-d}
\end{equation}
Where $d$ is the dimension of the system. This implies
\begin{equation}
    \tau \propto \varphi^{-\nicefrac{z}{d}}
\end{equation}
and using eq.\;\ref{eq:scalingHypFreeEnergy}, the scaling of the response time in terms of $t, h$ is given by:
\begin{equation}
        \tau \propto t^{-\nicefrac{bz}{d}} f ^{-\nicefrac{z}{d}} \left( \frac{h}{t^\Delta} \right)
\end{equation}

\subsection{Divergence of Lengths}
As is known, divergence of the metric doesn't necessarily imply divergence of distances in the manifold. One can construct metrics that diverge, while lengths remain finite. One known example of a metric with both a real divergence and a divergence with finite lengths is the Schwarzschild metric. While the metric itself diverges both at $ r=0 $ and at the Schwarzschild radius $r=r_s $, lengths diverge at $ r \rightarrow 0 $, but not at $ r \rightarrow r_s$. 


It is important to note that the divergence of dissipated availability depends not only on the shape on the protocol in parameter space, but also on its exact time parametrization. In other words, along any given path, if the protocol is executed sufficiently fast, the dissipated availability  diverges, since Eq. \ref{eqn:SivakDissAv} is sensitive to the parametrization. 
% However, at very high driving speeds the endoreversible/linear-response approximations used in deriving the dissipation expression are no longer valid.

This naturally raises the following question: can thermodynamic length provide insight into the divergence of dissipated availability? One direction is straightforward: for constant-dissipation-rate protocols, the dissipated availability and the thermodynamic length are connected directly through the Cauchy–Schwarz inequality. Therefore, whenever the thermodynamic length converges, there must exist a protocol for which the dissipated availability also converges.

The reverse implication is less obvious: if the thermodynamic length diverges, can there still exist a finite-time protocol yielding convergent dissipated availability? This case can also be addressed using the Cauchy–Schwarz inequality. Specifically, the minimal dissipated availability occurs when the protocol duration scales as $\tau \propto \mathcal{L}$. In this case, Eq. \ref{eq:CSineq} gives
\begin{equation}
    \mathcal{A}_{\text{min}} = \frac{\mathcal{L}}{c} \xrightarrow[\mathcal{L\rightarrow\infty}]{}\infty
\end{equation}
which shows that a divergent thermodynamic length necessarily implies divergent dissipated availability.

Thus, there are protocols that converge if and only if the thermodynamic length converges. Therefore we need to look only at the rate of divergence of
\begin{equation}
    \ell(s) = \sqrt{g_{\mu\nu} \partial_s \lambda^\mu \partial_s \lambda^\nu},
\end{equation}
i.e, assuming (as we assume in the rest of this paper) $\ell \xrightarrow[s\rightarrow 0]{} \infty$, the thermodynamic length  diverges if and only if $\ell \propto s^{-p}$ where $p>1$. Note that this cannot be directly translated to a divergence of the metric since it is parametrization dependent, similarly to how there exist coordinate systems where the $r_s$ divergence of the Schwarzschild metric do not appear (e.g Kruskal coordinates\cite{PhysRev.119.1743,Szekeres:1960gm}).


\section{Results}
\subsection{Scaling Hypothesis \label{subsec:ScalingHypothesisResults}}
\subsubsection{Scaling Hypothesis for the Free Energy}
We use the scaling hypothesis for the free energy to deduce the divergence of the Ruppiner metric, and understand wether the length diverges on paths that pass through a phase transition.


Assuming the divergent part of the Massieu potential scales as
\[ \psi \propto t^b f\left(\frac{h}{t^\Delta}\right) \]
With all the assumptions made so far, the scaling of the metric as $t,h\rightarrow 0 $ is given by:
\begin{align}
    \left(g_{\mu\nu}\right) \Big|_{\frac{h}{t^\Delta} \rightarrow \text{const.}} & \propto \left(\begin{array}{cc}
        t^{b-2} & t^{b-\Delta-1} \\
        t^{b-\Delta-1} & t^{b-2 \Delta}
    \end{array} \right) \\
    \left( g_{\mu\nu} \right) \Big|_{\frac{h}{t^\Delta} \rightarrow \infty} & \propto \left(\begin{array}{cc}
        t^{-2} h^{p} & t^{-1} h^{p-1} \\
        t^{-1} h^{p-1} & h^{p-2}
    \end{array} \right)
\end{align}
Note how we take the limits: We take both $t$ and $h$ to $0$, in such a way that either $\frac{h}{t^\Delta}$ tends to a constant value, or to infinity. This gives different limits for $f$, and therefore also different limits for the thermodynamic potentials and the metric, as seen in \ref{subsec:ScalingHypothesisFramework}.

To determine from this how lengths diverge, we need information about the path taken, as it determines the asymptotic {behaviour} of $ \frac{h}{t^\Delta} $. We therefore split our analysis to corresponding cases. The first is
\begin{equation} \label{case_1}
    \left( \begin{array}{c}
         t \\
         h 
    \end{array} \right) = \left( \begin{array}{c}
         s \\
         s^k 
    \end{array} \right) , \quad k \ge \Delta
\end{equation}
Which implies
\begin{equation*}
    \frac{h}{t^\Delta} \xrightarrow[s\rightarrow0]{} \text{const.}
\end{equation*}
In this case, the integrands are
\begin{equation*}
    g_{tt} \frac{dt}{ds} \frac{dt}{ds} \propto s^{b-2}
\end{equation*}
\begin{equation*}
    g_{th} \frac{dt}{ds} \frac{dh}{ds} \propto s^{b-\Delta-1} s^{k-1}=s^{b-2+\left(k-\Delta\right)}
\end{equation*}
\begin{equation*}
    g_{hh} \frac{dh}{ds} \frac{dh}{ds} \propto s^{b-2\Delta} s^{2\left(k-1\right)} =s^{b-2+2\left(k-\Delta\right)}
\end{equation*}
Since $k\ge\Delta$ we get that 
\begin{equation} \label{h_t_delta_non_divergent}
    b > 0
\end{equation} 
is a sufficient condition for the Ruppiener length, $\mathcal{L}$, to converge. Which, as we have established, is a necessary and sufficient condition for convergence of the Salomon dissipated availability, $\mathcal{A}$, for finite time protocols, unless $\bar{\epsilon}$ diverges. 

The second option is
\begin{equation} \label{case_2}
    \left( \begin{array}{c}
         t \\
         h 
    \end{array} \right) = \left( \begin{array}{c}
         s^{\nicefrac{1}{k}} \\
         s 
    \end{array} \right) , \quad k < \Delta
\end{equation}
which implies
\begin{equation*} 
    \frac{h}{t^\Delta} \xrightarrow[s\rightarrow 0]{} \infty
\end{equation*}
In this case
\begin{equation*}
    g_{tt} \frac{dt}{ds} \frac{dt}{ds} \propto s^{-\nicefrac{2}{k}}s^p s^{2\left(\frac{2}{k}-1\right)} = s^{p-2}
\end{equation*}
\begin{equation*}
    g_{th} \frac{dt}{ds} \frac{dh}{ds} \propto s^{-\nicefrac{1}{k}} s^{p-1} s^{\frac{1}{k}-1} = s^{p-2}
\end{equation*}
\begin{equation*}
    g_{hh} \frac{dh}{ds} \frac{dh}{ds} \propto s^{p-2}
\end{equation*}


So there exist finite time protocols that do not diverge for a system when
\begin{equation} \label{h_t_delta_divergent}
    p>0
\end{equation}
% \oren{Same as the discussion above -- why not $p>1$?}
One can observe in table \ref{tab:scaling-divergences} that both conditions \ref{h_t_delta_non_divergent} and \ref{h_t_delta_divergent} hold in multiple universality classes.  %\oren{Maybe it is better to put the specific mean field example here, as it actually belong to the above discussion, not to what follow.} \omer{The mean field discussion was supposed to be for the numerical part, should it appear also in the context of scaling of the free energy/its derivatives?}

 \begin{table}
    \centering
    \begin{tabular}{c||c|c}
         & $b$ & $p$ \\
         \hline \hline 
         Mean Field & $ 2 $ & $ \frac{4}{3} $ \\
         \hline 
         2D Ising & $ 2 $ & $ \frac{16}{15} $ \\
         \hline 
         3D Ising & $ 1.88991292(35) $ & $ 1.2087751(6) $ \\
         \hline 
         2D 3-state Potts model & $ \frac{5}{3} $ & $ \frac{15}{14} $ \\
         \hline 
         2D 4-state Potts model & $ \frac{4}{3} $ & $ \frac{16}{15} $
         % \\
         % \hline 
         % 3D XY Model & $ 2.0153(3) $ & $ 1.20923(6) $
    \end{tabular}
    \caption{Scaling Hypothesis parameters for sample universality classes}
    \label{tab:scaling-divergences}
\end{table}

\subsubsection{Dynamical Scaling Hypothesis}

As we have seen, all attempts to use metrics to study dissipation take into account the system's response time. In the Ruppiner formalism this appears in the mean response time element $\bar{\epsilon}$, and in the Sivak formalism it appears directly in the metric. Here we use the dynamical scaling hypothesis to study the scaling of the response time, $\tau$.

We have seen that $\tau$ scales as:
$ \tau \propto \psi ^ {-\nicefrac{z}{d} } $, so using the static scaling hypothesis
\begin{equation}
    \tau \propto t^{-\nicefrac{bz}{d}} g\left(\frac{h}{t^\Delta}\right)
\end{equation}
with $g$ continuous and 
\begin{equation}
    g \xrightarrow[x\rightarrow \infty]{} x^{-\nicefrac{pz}{d}},
\end{equation}
the divergence of $\tau$ is
\begin{equation}
    \tau \propto 
    \begin{cases}
        t^{-\nicefrac{bz}{d}}  &  \frac{h}{t^\Delta} \rightarrow \text{const.} \\
        h^{-\nicefrac{pz}{d}}  &  \frac{h}{t^\Delta} \rightarrow \infty
    \end{cases}
\end{equation}

Adding this factor to the case of the divergences in eq. \ref{case_1}
\begin{align}
    g_{tt} \left(\frac{dt}{ds}\right)^2 & \propto s^{b\left(1-\frac{z}{d}\right) -2 } \\
    g_{th} \frac{dt}{ds} \frac{dh}{ds} & \propto s^{b\left(1-\frac{z}{d}\right) -2 + \left(k-\Delta\right) } \\
    g_{hh} \left(\frac{dh}{ds}\right)^2 & \propto s^{b\left(1-\frac{z}{d}\right) -2 + 2\left(k-\Delta\right)} 
\end{align}
Meaning the convergence condition for the thermodynamic length will be 
\begin{equation}
    b\left(1-\frac{z}{d}\right) > 0
\end{equation}

Similarly in case \ref{case_2}:
\begin{equation}
    g_{ij}  \frac{dx^i}{ds} \frac{dx^j}{ds} \propto s^{p\left(1-\frac{z}{d}\right) -2}
\end{equation}
Meaning the convergence condition for the thermodynamic length will be 
\begin{equation}
    p\left(1-\frac{z}{d}\right) > 0
\end{equation}
One can see, using table \ref{tab:dynamical-divergences}, that the lengths will diverge for the 2D Ising model and Potts models (both 3- and 4-states), but in the 3D Ising model, lengths converge.
 \begin{table}
    \centering
    \begin{tabular}{c|c||c|c|c}
         
         \multicolumn{2}{c||}{} & $d$ & $z$ & $\frac{z}{d}$\\
         \hline \hline 
         \multirow{2}{*}{2D Ising}  & Class A & $ 2 $ & $ 2.165\left(10\right) $ & $1.083\left(5\right)$ \\
         \cline{2-5}
                                    & Class B & $ 2 $ & $ 2.235\left(10\right) $ & $1.117\left(5\right)$ \\ 
                                    % I'm not sure model B dynamics are interesting in our case.
         \hline 
         \multicolumn{2}{c||}{3D Ising} & $ 3 $ & $ 2.032\left(4\right) $ & $ 0.677\left(1\right) $ \\
         \hline
         \multicolumn{2}{c||}{2D 3-state Potts model} & $2$ & $2.198(2)$ & $1.099(1)$ \\
         \hline
         \multicolumn{2}{c||}{2D 4-state Potts model} & $2$ & $2.290(3)$ & $1.095(2)$ \\
    \end{tabular}
    \caption{Dynamical Scaling parameters for sample universality classes \cite{odor_universality_2004}}
    \label{tab:dynamical-divergences}
\end{table}

\subsection{Antiferromagnetic Mean-Field Model - Ruppiner metric}
As was mentioned before, in the Antiferromagnet it is impossible to build a path that crosses between the antiferromagnetic and disordered phases without passing through a second order phase transition. Therefore, it is a prime example of a system where our analysis is necessary.

We demonstrate finding the path of least dissipation between two points $\left(T_1, h_1\right) \rightarrow \left(T_2, h_2\right)$, where $\left(T_1, h_1\right)$ is in the antiferromagnetic phase, and $\left(T_2, h_2\right)$ is in the disordered phase. We use the Ruppiner metric, as it is analytically solvable.

\subsubsection{The Model}
The model we use is the Antiferromagnetic Mean-Field model, as laid out in \cite{vivesUnifiedMeanfieldStudy1997}. It consists of spins on a square lattice, divided into two sub-lattices, such that spins interact only with spins from the other sub-lattice. Denoting the magnetization of each sub-lattice by $m_A,m_B$, the Gibbs Free energy is:
\begin{equation}
    f = \frac{1}{2} \left[ K m_A m_B - h \left( m_A + m_B \right) + \frac{1}{2} T \left( \vartheta \left(m_A\right) +\vartheta\left(m_B\right) \right) \right]
\end{equation}
With 
\begin{equation}
    \vartheta \left(x\right) = \frac{1}{2} \left[ \left(1+x\right) \log \left(1+x\right) + \left(1-x\right) \log \left(1-x\right) \right]
\end{equation}
We define $K$ to be the interaction strength $J$ times the coordination number.

In thermodynamic equilibrium $m_A,m_B$ are the minimizers of $f$, and solve the self consistent equations
\begin{align}
    \label{eq:SCExy}
    m_A = & \tanh\left(\frac{h-Km_B}{T}\right) \\
    m_B = & \tanh\left(\frac{h-Km_A}{T}\right)
\end{align}

Taking the derivative gives the Ruppiner metric, since this is the gibbs free energy, is done w.r.t $T, h$, and gives:
\begin{widetext}

    \begin{equation}
        g_{hh} = \frac{T \left(2-m_A^2-m_B^2\right) - 2 K \left(1-m_A^2\right) \left(1-m_B^2\right)}
        {2 \left[ T^2- \left(1-m_A^2\right) \left(1-m_B^2\right) K^2\right]}
    \end{equation}
    \begin{equation}
        g_{hT} = - \frac{\left(1-m_A^2\right) \arctanh(m_A) \left(T-\left(1-m_B^2\right) K\right)+\left(1-m_B^2\right) \arctanh (m_B) \left(T-\left(1-m_A^2\right) K\right)}{2 \left[ T^2- \left(1-m_A^2\right) \left(1-m_B^2\right) K^2\right]}
    \end{equation}
    \begin{equation}
        g_{TT} = \frac{T \left[ \arctanh^2( m_B ) \left(1-m_B^2\right) + \arctanh^2( m_A ) \left(1-m_A^2\right)\right] - K \left( 1-m_A^2 \right) \left(1-m_B^2\right) \arctanh(m_B) \arctanh(m_A) 
        }{2 \left[T^2- \left(1-m_A^2\right) \left(1-m_B^2\right) K^2\right]}
    \end{equation}
\end{widetext}
% With the determinant of the metric being
% % \begin{widetext}
%     \begin{equation}
%         g = \frac{\left(1-m_A^2\right) \left(1-m_B^2\right) \left(\arctanh(m_A)-\arctanh(m_B)\right)^2}{4 \left(T^2-\left(1-m_A^2\right) \left(1-m_B^2\right) (K)^2\right)}
%     \end{equation}
% % \end{widetext}

\subsubsection{Disordered Phase}
One interesting property that can be readily observed is that in the disordered phase, where $m_A=m_B\equiv m$, one gets that $g=0$, i.e the determinant of the metric vanishes. This has been observed before in the Ferromagnetic mean-field model \cite{janyszekRiemannianGeometryThermodynamics1989}. It can be shown that the 0 eigenvectors in the disordered phase point along lines of constant magnetization (or equivalently lines of constant $m_A$). As a result, the problem becomes a 1D minimization problem - at what distance from the phase transition line will the length be minimal? As we show below, the answer is simply at the phase transition line itself.

In the following section we will assume when writing $T_c$ that it is the critical temperature for a specific magnetization. Two central results that can be easily proven by derivation of the free energy are that on the phase transition line
\begin{equation}
    m_A=m_B=\pm \sqrt{1-\frac{T_c}{K}}
\end{equation}
And
\begin{equation} \label{eq:h_c(T_c)}
    h_c (T_c) = \pm\left(K\sqrt{1-\frac{T_c}{K}} + T_c \arctanh\left( \sqrt{1-\frac{T_c}{K}} \right) \right)
\end{equation}


The metric becomes in the disordered phase ($m_A=m_B$)
\begin{equation}
    g_{ij} = \frac{T_c}{K \left(T+T_c\right)}
    \left( \begin{array}{cc}
         \arctanh(m)^2 & -\arctanh(m) \\
         -\arctanh(m) & 1 \\
    \end{array} \right)
\end{equation}
The eigenvalues and eigenvectors of this metric are
\begin{align}
    \lambda_1 & = 0,\quad \boldsymbol{v}_1 = \left( \begin{array}{c}
         1  \\
        \arctanh(m_A)
    \end{array} \right) \\
    \lambda_2 & = \arctanh^2(m_A)+1,\quad \boldsymbol{v}_2 = \left( \begin{array}{c}
        \arctanh(m_A) \\
         -1 
    \end{array} \right) 
\end{align}
One can see an illustration of the "0-length" lines in Fig. (\ref{fig:0length}).

\begin{figure}
    \centering
    \includegraphics[width=1\linewidth]{figures/disorderedPhaseLines.jpg}
    \caption{lines of "0-length" in disordered phase}
    \label{fig:0length}
\end{figure}

And so we need to minimize the expression
$ \sqrt{\dot{\lambda}^\mu g_{\mu\nu} \dot{\lambda}^\nu} \,dt$
along the $\boldsymbol{v}_1$ axis. We know $g_{\mu\nu}$ scales as $\left(T+T_c\right)^{-1}$. We will argue that the rest of the expression scales as $T-T_0$ for some $T_0$.

\omer{It feels to me like the next part could be made more precise and clear with some statement of curvature, this is just how I  calculated it originally, but if you can formalize it better I would be happy.}
The rest of the expression is simply euclidean distance projected onto the normal to the lines of constant magnetization. Hence, it should scale as the radial distance from the point of meeting of two adjacent lines of constant magnetization.
The radial distance, since the lines aren't vertical or horizontal (except for one) is proportional to the distance along the $T$-axis.

Taking some point $\left(T_c,h_c\right)$ and an adjacent point on the phase transition line
\begin{equation*}
    (T_c+\delta_T, h_c + \frac{\partial h_c}{\partial T_c}\delta_T)
\end{equation*}
We want the intersection of the lines (i.e, to solve for $T_0$):
\begin{align} \label{eq:T0,h0}
    \left( \begin{array}{c}
         T_0 \\
         h_0
    \end{array}\right) & = a \, \boldsymbol{v}_1\left(T_c\right) + \left( \begin{array}{c}
         T_c \\
         h_c
    \end{array}\right) \\
    \left( \begin{array}{c}
         T_0 \\
         h_0
    \end{array}\right) & = b \, \boldsymbol{v}_1\left(T_c + \delta_T \right) + \left( \begin{array}{c}
         T_c + \delta_T \\
         h_c + \frac{\partial h_c}{\partial T_c}\delta_T
    \end{array}\right)
\end{align}

We first subtract the equations to get
\begin{equation}
    \delta_T \left( \begin{array}{c}
         1  \\
         \frac{\partial h_c}{\partial T_c}
    \end{array}\right) = \left( \begin{array}{cc}
        1 & 1 \\
        \arctanh(m(T_c+ \delta_T)) & \arctanh(m(T_c))
    \end{array} \right)
    \left( \begin{array}{c}
         -b  \\
         a
    \end{array}\right)
\end{equation}
Inverting the matrix gives:
\begin{widetext}
\begin{equation}
    \begin{array}{rl}
        \left( \begin{array}{cc}
            1 & 1 \\
            \arctanh(m(T_c+\delta_T)) & \arctanh(m(T_c))
        \end{array} \right)^{-1} & = \frac{1}{\arctanh\left(m(T_c)\right) - \arctanh\left(m(T_c + \delta_T)\right)}\left( \begin{array}{cc}
            \arctanh(m(T_c)) & -1 \\
            -\arctanh(m(T_c + \delta_T)) & 1
        \end{array} \right) \\ & = -\frac{2T_c \sqrt{1-\frac{T_c}{K }}}{K  \, \delta_T}\left( \begin{array}{cc}
            \arctanh\left(\sqrt{1-\frac{T_c}{K }}\right) & -1 \\
            -\arctanh\left(\sqrt{1-\frac{T_c + \delta_T}{K }}\right) & 1
        \end{array} \right)
    \end{array}
\end{equation}
giving
\begin{equation} \label{eq:a}
    a = \frac{2}{K }\arctanh\left(\sqrt{1-\frac{T_c}{K }}\right) T_c \sqrt{1-\frac{T_c}{K }} - 
    \frac{2}{K } T_c \sqrt{1-\frac{T_c}{K }} \frac{\partial h_c}{\partial T_c}
\end{equation}
\end{widetext}

Deriving equation \ref{eq:h_c(T_c)} one gets
\begin{equation}
    \frac{\partial h_c}{\partial T_c} = -\frac{1}{\sqrt{1-\frac{T_c}{K}}} + \arctanh\sqrt{1-\frac{T_c}{K }}
\end{equation}

And substituting this into equation \ref{eq:a} gives
\begin{equation}
    a = \frac{2 T_c}{ K }
\end{equation}

By substituting this result into \ref{eq:T0,h0} we get
\begin{equation}
    T_0 = T_c + \frac{2}{K} T_c = \frac{K + 2}{K} T_c
\end{equation}

Accordingly, We get that the differential length between two different points in the disordered phase with different magnetization scales as
\begin{equation}
    \frac{\left(T + \frac{K + 2}{K} T_c\right)^2}{T+T_c}
\end{equation}
the minimum in the disordered phase will be attained at $T=T_c$.
\subsubsection{Antiferromagnetic Phase}

As previously mentioned, in the anti-ferromagnetic phase the metric is non-degenerate. Hence, the problem of finding shortest paths is 2D. We used numeric methods to find the shortest paths from a given point to different points on the phase transition line.

Calculation of the shortest path to the phase transition line from the inner point was done using the fast marching method\cite{kimmelComputingGeodesicPaths1998}.
% Some more text and graphs once I finish the calculations.
 \subsection{Antiferromagnetic Mean-Field Model - Sivak metric}
 \omer{I wrote this assuming we will only publish this, and not the Ruppiner metric}
 

We now determine the least-dissipation path between two states 
$\left(T_1,h_1\right)\!\to\!\left(T_2,h_2\right)$,
with $\left(T_1,h_1\right)$ in the antiferromagnetic phase and $\left(T_2,h_2\right)$ in the disordered phase. 
In particular, we compute the Sivak metric for the MF antiferromagnetic model of \cite{vivesUnifiedMeanfieldStudy1997}.

\subsubsection{The Model}
The Massieu potential is
\begin{widetext}
\begin{equation}
    \psi \;=\ln Z=\; -\frac{1}{2}\!\left[ \beta K\, m_A m_B \;-\; \beta h\, (m_A+m_B)
    \;+\; \frac{1}{2}\,\big(\vartheta(m_A)+\vartheta(m_B)\big) \right]
\end{equation}
\end{widetext}
where $\beta=1/T$.

\subsubsection{Calculation of the Metric}
The computation proceeds in three steps:
\begin{enumerate}
    \item \textbf{Equilibrium covariances:} evaluate 
    $\langle \delta m_i(0)\,\delta m_j(0)\rangle$ ($i,j\in\{A,B\}$).
    \item \textbf{Dynamics and time integral:} incorporate linearized relaxation to obtain the integrated correlation.
    \item \textbf{Variable transform:} convert to the $\boldsymbol{\lambda}=(\beta,\beta h)\equiv(\beta,\alpha)$ representation, which is the representation of our control parameters, via the Jacobian from $(m_A,m_B)$ to $(-E,m)$.
\end{enumerate}
Note we set $\alpha \equiv \beta h$ for brevity.

Let $\zeta_i\equiv 1-m_i^2$ for $i\in\{A,B\}$. 
The inverse covariance is the Hessian of $\psi$ with respect to $(m_A,m_B)$:
\begin{equation}
    -\Sigma^{-1} \;=\; \frac{1}{2}\!
    \begin{pmatrix}
        \dfrac{1}{\zeta_A} & \beta K \\
        \beta K & \dfrac{1}{\zeta_B}
    \end{pmatrix}.
\end{equation}
Its inverse is
\begin{equation}
    \Sigma \;=\; \frac{2}{\,1-\beta^2 K^2 \zeta_A \zeta_B\,}
    \begin{pmatrix}
        \zeta_A & -\beta K\,\zeta_A\zeta_B \\
        -\beta K\,\zeta_A\zeta_B & \zeta_B
    \end{pmatrix}
\end{equation}

The relaxation dynamics \cite{klichMpembaIndexAnomalous2019} are
\begin{align}
    \dot m_A &= \Gamma\!\left[\tanh\!\left(\alpha - \beta K\, m_B\right) - m_A\right]\\
    \dot m_B &= \Gamma\!\left[\tanh\!\left(\alpha - \beta K\, m_A\right) - m_B\right]
\end{align}
with microscopic attempt rate $\Gamma$. 
Linearizing about equilibrium gives
\begin{align}
    \delta \dot m_A &= \Gamma\left(-\,\delta m_A \;-\; \beta K\,\zeta_A\,\delta m_B\right)\\
    \delta \dot m_B &= \Gamma\left(-\,\beta K\,\zeta_B\,\delta m_A \;-\; \delta m_B\right)
\end{align}
In vector form, with $y=(\delta m_A,\delta m_B)^{\!\top}$
\begin{equation}
    \dot y \;=\; -\Gamma 
    \begin{pmatrix}
        1 & \beta K\,\zeta_A \\
        \beta K\,\zeta_B & 1
    \end{pmatrix} y \;=\; -M y
\end{equation}

We require
\begin{equation}
    \beta \int_{0}^{\infty}\!\langle \delta m_i(0)\,\delta m_j(t)\rangle\,dt
    \;=\; \beta \int_0^\infty \!\langle y(0)\,y^{\!\top}(t)\rangle\,dt
\end{equation}
Since $\langle \delta m_i(0)\,\eta(t)\rangle=0$ for $t>0$, substituting $y(t)=e^{-Mt}y(0)$ yields
\begin{widetext}
\begin{equation}
\begin{aligned}
    \beta \int_0^\infty \!\langle y(0)\,y^{\!\top}(t)\rangle\,dt
    &= \beta\,\langle y(0)\,y^{\!\top}(0)\rangle \int_0^\infty \!e^{-M^{\!\top}t}\,dt
     \;=\; \beta\,\Sigma\,(M^{\!\top})^{-1} \\[3pt]
    &= \frac{2\beta}{\Gamma\big(1-\beta^2 K^2 \zeta_A \zeta_B\big)^{\!2}}
    \begin{pmatrix}
        \zeta_A\big(1+\beta^2 K^2 \zeta_A \zeta_B\big) & -\,2\beta K\,\zeta_A\zeta_B\\
        -\,2\beta K\,\zeta_A\zeta_B & \zeta_B\big(1+\beta^2 K^2 \zeta_A \zeta_B\big)
    \end{pmatrix}\!
\end{aligned}
\end{equation}
\end{widetext}

To express the Sivak metric in the control variables $\boldsymbol{\lambda}=(\beta,\alpha)$, we transform fluctuations to $(-E,m)$ via
\begin{align}
    E &= \tfrac{1}{2} K\, m_A m_B\\
    m &= \tfrac{1}{2}(m_A+m_B)
\end{align}
so the Jacobian is
\begin{equation}
    P \;=\; \frac{\partial(-E,m)}{\partial(m_A,m_B)} 
    \;=\; \frac{1}{2}
    \begin{pmatrix}
        -K m_B & -K m_A \\
        \;1 & \;1
    \end{pmatrix}
\end{equation}
The Sivak metric then reads
\begin{widetext}
\begin{equation}
\resizebox{\columnwidth}{!}{$
\begin{aligned}
    g^{\text{Sivak}}_{\mu\nu}
    \;= & \; \beta\, P\,\Sigma\,(M^{\!\top})^{-1} P^{\!\top} \\
    \;= & \; \frac{\beta}{2\Gamma\big(1-\beta^2 K^2 \zeta_A \zeta_B\big)^{\!2}}
    \begin{pmatrix}
    \scriptstyle K^2 \!\left[\!\big(1+\beta^2 K^2 \zeta_A \zeta_B\big)\big(m_A^2 \zeta_B + m_B^2 \zeta_A\big) - 4\beta K\, m_A m_B \zeta_A \zeta_B \!\right]
    &
    \scriptstyle -K \!\left[\!\big(1+\beta^2 K^2 \zeta_A \zeta_B\big)\big(m_A \zeta_B+m_B \zeta_A\big) - 2\beta K (m_A+m_B)\zeta_A\zeta_B \!\right]
    \\[6pt]
    \scriptstyle -K \!\left[\!\big(1+\beta^2 K^2 \zeta_A \zeta_B\big)\big(m_A \zeta_B+m_B \zeta_A\big) - 2\beta K (m_A+m_B)\zeta_A\zeta_B \!\right]
    &
    \scriptstyle \big(1+\beta^2 K^2 \zeta_A \zeta_B\big)(\zeta_A+\zeta_B) - 4\beta K\,\zeta_A\zeta_B
    \end{pmatrix}\!
\end{aligned}
$}
\end{equation}
\end{widetext}

\begin{figure}[h]
    \centering
    \begin{subfigure}[b]{0.45\textwidth}
        \centering
        \includegraphics[width=0.9\textwidth]{figures/sivakmetric_bb.png}
        \caption{Caloric element $g_{\beta\beta}$}
    \end{subfigure}%\hfill
    
    \begin{subfigure}[b]{0.45\textwidth}
        \centering
        \includegraphics[width=0.9\textwidth]{figures/sivakmetric_ab.png}
        \caption{Magneto–caloric element $g_{\alpha\beta}$}
    \end{subfigure}
    % \vspace{0.75ex}
    \begin{subfigure}[b]{0.45\textwidth}
        \centering
        \includegraphics[width=0.9\textwidth]{figures/sivakmetric_aa.png}
        \caption{Magnetic element $g_{\alpha\alpha}$}
    \end{subfigure}
    \caption{Elements of the Sivak metric for the MF antiferromagnet}
    \label{fig:sivakmetric_antiferro}
\end{figure}

Figures~\ref{fig:sivakmetric_antiferro} show the resulting metric components.


\subsubsection{Disordered phase} \label{subsec:SivakDisorderedPhase}
It is convenient to define the total and staggered magnetizations
\begin{align}
    m &= \frac{m_A+m_B}{2}\\
    s &= \frac{m_A-m_B}{2}
\end{align}

For each $m\in[-1,1]$, the corresponding critical parameters $(\beta_c,\alpha_c)$ on the phase boundary satisfy
\begin{align}
    m &= \pm\sqrt{1-\frac{1}{\beta_c K}} \label{m(beta)}\\
    \alpha_c &= \pm\Big(\beta_c K\, m + \arctanh m\Big) \label{alpha(beta,m)}
\end{align}

In the disordered phase, $m_A=m_B\equiv m$ (hence $\zeta_A=\zeta_B\equiv \zeta$). 
The metric in this case reduces to
\begin{equation} \label{eq:disorderedmetric}
    g_{\mu\nu} \;=\; \frac{\beta\,\zeta}{\Gamma\bigl(1+\beta K \zeta\bigr)^2}
    \begin{pmatrix}
        K^2 m^2 & -K m \\[2pt]
        -K m & 1
    \end{pmatrix}.
\end{equation}
As in the ruppiner case, the metric is rank-one, with the null eigenvector aligning with directions of constant $m$.
The problem of minimizing distances again reduces to a 1D minimization: at what distance from the phase boundary does the length become minimal? 
This will be calculated in the following paragraphs:

For each $m\in[-1,1]$, the corresponding critical parameters $(\beta_c,\alpha_c)$ on the phase boundary satisfy
\begin{align}
    m &= \pm\sqrt{1-\frac{1}{\beta_c K}} \label{m(beta)}\\
    \alpha_c &= \pm\Big(\beta_c K\, m + \arctanh m\Big) \label{alpha(beta,m)}
\end{align}
The eigenpairs of the metric \ref{eq:disorderedmetric} are
\begin{align}
    \lambda_1 &= 0 \quad :
    & \boldsymbol{v}_1 &= \begin{pmatrix} 1 \\ K m \end{pmatrix}
    \\
    \lambda_2 &= 1+K^2 m^2 \quad:
    & \boldsymbol{v}_2 &= \begin{pmatrix} K m \\ -1 \end{pmatrix}
\end{align}
Thus the zero mode $\boldsymbol{v}_1$ follows lines of constant $m$ in $(\beta,\alpha)$.

\paragraph{Scaling of the metric prefactor.}
From \eqref{m(beta)} we have $\zeta=1-m^2=1/(\beta_c K)$ along the phase boundary labeled by magnetization $m$, with critical parameter $\beta_c$.
Using this in the prefactor gives
\begin{equation}
    g_{\mu\nu} \;\propto\; 
    \frac{\beta}{\bigl(\beta_c+\beta\bigr)^2}\,\bigl(1+K^2 m^2\bigr).
\end{equation}

\paragraph{Scaling of the displacement along the zero mode.}
Consider two neighboring constant-$m$ lines labeled by 
$(\beta_{c1},\alpha_{c1})$ and $(\beta_{c2},\alpha_{c2})=(\beta_{c1}+\delta_\beta,\alpha_{c1}+\delta_\alpha)$.
We find their intersection with a straight line parallel to the zero mode at each point:
\begin{align}
    \begin{pmatrix}\beta\\ \alpha\end{pmatrix}
    &= a\,\boldsymbol{v}_1(\beta_{c1}) + \begin{pmatrix}\beta_{c1}\\ \alpha_{c1}\end{pmatrix}
    \label{eq:b0a0-1}\\[2pt]
    \begin{pmatrix}\beta\\ \alpha\end{pmatrix}
    &= b\,\boldsymbol{v}_1(\beta_{c2}) + \begin{pmatrix}\beta_{c1}+\delta_\beta\\ \alpha_{c1} + \dfrac{\partial \alpha_c}{\partial \beta_c}\,\delta_\beta \end{pmatrix}
    \label{eq:b0a0-2}
\end{align}
Subtracting \eqref{eq:b0a0-1} from \eqref{eq:b0a0-2} yields
\begin{equation}
    \delta_\beta 
    \begin{pmatrix} 1 \\[2pt] \dfrac{\partial \alpha_c}{\partial \beta_c} \end{pmatrix}
    =
    \begin{pmatrix}
        1 & 1 \\
        K m_2 & K m_1
    \end{pmatrix}
    \begin{pmatrix} -b \\[2pt] a \end{pmatrix}
\end{equation}
where $m_1=m(\beta_{c1})$, $m_2=m(\beta_{c2})$.
From \eqref{alpha(beta,m)} we have 
\(
\dfrac{\partial \alpha_c}{\partial \beta_c} = \dfrac{1}{m}
\)
Inverting gives
\begin{equation}
\begin{aligned}
    a \;&=\; \frac{1-m^2}{K m}\,\frac{\partial m}{\partial \beta_c}
      \;=\; \frac{1-m^2}{K m}\,\frac{2m}{(1-m^2)^2}\\
      \;&=\; \frac{2}{K(1-m^2)}
      \;=\; 2\beta_c
\end{aligned}
\end{equation}
where we used $1-m^2=1/(\beta_c K)$ and $\dfrac{\partial m}{\partial \beta_c} = \dfrac{2m}{(1-m^2)^2}$ from \eqref{m(beta)}.
Hence the displacement along the zero mode scales as 
\(
\delta\boldsymbol{\lambda}\sim (\beta+\beta_c).
\)

\paragraph{Conclusion: minimum at $\beta=0$.}
Combining the prefactor and displacement scalings,
\[
\sqrt{\delta\lambda^\mu\, g_{\mu\nu}\, \delta\lambda^\nu}
\;\propto\;
\sqrt{\beta}
\]
so the infinitesimal thermodynamic length is minimized at $\beta=0$ (infinite temperature).
Therefore, the optimal trajectory between two points in this phase follows the null direction (of constant magnetization) from the initial point to $\beta=0$, then traverse along $\beta=0$, then return along a null trajectory to the final point. 

This is opposite to our result for the Ruppiner metric, where the optimal trajectory is along the phase transition line. Meaning, taking into account the divergence in response time substantially changes the optimal trajectory in this model.
\subsubsection{Divergence at the Phase transition}

In the ordered phase, near the critical line, the sublattice magnetizations can be expanded as shown in \cite{vivesUnifiedMeanfieldStudy1997}:
\begin{align}
    m &\approx \pm\sqrt{1-\frac{1}{\beta_c K}} + a_m\,\Delta\alpha - b_m\,\Delta\beta \\
    s &\approx \sqrt{a_s\,\Delta\alpha + b_s\,\Delta\beta}
\end{align}
where $a_m,b_m,a_s,b_s>0$, and
\begin{align*}
    \Delta\alpha &= \alpha - \alpha_c &
    \Delta\beta &= \beta - \beta_c
\end{align*}
For brevity we define
\begin{align*}
    \Delta_m &= a_m\,\Delta\alpha - b_m\,\Delta\beta &
    \Delta_s &= a_s\,\Delta\alpha + b_s\,\Delta\beta
\end{align*}

To lowest order in these small deviations, the sublattice magnetizations and corresponding susceptibilities satisfy
\begin{align}
    m_{A,B} &\approx \pm\sqrt{1-\frac{1}{\beta_c K}} \;\pm\; \sqrt{\Delta_s} \\
    \zeta_{A,B} &\approx \frac{1}{\beta_c K}\bigl(1 \pm 2\sqrt{\Delta_s}\bigr)
\end{align}

Substituting these expressions into the Sivak metric yields, to leading order,
\[
g_{\mu\nu}\;\propto\;\Delta_s^{-1}
\]
Hence, while the metric diverges as $\Delta_s\!\to\!0$, the associated thermodynamic length remains finite. 

\subsubsection{Minimal Paths}

We identify three distinct classes of minimal (shortest) paths, two of which cross the phase transition line.

\paragraph{Type A: Crossing between phases.}
In the first case, the initial and final states lie in different phases. 
The optimal trajectory proceeds from the initial point in the antiferromagnetic phase to the nearest point on the phase transition line, crosses into the disordered phase, and then ascends to infinite temperature ($\beta = 0$). 
Once at $\beta = 0$, the system moves horizontally along this line—where the thermodynamic length vanishes—until reaching the final magnetization, and finally descends along the constant-magnetization line to the destination point.

\paragraph{Type B: Crossing and returning.}
The second case occurs when both endpoints are in the antiferromagnetic phase, but it is shorter to cross the phase transition than to remain entirely within the ordered region. 
This trajectory can be viewed as a composition of two Type~I paths: the system transitions from the ordered phase to the disordered phase and then back again. 
Because, as shown in Sec.~\ref{subsec:SivakDisorderedPhase}, paths within the disordered phase have zero thermodynamic length, the total length of such a trajectory is dominated by the two crossings.

\paragraph{Type C: Entirely within one phase.}
In the third case, both points are again in the antiferromagnetic phase, but the direct geodesic connecting them is shorter than any path that detours through the disordered region. 
Here, the minimal path lies entirely within the ordered manifold.

An illustration of these three classes of paths is shown in Fig.~\ref{fig:shortestPaths}.
\begin{figure}[h]
    \centering
    \includegraphics[width=0.95\linewidth]{figures/ShortestPaths.png}
    \caption{Illustration of the three types of minimal paths in the mean-field antiferromagnet using the Sivak metric.}
    \label{fig:shortestPaths}
\end{figure}

The thermodynamic lengths within the antiferromagnetic phase were computed using the Fast Marching Method \cite{kimmelComputingGeodesicPaths1998}, following the approach of \cite{rotskoffDynamicRiemannianGeometry2015}. 
Minimal paths were obtained by performing gradient descent along the numerically computed distance function.

\subsection{2D Ising model}
As an example for a system with a diverging Ruppiener metric but  nondiverging distance through the phase transition, let us consider the 2D Ising model on a triangular lattice with nearest neighbors interactions. This model has an exact solution due to Onsager, therefore it is possible to analytically calculate the thermodynamic distance for a trajectory as long as the magnetic field stays zero. This model served to numerically demonstrate trajectories with minimal thermodynamic length in \cite{rotskoffDynamicRiemannianGeometry2015}. Since we are considering  trajectories with a constant external field $h=0$, the distance between two states $T_1$ and $T_2$ is given by
\begin{equation}
    \mathcal{L} = \int_{T_1}^{T_2}\sqrt{g_{TT}}dT
\end{equation}
For the Ruppiener metric, 
\begin{eqnarray}
    g_{TT}&=& \frac{c(T)}{k_{\mathrm B}T^2}\nonumber \\
		&=& \frac{4}{T^2\pi}\big(K\coth2K\big)^2\!\left[K_1(q)-E_1(q)\right]\nonumber\\
		& &-\frac{1-\tanh^2\!2K}{T^2}\left[\frac{\pi}{2}+\frac{2}{\pi}\big(2\tanh^2\!2K-1\big)K_1(q)\right]\nonumber
\end{eqnarray}
where $K=\beta J$, $K_1(\cdot)$ and $E_1(\cdot)$ denote the complete elliptic integrals of the first and second kinds respectively, and $q(K)=\frac{2\sinh(2K)}{\cosh^2(2K)}$ \cite{}.

Near the critical point, the metric can be expand as
\begin{equation}
    g_{TT}(T) \;\sim\; \frac{1}{T_c^{2}}\;\frac{2}{\pi}\left(\frac{2J}{k_{\mathrm B}T_c}\right)^{\!2}\,
		\ln\!\left|1-\frac{T}{T_c}\right|, 
		\qquad T\to T_c^\pm.
\end{equation}
and therefore the integral converges even when $T_c$ is in between $T_1$ and $T_2$. Note that $g_{TT}$ diverges as $\ln|1-T/T_c|$, which implies that the integrals in both $\mathcal{L}$ (Eq. \ref{eq:L-def}) and $\mathcal{A}$ (Eq. \ref{SalomonDissipatedAvail}) converge. Moreover, it is a (local) minimal distance  trajectory. To show that, consider a trajectory that ```bypass'' the critical point from $(t,h)=(-\epsilon,0)$ to $(t,h) = (+\epsilon,0)$ along some trajectory $(t(s), h(s))$. The length along this trajectory is
\begin{eqnarray}
    |\dot{\boldsymbol{\lambda}}|=\sqrt{g_{tt}\dot t^2 + g_{hh}(\dot h + \frac{g_{th}}{g_{hh}}\dot t)^2-\frac{g_{th}^2}{g_{hh}}\dot t^2}
\end{eqnarray}
For $h=0$, $g_{ht}=0$ (from symmetry with respect to $h$
 around $h=0$), so at fixed $\dot t$, the minimum over $\dot h$ is at $\dot h=0$. Any deviation from $\dot h=0$ increases the integral by 
 \begin{eqnarray}
     \Delta|\dot{\boldsymbol{\lambda}}|\approx \frac{g_{hh}\dot h^2}{2\sqrt{g_{tt}}}.
 \end{eqnarray}
This implies that the trajectory along the $h=0$ is a local minimum of the distance function, even though it is not a solution of the geodesic equation due to the singularity of the metric.

%Note, however, that the integral over the Sivak metric does not converge due to the critical slow down.




\bibliography{ThermodynamicGeometry}
\end{document}