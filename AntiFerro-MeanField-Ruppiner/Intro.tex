As was mentioned before, in the Antiferromagnet it is impossible to build a path that crosses between the antiferromagnetic and disordered phases without passing through a second order phase transition. Therefore, it is a prime example of a system where our analysis is necessary.

We demonstrate finding the path of least dissipation between two points $\left(T_1, h_1\right) \rightarrow \left(T_2, h_2\right)$, where $\left(T_1, h_1\right)$ is in the antiferromagnetic phase, and $\left(T_2, h_2\right)$ is in the disordered phase. We use the Ruppiner metric, as it is analytically solvable.

\subsubsection{The Model}
The model we use is the Antiferromagnetic Mean-Field model, as laid out in \cite{vivesUnifiedMeanfieldStudy1997}. It consists of spins on a square lattice, divided into two sub-lattices, such that spins interact only with spins from the other sub-lattice. Denoting the magnetization of each sub-lattice by $m_A,m_B$, the Gibbs Free energy is:
\begin{equation}
    f = \frac{1}{2} \left[ K m_A m_B - h \left( m_A + m_B \right) + \frac{1}{2} T \left( \vartheta \left(m_A\right) +\vartheta\left(m_B\right) \right) \right]
\end{equation}
With 
\begin{equation}
    \vartheta \left(x\right) = \frac{1}{2} \left[ \left(1+x\right) \log \left(1+x\right) + \left(1-x\right) \log \left(1-x\right) \right]
\end{equation}
We define $K$ to be the interaction strength $J$ times the coordination number.

In thermodynamic equilibrium $m_A,m_B$ are the minimizers of $f$, and solve the self consistent equations
\begin{align}
    \label{eq:SCExy}
    m_A = & \tanh\left(\frac{h-Km_B}{T}\right) \\
    m_B = & \tanh\left(\frac{h-Km_A}{T}\right)
\end{align}

Taking the derivative gives the Ruppiner metric, since this is the gibbs free energy, is done w.r.t $T, h$, and gives:
\begin{widetext}

    \begin{equation}
        g_{hh} = \frac{T \left(2-m_A^2-m_B^2\right) - 2 K \left(1-m_A^2\right) \left(1-m_B^2\right)}
        {2 \left[ T^2- \left(1-m_A^2\right) \left(1-m_B^2\right) K^2\right]}
    \end{equation}
    \begin{equation}
        g_{hT} = - \frac{\left(1-m_A^2\right) \arctanh(m_A) \left(T-\left(1-m_B^2\right) K\right)+\left(1-m_B^2\right) \arctanh (m_B) \left(T-\left(1-m_A^2\right) K\right)}{2 \left[ T^2- \left(1-m_A^2\right) \left(1-m_B^2\right) K^2\right]}
    \end{equation}
    \begin{equation}
        g_{TT} = \frac{T \left[ \arctanh^2( m_B ) \left(1-m_B^2\right) + \arctanh^2( m_A ) \left(1-m_A^2\right)\right] - K \left( 1-m_A^2 \right) \left(1-m_B^2\right) \arctanh(m_B) \arctanh(m_A) 
        }{2 \left[T^2- \left(1-m_A^2\right) \left(1-m_B^2\right) K^2\right]}
    \end{equation}
\end{widetext}
% With the determinant of the metric being
% % \begin{widetext}
%     \begin{equation}
%         g = \frac{\left(1-m_A^2\right) \left(1-m_B^2\right) \left(\arctanh(m_A)-\arctanh(m_B)\right)^2}{4 \left(T^2-\left(1-m_A^2\right) \left(1-m_B^2\right) (K)^2\right)}
%     \end{equation}
% % \end{widetext}
