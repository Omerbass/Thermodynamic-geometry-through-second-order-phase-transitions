


As discussed earlier, the model used as a case study is the mean-field (MF) Ising model with antiferromagnetic interactions and a uniform magnetic field, in contrast to other treatments of antiferromagnets that introduce a staggered magnetic field acting oppositely on neighboring spins.
In this model, it is impossible to construct a continuous path connecting the antiferromagnetic and disordered phases without crossing a second-order phase transition.
This property makes it an ideal test case for applying the thermodynamic geometry framework laid out above.

I now show a calculation of the Ruppiner metric as well as an interesting result concerning minimal paths in the disordered phase, that turns out to be a general result for MF models.

\subsubsection{The Model}
We consider the MF antiferromagnet introduced by Vives \textit{et al.}~\cite{vivesUnifiedMeanfieldStudy1997}. 
The system consists of spins arranged on a square lattice divided into two interpenetrating sublattices, $A$ and $B$, such that spins on one sublattice interact only with spins on the other. 
Denoting the sublattice magnetizations by $m_A$ and $m_B$, the Gibbs free energy per spin is
\begin{equation}
    f = \frac{1}{2} \!\left[ K m_A m_B - h \left(m_A + m_B\right)
    + \frac{1}{2} T \left(\vartheta(m_A) + \vartheta(m_B)\right) \right]
\end{equation}
where
\begin{equation}
    \vartheta(x) = \frac{1}{2}\!\left[(1+x)\log(1+x) + (1-x)\log(1-x)\right]
\end{equation}
and $K$ is the product of the exchange interaction $J$ and the coordination number.

At thermodynamic equilibrium, $m_A$ and $m_B$ minimize $f$ and satisfy the self-consistency relations
\begin{align}
    m_A &= \tanh\!\left(\frac{h - K m_B}{T}\right) \label{eq:SCEmA}\\
    m_B &= \tanh\!\left(\frac{h - K m_A}{T}\right) \label{eq:SCEmB}
\end{align}

Differentiating the Gibbs free energy with respect to $T$ and $h$ yields the Ruppeiner metric components:
\begin{widetext}
\begin{align}
    g_{hh} &= 
    \frac{T(2 - m_A^2 - m_B^2) - 2K(1 - m_A^2)(1 - m_B^2)}
    {2\!\left[T^2 - K^2(1 - m_A^2)(1 - m_B^2)\right]} \\[4pt]
    g_{hT} &= 
    -\frac{(1 - m_A^2)\arctanh(m_A)\!\left[T - K(1 - m_B^2)\right]
    + (1 - m_B^2)\arctanh(m_B)\!\left[T - K(1 - m_A^2)\right]}
    {2\!\left[T^2 - K^2(1 - m_A^2)(1 - m_B^2)\right]} \\[4pt]
    g_{TT} &= 
    \frac{T\!\left[\arctanh^2(m_A)(1 - m_A^2) + \arctanh^2(m_B)(1 - m_B^2)\right]
    - K(1 - m_A^2)(1 - m_B^2)\arctanh(m_A)\arctanh(m_B)}
    {2\!\left[T^2 - K^2(1 - m_A^2)(1 - m_B^2)\right]}
\end{align}
\end{widetext}
