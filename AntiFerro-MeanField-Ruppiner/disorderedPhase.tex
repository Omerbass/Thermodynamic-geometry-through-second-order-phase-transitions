\subsubsection{Disordered Phase}

In the disordered phase, the two sublattices become equivalent, $m_A = m_B \equiv m$.  
A notable property of the Ruppeiner metric in this regime is that its determinant vanishes, $g = 0$.  
This degeneracy, also observed in the ferromagnetic mean-field model \cite{janyszekRiemannianGeometryThermodynamics1989}, implies that one eigenvalue of the metric is zero.  
The corresponding null eigenvector aligns with directions of constant magnetization (equivalently, constant $m_A,m_B$).  
Consequently, motion along these directions incurs zero thermodynamic length, and the problem reduces effectively to a one-dimensional minimization:  
at what distance from the critical line is the thermodynamic length minimal?
As shown below, the minimum occurs precisely on the phase transition line itself.

For clarity, we denote by $T_c$ the critical temperature corresponding to a given magnetization.  
From differentiation of the free energy, two key relations follow along the phase transition line:
\begin{equation}
    m_A = m_B = m \equiv \pm \sqrt{1 - \frac{T_c}{K}}
\end{equation}
and
\begin{equation} \label{eq:h_c(T_c)}
    h_c(T_c) = \pm\!\left[ K\sqrt{1 - \frac{T_c}{K}}
    + T_c\, \arctanh\!\left(\sqrt{1 - \frac{T_c}{K}}\right) \right]
\end{equation}

In the disordered phase ($m_A = m_B = m$), the Ruppeiner metric simplifies to
\begin{equation}
    g_{ij} = \frac{T_c}{K (T + T_c)}
    \begin{pmatrix}
        \arctanh^2(m) & -\arctanh(m) \\
        -\arctanh(m) & 1
    \end{pmatrix}
\end{equation}
Its eigenvalues and eigenvectors are
\begin{align}
    \lambda_1 &= 0, \quad 
    \boldsymbol{v}_1 =
    \begin{pmatrix}
        1 \\[2pt]
        \arctanh(m)
    \end{pmatrix} \\
    \lambda_2 &= 1 + \arctanh^2(m), \quad 
    \boldsymbol{v}_2 =
    \begin{pmatrix}
        \arctanh(m) \\[2pt]
        -1
    \end{pmatrix}
\end{align}
The null direction $\boldsymbol{v}_1$ defines “zero-length” lines, corresponding to constant-$m$ curves in the $(T,h)$ plane (see Fig.~\ref{fig:0length}).

\begin{figure}[h]
    \centering
    \includegraphics[width=\linewidth]{figures/disorderedPhaseLines.jpg}
    \caption{Lines of zero thermodynamic length ($\lambda_1=0$) in the disordered phase.}
    \label{fig:0length}
\end{figure}

To find the minimal thermodynamic length, we consider displacements along the $\boldsymbol{v}_1$ direction.  
Since $g_{\mu\nu}$ scales as $(T + T_c)^{-1}$, while the differential displacement scales as the Euclidean distance projected onto the normal of the constant-magnetization lines, the relevant scaling is determined by the distance between neighboring constant-$m$ curves.

Geometrically, this “radial distance” is proportional to a displacement along the $T$-axis, with some offset $T_0$ -- since the constant-$m$ lines are neither vertical nor horizontal except at symmetry points.  
Let $(T_c, h_c)$ denote a point on the phase transition line and $(T_c + \delta_T,\, h_c + \frac{\partial h_c}{\partial T_c}\delta_T)$ a neighboring point.  
The intersection of the two constant-$m$ lines can be expressed as
\begin{align}
    \begin{pmatrix} T_0 \\ h_0 \end{pmatrix}
    &= a\, \boldsymbol{v}_1(T_c)
       + \begin{pmatrix} T_c \\ h_c \end{pmatrix} \label{eq:T0h0_1}\\
    \begin{pmatrix} T_0 \\ h_0 \end{pmatrix}
    &= b\, \boldsymbol{v}_1(T_c + \delta_T)
       + \begin{pmatrix} T_c + \delta_T \\[2pt]
         h_c + \frac{\partial h_c}{\partial T_c}\delta_T \end{pmatrix}
       \label{eq:T0h0_2}
\end{align}
Subtracting Eqs.~\eqref{eq:T0h0_1}–\eqref{eq:T0h0_2} gives
\begin{equation}
    \delta_T 
    \begin{pmatrix}
        1 \\[2pt]
        \frac{\partial h_c}{\partial T_c}
    \end{pmatrix}
    =
    \begin{pmatrix}
        1 & 1 \\[2pt]
        \arctanh[m(T_c + \delta_T)] & \arctanh[m(T_c)]
    \end{pmatrix}
    \begin{pmatrix}
        -b \\[2pt] a
    \end{pmatrix}
\end{equation}
Inverting this matrix yields
\begin{widetext}
\begin{equation}
\begin{aligned}
    \begin{pmatrix}
        1 & 1 \\[2pt]
        \arctanh[m(T_c + \delta_T)] & \arctanh[m(T_c)]
    \end{pmatrix}^{-1}
    &= 
    \frac{1}{\arctanh[m(T_c)] - \arctanh[m(T_c + \delta_T)]}
    \begin{pmatrix}
        \arctanh[m(T_c)] & -1 \\[2pt]
        -\arctanh[m(T_c + \delta_T)] & 1
    \end{pmatrix} \\[4pt]
    &\simeq
    -\frac{2T_c\sqrt{1 - \frac{T_c}{K}}}{K\,\delta_T}
    \begin{pmatrix}
        \arctanh\!\sqrt{1 - \frac{T_c}{K}} & -1 \\[2pt]
        -\arctanh\!\sqrt{1 - \frac{T_c + \delta_T}{K}} & 1
    \end{pmatrix}
\end{aligned}
\end{equation}
\end{widetext}

Solving for $a$ gives
\begin{equation} \label{eq:a_raw}
    a = \frac{2T_c}{K}\!\left[
        \arctanh\!\sqrt{1 - \frac{T_c}{K}}
        - \frac{\partial h_c}{\partial T_c}\sqrt{1 - \frac{T_c}{K}}
    \right]
\end{equation}
Differentiating Eq.~\eqref{eq:h_c(T_c)} yields
\begin{equation}
    \frac{\partial h_c}{\partial T_c}
    = -\frac{1}{\sqrt{1 - \frac{T_c}{K}}}
      + \arctanh\!\sqrt{1 - \frac{T_c}{K}}
\end{equation}
Substituting this into Eq.~\eqref{eq:a_raw} gives a remarkably simple result:
\begin{equation}
    a = \frac{2T_c}{K}
\end{equation}

Substituting back into Eq.~\eqref{eq:T0h0_1} yields
\begin{equation}
    T_0 = T_c + \frac{2T_c}{K} = T_c\,\frac{K + 2}{K}
\end{equation}
Thus, the differential thermodynamic length between two neighboring constant-magnetization lines in the disordered phase scales as
\begin{equation}
    \frac{(T + \frac{K + 2}{K}T_c)^2}{T + T_c}
\end{equation}
which attains its minimum at $T = T_c$.  
Therefore, the least-dissipation path in the disordered phase lies precisely along the phase transition line.