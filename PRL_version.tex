\documentclass[%
 reprint,
%superscriptaddress,
%groupedaddress,
%unsortedaddress,
%runinaddress,
%frontmatterverbose, 
%preprint,
%preprintnumbers,
%nofootinbib,
%nobibnotes,
%bibnotes,
 amsmath,amssymb,
 aps,
%pra,
%prb,
%rmp,
%prstab,
%prstper,
%floatfix,
]{revtex4-2}

% For Thesis
% \documentclass[pra,preprint, ,superscriptaddress,notitlepage,longbibliography]{revtex4-2}
% \usepackage{anyfontsize}
% \renewcommand{\normalsize}{\fontsize{12}{14}\selectfont}
% \linespread{1.5}
% \usepackage[margin=1.5cm]{geometry}
% \usepackage{newtxtext,newtxmath}

\usepackage[normalem]{ulem}
\usepackage{graphicx}% Include figure files
\usepackage{dcolumn}% Align table columns on decimal point
\usepackage{bm}% bold math
\usepackage{cancel}
\usepackage{nicefrac}
\usepackage{multirow}
\usepackage[dvipsnames]{xcolor} % for colors 
\usepackage{xspace}
\usepackage[hidelinks]{hyperref}
\usepackage{caption}
\usepackage{subcaption}
\usepackage{comment}

%\newcommand{\oren}[1]{{\color{red} [Oren:]\xspace}}%\usepackage{hyperref}% add hypertext capabilities
\newcommand{\oren}[1]{{\color{red} [Oren: #1]\xspace}}
\newcommand{\omer}[1]{{\color{ForestGreen} [Omer: #1]\xspace}}
\newcommand{\arctanh}[0]{\text{arctanh}}
\newcommand{\red}[1]{{\color{red}{#1}}}
\newcommand{\done}[0]{{\color{ForestGreen} \checkmark \quad}}

%\usepackage[mathlines]{lineno}% Enable numbering of text and display math
%\linenumbers\relax % Commence numbering lines

%\usepackage[showframe,%Uncomment any one of the following lines to test 
%%scale=0.7, marginratio={1:1, 2:3}, ignoreall,% default settings
%%text={7in,10in},centering,
%%margin=1.5in,
%%total={6.5in,8.75in}, top=1.2in, left=0.9in, includefoot,
%%height=10in,a5paper,hmargin={3cm,0.8in},
%]{geometry}


\begin{document}

\preprint{APS/123-QED}

\title{Thermodynamic Geometry Through Second Order Phase Transitions}

\author{Omer Michael Basri}
 \email{omer.basri@weizmann.ac.il}
\author{Oren Raz}
 \email{oren.raz@weizmann.ac.il}
\affiliation{%
 Department of Physics of Complex Systems,\\Weizmann Institute of Science, Rehovot 76100, Israel. 
}%

\date{\today}

\begin{abstract}
A novel approach for calculating excess dissipation in slowly driven thermodynamic processes is through the use of a Riemannian metric on the space of control parameters, where optimal driving protocols follow geodesics.
Near phase transitions, this geometric picture breaks down as the metric diverges and geodesics may cease to exist.
Using the scaling hypothesis, we analyze this framework across several universality classes and show that the thermodynamic length can remain finite.
We then demonstrate a numerical approach for computing minimal paths in such systems. We show that, in some regimes, the shortest path crosses the phase transition -- even when alternative paths confined to a single phase exist.
\end{abstract}

\maketitle


\paragraph{Introduction}
Since the earliest developments of equilibrium statistical mechanics to modern nonequilibrium statistical mechanics there have been attempts to cast the theory in a geometric form \cite{gibbs1873method,ruppeinerThermodynamicsRiemannianGeometric1979, weinholdMetricGeometryEquilibrium1975,crooksMeasuringThermodynamicLength2007,janyszekGeometricalStructureState1989,PhysRevE.111.034113,raz2016geometric,van2024geometric,brandner2020thermodynamic,RevModPhys.67.605, mandalAnalysisSlowTransitions2016,zhong2024beyond}. 

One of the most useful approaches is to regard the excess dissipation as a Riemannian metric. This idea was introduced already in the 80's \cite{salamonThermodynamicLengthDissipated1983,salamonRelationEntropyEnergy1984}, following purely geometrical concepts suggested already in the 70's \cite{ruppeinerThermodynamicsRiemannianGeometric1979, weinholdMetricGeometryEquilibrium1975}. The framework in  \cite{salamonThermodynamicLengthDissipated1983,salamonRelationEntropyEnergy1984} required two assumptions: (i)endoreversibility---the system remains in internal equilibrium throughout the process, but may be out of equilibrium with its environment; (ii) the dissipation is only a function of the system's mean response time. The latter assumption restricts the formalism’s applicability to systems with nearly constant relaxation time. Despite these limitations, the framework has proven valuable for identifying minimal-entropy-production protocols in processes such as chemical reactions and distillation \cite{Andresen2000,andresenCurrentTrendsFiniteTime2011}. 

The two limiting assumptions were lifted by using a closely related metric, which only assumes linear response theory to be applicable \cite{sivakThermodynamicMetricsOptimal2012, zulkowski2012geometry,rotskoffDynamicRiemannianGeometry2015,sivak2016thermodynamic,blaber2020skewed, rotskoffGeometricApproachOptimal2017}. In this formulation, the system’s relaxation dynamics are incorporated into the metric, hence removing the dependence on a constant response-time and eliminating the endoreversible assumption. This generalized framework extends thermodynamic geometry to a much broader class of systems---including those with widely varying time scales and microscopic systems such as biomolecular or single-molecule systems---and has found applications in optimizing computational and physical processes \cite{rotskoffGeometricApproachOptimal2017}.

A particularly challenging class of systems exhibiting widely separated time scales are those undergoing phase transitions. The application of thermodynamic geometry to such systems has drawn significant attention \cite{janyszekRiemannianGeometryThermodynamics1989, PhysRevE.51.1006}, including efforts to identify optimal driving protocols across critical regions (e.g., the 2D ferromagnetic Ising model \cite{rotskoffDynamicRiemannianGeometry2015}). In many cases of interest, e.g. antiferromagnets \cite{vivesUnifiedMeanfieldStudy1997}, crossing the phase transition cannot be avoided as the two phases are topologically separated by a phase transition surface. Yet, the divergence of thermodynamic length near phase transitions has received limited study. Notably, a divergence in the metric tensor does not necessarily imply a divergence in thermodynamic length. However, even if the length remains finite, other geometric quantities such as the curvature may diverge, signaling the critical behavior. 

In this paper, we analyze the behavior of the thermodynamic metric near second-order phase transitions. Using the Widom scaling and the dynamical scaling hypothesis, we show that in some class of models, the divergence of the metric at the phase transition does not imply divergence of dissipation when crossing the phase transition. Therefore, the thermodynamic distance between states in the different phases is finite. The class of models with this property includes several mean field models, as well as the Ising model in dimensions 3 and above. To demonstrate our findings, we apply the formalism to the mean field antiferromagnetic Ising model, where  we determine the optimal protocols connecting two points. Surprisingly, we find that in some cases the optimal protocol between two points that are in the same phase nevertheless crosses to the other phase and back.


\paragraph{Framework} \label{sec:framework}
Here we briefly define the key metrics and scaling tools used in our study. The dissipation metric, first introduced in \cite{sivakThermodynamicMetricsOptimal2012}, is defined as
\begin{equation}
    g_{\mu\nu} = \beta \int_0^\infty \langle \delta X_\mu(t)\, \delta X_\nu(0) \rangle\, dt =
    \beta \mathcal{T}_{\mu\nu} \frac{\partial^2 \ln Z}{\partial \lambda^\mu \partial \lambda^\nu}
\end{equation}
where $\mathcal{T}_{\mu\nu}$ is the time-response matrix, $Z$ the partition function, and $X_\mu$ the conjugate variable to $\lambda^\mu$ -- the natural variables of the free energy, e.g., $\{X_1,X_2\} = \{S, M\}$ for $F(T, h)$.
The dissipated availability, which expresses the excess dissipation in a performing the trajectory $\boldsymbol{\lambda(t)}$ in parameter space, can be expressed as
\begin{equation}
    \mathcal{A} = \int_{\boldsymbol{\lambda}(t)} g_{\mu\nu}\, \dot{\lambda}^\mu \dot{\lambda}^\nu\, dt.
\end{equation}
It is related to the length associated with the thermodynamic metric, defined by
\begin{equation}
    \mathcal{L} = \int_{\boldsymbol{\lambda}(s)} \sqrt{g_{\mu\nu}\, \dot{\lambda}^\mu \dot{\lambda}^\nu}\, ds
\end{equation}
through the inequality
\begin{equation}\label{Eq:A_bound}
    \mathcal{A} \ge \frac{\mathcal{L}^2}{\tau}
\end{equation}
 (here $\tau$ is the protocol duration, and $\lambda(s)$ parametrize the trajectory in parameter space). This bound is saturated by a constant-speed protocols, where $g_{\mu\nu}\, \dot{\lambda}^\mu \dot{\lambda}^\nu=const.$ throughout the protocol.

%\subsection{Scaling Hypothesis} \label{subsec:ScalingHypothesisFramework}
Near second order phase transitions, second order derivatives of the free energy derivatives, and hence the dissipation metric diverges as well. The asymptotic form of these divergences is determined by critical exponents defining universality classes. These critical exponents can be analyzed using the scaling hypothesis, which relates the free energy $\varphi$ near the critical point $(T_c, H_c)$ to reduced variables, which for models in the same universality class as the Ising model are often chosen to be $t = (T - T_c)/T_c$ and $h = (H - H_c)/H_c$. The scaling hypothesis then states that
\begin{equation} \label{eq:ScalingHyp}
    \varphi \propto t^b f\!\left(\frac{h}{t^\Delta}\right),
\end{equation}
where $f(x)$ is continuous and $f(x) \sim x^p$ as $x \to \infty$ \cite{Kardar_2007}.
The exponents $b$, $\Delta$, $p$, are directly related to the standard critical exponents through the hyperscaling relations, e.g., $b = 2 - \alpha$ from the heat capacity exponent $\alpha$.
This approach captures divergences of the free energy second derivatives across phase transitions, but does not address relaxation times, which are critical for the dissipation metric. The dynamic scaling hypothesis relates the relaxation time $\tau$ and correlation length $\xi$ as
\begin{equation}
    \tau \propto \xi^z
\end{equation}
and the free energy scales as $\varphi \propto \xi^{-d}$ \cite{Kardar_2007, tauberCriticalDynamicsField2014}.
Combining these yields
\begin{equation}\label{eq:TimeResponsScaling}
    \tau \propto \varphi^{-z/d} \propto t^{-bz/d}\, f^{-z/d}\!\left(\frac{h}{t^\Delta}\right)
\end{equation}
linking dynamical and thermodynamic critical behavior.

%\subsection{Divergence of Lengths}
With the scaling hypothesis it is thus possible to find the divergence of the metric near the phase transition at various equivalence classes, but  diverging metric does not necessarily imply a divergent thermodynamic length.
Since it is always possible to saturate the equality in  Eq.(\ref{Eq:A_bound}),  finite thermodynamic length guarantees the existence of a finite-dissipation protocol. In addition, from the Cauchy-Schwarz inequality $\mathcal{L} \to \infty$ implies $\mathcal{A} \to \infty$.
Consequently, the convergence of thermodynamic length determines whether finite-dissipation protocols exist. 
If $\ell(s) = \sqrt{g_{\mu\nu}\dot{\lambda}^\mu\!\left(s\right)\dot{\lambda}^\nu\!\left(s\right)} \propto s^{-a}$ near criticality (as $s \rightarrow 0$), the total length diverges only for $a > 1$. 

Note that this observation does not necessarily inform  us about the metric itself, or intrinsic geometry of the manifold. The metric might diverge for flat space \cite{PhysRev.119.1743,Szekeres:1960gm}, and the manifold might have divergent curvature despite having finite lengths and a non-divergent metric. 


\paragraph{Results}

%\subsection{Scaling Hypothesis} \label{subsec:ScalingHypothesisResults}

%\subsubsection{Scaling Hypothesis for the Free Energy}
The metric near a phase transition diverges when either one of the free energy derivatives or the time responses diverge. Assuming the singular part of the thermodynamic potential scales as Eq.(\ref{eq:ScalingHyp}), the scaling of the derivatives depends on the direction of the approach to criticality. Specifically, the two possible asymptotic behaviors are 
\begin{align}
    \left(\frac{\partial^2 \ln Z}{\partial \lambda^\mu \partial \lambda^\nu}\right) \Big|_{\frac{h}{t^\Delta} \to \text{const.}} &\propto 
    \begin{pmatrix}
        t^{b-2} & t^{b-\Delta-1} \\
        t^{b-\Delta-1} & t^{b-2\Delta}
    \end{pmatrix}, \\[4pt]
    \left(\frac{\partial^2 \ln Z}{\partial \lambda^\mu \partial \lambda^\nu}\right) \Big|_{\frac{h}{t^\Delta} \to \infty} &\propto
    \begin{pmatrix}
        t^{-2}h^p & t^{-1}h^{p-1} \\
        t^{-1}h^{p-1} & h^{p-2}
    \end{pmatrix}.
\end{align}
The time response divergence can be obtained from  Eq.(\ref{eq:TimeResponsScaling}). From the static scaling relation, we obtain that the divergence of $\tau$ follows
\begin{equation}
    \tau \propto
    \begin{cases}
        t^{-bz/d}, & \frac{h}{t^\Delta} \to \text{const.} \\[4pt]
        h^{-pz/d}, & \frac{h}{t^\Delta} \to \infty
    \end{cases}
\end{equation}
Together, the derivatives of the free energy and the response time imply the following scaling: For $\frac{h}{t^\Delta} \to \text{const.}$, one finds that
\begin{align}
    g_{tt}\,\dot{t}^2 &\propto s^{b(1-z/d)-2} \\
    g_{th}\,\dot{t}\,\dot{h} &\propto s^{b(1-z/d)-2+(k-\Delta)} \\
    g_{hh}\,\dot{h}^2 &\propto s^{b(1-z/d)-2+2(k-\Delta)}
\end{align}
Thus, the convergence condition becomes
\begin{equation}
    b\!\left(1 - \frac{z}{d}\right) > 0
\end{equation}
Similarly, for $\frac{h}{t^\Delta} \to \infty$,
\begin{equation}
    g_{ij}\dot{x}^i\dot{x}^j \propto s^{p(1-z/d)-2}
\end{equation}
so convergence requires
\begin{equation}
    p\!\left(1 - \frac{z}{d}\right) > 0
\end{equation}
These relations can be used to explore the distance divergence across phase transitions in various models. As summarized in Table~\ref{tab:dynamical-divergences}, the thermodynamic lengths diverge for the 2D Ising and 2D Potts models. Surprisingly, however, it remains finite for the 3D Ising model.

\begin{table}[h]
    \centering
    \begin{tabular}{c|c||c|c|c}
        \multicolumn{2}{c||}{} & $d$ & $z$ & $z/d$ \\
        \hline \hline
        \multirow{2}{*}{2D Ising} & Class A & $2$ & $2.165(10)$ & $1.083(5)$ \\
                                  & Class B & $2$ & $2.235(10)$ & $1.117(5)$ \\
        \hline
        \multicolumn{2}{c||}{3D Ising} & $3$ & $2.032(4)$ & $0.677(1)$ \\
        \hline
        \multicolumn{2}{c||}{2D 3-state Potts} & $2$ & $2.198(2)$ & $1.099(1)$ \\
        \hline
        \multicolumn{2}{c||}{2D 4-state Potts} & $2$ & $2.290(3)$ & $1.095(2)$ \\
    \end{tabular}
    \caption{Dynamical scaling parameters for representative universality classes \cite{odor_universality_2004}.}
    \label{tab:dynamical-divergences}
\end{table}

\begin{comment}
This implies that to study length divergences, we must specify the path in parameter space, which determines the scaling of $\frac{h}{t^\Delta}$.  We therefore consider the two relevant cases.
Case 1: $\frac{h}{t^\Delta} \to \text{const.}$ Near the phase transition, such a trajectory can be parametrized using
\begin{equation}
    \begin{pmatrix}
        t \\ h
    \end{pmatrix} =
    \begin{pmatrix}
        s \\ s^k
    \end{pmatrix}, \quad k \ge \Delta
\end{equation}
for which
\begin{equation*}
    \frac{h}{t^\Delta} \xrightarrow[s \to 0]{} \text{const.}
\end{equation*}
The other case is $\frac{h}{t^\Delta} \to \infty$, which can be parametrized using 
\begin{equation}
    \begin{pmatrix}
        t \\ h
    \end{pmatrix} =
    \begin{pmatrix}
        s^{1/k} \\ s
    \end{pmatrix}, \quad k < \Delta
\end{equation}
which implies
\begin{equation*}
    \frac{h}{t^\Delta} \xrightarrow[s \to 0]{} \infty
\end{equation*}

The metric contributions then scale as
\begin{align*}
    g_{tt} \dot{t}^2 &\propto s^{b-2} \\
    g_{th} \dot{t}\dot{h} &\propto s^{b-2+(k-\Delta)} \\
    g_{hh} \dot{h}^2 &\propto s^{b-2+2(k-\Delta)}
\end{align*}
Since $k \ge \Delta$, the sufficient condition for convergence of the Ruppeiner length $\mathcal{L}$ is
\begin{equation}
    b > 0
\end{equation}
As established earlier, convergence of $\mathcal{L}$ is necessary and sufficient for convergence of the Salamon dissipated availability $\mathcal{A}$ in finite-time protocols, provided $\bar{\epsilon}$ remains finite.

\paragraph{Case 2: $\frac{h}{t^\Delta} \to \infty$ \label{par:scalingCase_2}}
\begin{equation}
    \begin{pmatrix}
        t \\ h
    \end{pmatrix} =
    \begin{pmatrix}
        s^{1/k} \\ s
    \end{pmatrix}, \quad k < \Delta
\end{equation}
implying
\begin{equation*}
    \frac{h}{t^\Delta} \xrightarrow[s \to 0]{} \infty
\end{equation*}
In this case,
\begin{align*}
    g_{tt} \,\dot{t}^2 &\propto s^{p-2} &
    g_{th} \,\dot{t}\,\dot{h} &\propto s^{p-2} &
    g_{hh} \,\dot{h}^2 &\propto s^{p-2}
\end{align*}
Hence, convergence requires
\begin{equation}
    p > 0
\end{equation}
Both conditions ($b > 0$ and $p > 0$) hold for several universality classes, as shown in Table~\ref{tab:scaling-divergences}.

\begin{table}[h]
    \centering
    \begin{tabular}{c||c|c}
        & $b$ & $p$ \\
        \hline \hline
        Mean Field & $2$ & $\tfrac{4}{3}$ \\
        2D Ising & $2$ & $\tfrac{16}{15}$ \\
        3D Ising & $1.8899(3)$ & $1.2088(1)$ \\
        2D 3-state Potts & $\tfrac{5}{3}$ & $\tfrac{15}{14}$ \\
        2D 4-state Potts & $\tfrac{4}{3}$ & $\tfrac{16}{15}$ \\
        % 3D XY Model & $ 2.0153(3) $ & $ 1.20923(6) $
    \end{tabular}
    \caption{Scaling parameters for representative universality classes.}
    \label{tab:scaling-divergences}
\end{table}

%\subsubsection{Dynamical Scaling Hypothesis}
In both Ruppeiner and Sivak formalisms, dissipation depends on the system’s response time—appearing as $\bar{\epsilon}$ in the former and directly in the metric for the latter.  
Using the dynamical scaling hypothesis, we can study how the relaxation time $\tau$ diverges.

From Eq.(\ref{eq:TimeResponsScaling}) and the static scaling relation, we obtain
that the divergence of $\tau$ follows
\begin{equation}
    \tau \propto
    \begin{cases}
        t^{-bz/d}, & \frac{h}{t^\Delta} \to \text{const.} \\[4pt]
        h^{-pz/d}, & \frac{h}{t^\Delta} \to \infty
    \end{cases}
\end{equation}
Together, the derivatives of the free energy and the response time imply the following scaling in the  \hyperref[par:scalingCase_1]{Case 1}
\begin{align}
    g_{tt}\,\dot{t}^2 &\propto s^{b(1-z/d)-2} \\
    g_{th}\,\dot{t}\,\dot{h} &\propto s^{b(1-z/d)-2+(k-\Delta)} \\
    g_{hh}\,\dot{h}^2 &\propto s^{b(1-z/d)-2+2(k-\Delta)}
\end{align}
Thus, the convergence condition becomes
\begin{equation}
    b\!\left(1 - \frac{z}{d}\right) > 0
\end{equation}
Similarly, in \hyperref[par:scalingCase_2]{Case 2}
\begin{equation}
    g_{ij}\dot{x}^i\dot{x}^j \propto s^{p(1-z/d)-2}
\end{equation}
so convergence requires
\begin{equation}
    p\!\left(1 - \frac{z}{d}\right) > 0
\end{equation}

As summarized in Table~\ref{tab:dynamical-divergences}, the thermodynamic lengths diverge for the 2D Ising and 2D Potts models but remain finite for the 3D Ising model.

\begin{table}[h]
    \centering
    \begin{tabular}{c|c||c|c|c}
        \multicolumn{2}{c||}{} & $d$ & $z$ & $z/d$ \\
        \hline \hline
        \multirow{2}{*}{2D Ising} & Class A & $2$ & $2.165(10)$ & $1.083(5)$ \\
                                  & Class B & $2$ & $2.235(10)$ & $1.117(5)$ \\
        \hline
        \multicolumn{2}{c||}{3D Ising} & $3$ & $2.032(4)$ & $0.677(1)$ \\
        \hline
        \multicolumn{2}{c||}{2D 3-state Potts} & $2$ & $2.198(2)$ & $1.099(1)$ \\
        \hline
        \multicolumn{2}{c||}{2D 4-state Potts} & $2$ & $2.290(3)$ & $1.095(2)$ \\
    \end{tabular}
    \caption{Dynamical scaling parameters for representative universality classes \cite{odor_universality_2004}.}
    \label{tab:dynamical-divergences}
\end{table}
\end{comment}

%\subsection{Antiferromagnetic Mean-Field Model - Ruppiner metric}
%As was mentioned before, in the Antiferromagnet it is impossible to build a path that crosses between the antiferromagnetic and disordered phases without passing through a second order phase transition. Therefore, it is a prime example of a system where our analysis is necessary.

We demonstrate finding the path of least dissipation between two points $\left(T_1, h_1\right) \rightarrow \left(T_2, h_2\right)$, where $\left(T_1, h_1\right)$ is in the antiferromagnetic phase, and $\left(T_2, h_2\right)$ is in the disordered phase. We use the Ruppiner metric, as it is analytically solvable.

\subsubsection{The Model}
The model we use is the Antiferromagnetic Mean-Field model, as laid out in \cite{vivesUnifiedMeanfieldStudy1997}. It consists of spins on a square lattice, divided into two sub-lattices, such that spins interact only with spins from the other sub-lattice. Denoting the magnetization of each sub-lattice by $m_A,m_B$, the Gibbs Free energy is:
\begin{equation}
    f = \frac{1}{2} \left[ K m_A m_B - h \left( m_A + m_B \right) + \frac{1}{2} T \left( \vartheta \left(m_A\right) +\vartheta\left(m_B\right) \right) \right]
\end{equation}
With 
\begin{equation}
    \vartheta \left(x\right) = \frac{1}{2} \left[ \left(1+x\right) \log \left(1+x\right) + \left(1-x\right) \log \left(1-x\right) \right]
\end{equation}
We define $K$ to be the interaction strength $J$ times the coordination number.

In thermodynamic equilibrium $m_A,m_B$ are the minimizers of $f$, and solve the self consistent equations
\begin{align}
    \label{eq:SCExy}
    m_A = & \tanh\left(\frac{h-Km_B}{T}\right) \\
    m_B = & \tanh\left(\frac{h-Km_A}{T}\right)
\end{align}

Taking the derivative gives the Ruppiner metric, since this is the gibbs free energy, is done w.r.t $T, h$, and gives:
\begin{widetext}

    \begin{equation}
        g_{hh} = \frac{T \left(2-m_A^2-m_B^2\right) - 2 K \left(1-m_A^2\right) \left(1-m_B^2\right)}
        {2 \left[ T^2- \left(1-m_A^2\right) \left(1-m_B^2\right) K^2\right]}
    \end{equation}
    \begin{equation}
        g_{hT} = - \frac{\left(1-m_A^2\right) \arctanh(m_A) \left(T-\left(1-m_B^2\right) K\right)+\left(1-m_B^2\right) \arctanh (m_B) \left(T-\left(1-m_A^2\right) K\right)}{2 \left[ T^2- \left(1-m_A^2\right) \left(1-m_B^2\right) K^2\right]}
    \end{equation}
    \begin{equation}
        g_{TT} = \frac{T \left[ \arctanh^2( m_B ) \left(1-m_B^2\right) + \arctanh^2( m_A ) \left(1-m_A^2\right)\right] - K \left( 1-m_A^2 \right) \left(1-m_B^2\right) \arctanh(m_B) \arctanh(m_A) 
        }{2 \left[T^2- \left(1-m_A^2\right) \left(1-m_B^2\right) K^2\right]}
    \end{equation}
\end{widetext}
% With the determinant of the metric being
% % \begin{widetext}
%     \begin{equation}
%         g = \frac{\left(1-m_A^2\right) \left(1-m_B^2\right) \left(\arctanh(m_A)-\arctanh(m_B)\right)^2}{4 \left(T^2-\left(1-m_A^2\right) \left(1-m_B^2\right) (K)^2\right)}
%     \end{equation}
% % \end{widetext}

%\subsubsection{Disordered Phase}
One interesting property that can be readily observed is that in the disordered phase, where $m_A=m_B\equiv m$, one gets that $g=0$, i.e the determinant of the metric vanishes. This has been observed before in the Ferromagnetic mean-field model \cite{janyszekRiemannianGeometryThermodynamics1989}. It can be shown that the 0 eigenvectors in the disordered phase point along lines of constant magnetization (or equivalently lines of constant $m_A$). As a result, the problem becomes a 1D minimization problem - at what distance from the phase transition line will the length be minimal? As we show below, the answer is simply at the phase transition line itself.

In the following section we will assume when writing $T_c$ that it is the critical temperature for a specific magnetization. Two central results that can be easily proven by derivation of the free energy are that on the phase transition line
\begin{equation}
    m_A=m_B=\pm \sqrt{1-\frac{T_c}{K}}
\end{equation}
And
\begin{equation} \label{eq:h_c(T_c)}
    h_c (T_c) = \pm\left(K\sqrt{1-\frac{T_c}{K}} + T_c \arctanh\left( \sqrt{1-\frac{T_c}{K}} \right) \right)
\end{equation}


The metric becomes in the disordered phase ($m_A=m_B$)
\begin{equation}
    g_{ij} = \frac{T_c}{K \left(T+T_c\right)}
    \left( \begin{array}{cc}
         \arctanh(m)^2 & -\arctanh(m) \\
         -\arctanh(m) & 1 \\
    \end{array} \right)
\end{equation}
The eigenvalues and eigenvectors of this metric are
\begin{align}
    \lambda_1 & = 0,\quad \boldsymbol{v}_1 = \left( \begin{array}{c}
         1  \\
        \arctanh(m_A)
    \end{array} \right) \\
    \lambda_2 & = \arctanh^2(m_A)+1,\quad \boldsymbol{v}_2 = \left( \begin{array}{c}
        \arctanh(m_A) \\
         -1 
    \end{array} \right) 
\end{align}
One can see an illustration of the "0-length" lines in Fig. (\ref{fig:0length}).

\begin{figure}
    \centering
    \includegraphics[width=1\linewidth]{figures/disorderedPhaseLines.jpg}
    \caption{lines of "0-length" in disordered phase}
    \label{fig:0length}
\end{figure}

And so we need to minimize the expression
$ \sqrt{\dot{\lambda}^\mu g_{\mu\nu} \dot{\lambda}^\nu} \,dt$
along the $\boldsymbol{v}_1$ axis. We know $g_{\mu\nu}$ scales as $\left(T+T_c\right)^{-1}$. We will argue that the rest of the expression scales as $T-T_0$ for some $T_0$.

\omer{It feels to me like the next part could be made more precise and clear with some statement of curvature, this is just how I  calculated it originally, but if you can formalize it better I would be happy.}
The rest of the expression is simply euclidean distance projected onto the normal to the lines of constant magnetization. Hence, it should scale as the radial distance from the point of meeting of two adjacent lines of constant magnetization.
The radial distance, since the lines aren't vertical or horizontal (except for one) is proportional to the distance along the $T$-axis.

Taking some point $\left(T_c,h_c\right)$ and an adjacent point on the phase transition line
\begin{equation*}
    (T_c+\delta_T, h_c + \frac{\partial h_c}{\partial T_c}\delta_T)
\end{equation*}
We want the intersection of the lines (i.e, to solve for $T_0$):
\begin{align} \label{eq:T0,h0}
    \left( \begin{array}{c}
         T_0 \\
         h_0
    \end{array}\right) & = a \, \boldsymbol{v}_1\left(T_c\right) + \left( \begin{array}{c}
         T_c \\
         h_c
    \end{array}\right) \\
    \left( \begin{array}{c}
         T_0 \\
         h_0
    \end{array}\right) & = b \, \boldsymbol{v}_1\left(T_c + \delta_T \right) + \left( \begin{array}{c}
         T_c + \delta_T \\
         h_c + \frac{\partial h_c}{\partial T_c}\delta_T
    \end{array}\right)
\end{align}

We first subtract the equations to get
\begin{equation}
    \delta_T \left( \begin{array}{c}
         1  \\
         \frac{\partial h_c}{\partial T_c}
    \end{array}\right) = \left( \begin{array}{cc}
        1 & 1 \\
        \arctanh(m(T_c+ \delta_T)) & \arctanh(m(T_c))
    \end{array} \right)
    \left( \begin{array}{c}
         -b  \\
         a
    \end{array}\right)
\end{equation}
Inverting the matrix gives:
\begin{widetext}
\begin{equation}
    \begin{array}{rl}
        \left( \begin{array}{cc}
            1 & 1 \\
            \arctanh(m(T_c+\delta_T)) & \arctanh(m(T_c))
        \end{array} \right)^{-1} & = \frac{1}{\arctanh\left(m(T_c)\right) - \arctanh\left(m(T_c + \delta_T)\right)}\left( \begin{array}{cc}
            \arctanh(m(T_c)) & -1 \\
            -\arctanh(m(T_c + \delta_T)) & 1
        \end{array} \right) \\ & = -\frac{2T_c \sqrt{1-\frac{T_c}{K }}}{K  \, \delta_T}\left( \begin{array}{cc}
            \arctanh\left(\sqrt{1-\frac{T_c}{K }}\right) & -1 \\
            -\arctanh\left(\sqrt{1-\frac{T_c + \delta_T}{K }}\right) & 1
        \end{array} \right)
    \end{array}
\end{equation}
giving
\begin{equation} \label{eq:a}
    a = \frac{2}{K }\arctanh\left(\sqrt{1-\frac{T_c}{K }}\right) T_c \sqrt{1-\frac{T_c}{K }} - 
    \frac{2}{K } T_c \sqrt{1-\frac{T_c}{K }} \frac{\partial h_c}{\partial T_c}
\end{equation}
\end{widetext}

Deriving equation \ref{eq:h_c(T_c)} one gets
\begin{equation}
    \frac{\partial h_c}{\partial T_c} = -\frac{1}{\sqrt{1-\frac{T_c}{K}}} + \arctanh\sqrt{1-\frac{T_c}{K }}
\end{equation}

And substituting this into equation \ref{eq:a} gives
\begin{equation}
    a = \frac{2 T_c}{ K }
\end{equation}

By substituting this result into \ref{eq:T0,h0} we get
\begin{equation}
    T_0 = T_c + \frac{2}{K} T_c = \frac{K + 2}{K} T_c
\end{equation}

Accordingly, We get that the differential length between two different points in the disordered phase with different magnetization scales as
\begin{equation}
    \frac{\left(T + \frac{K + 2}{K} T_c\right)^2}{T+T_c}
\end{equation}
the minimum in the disordered phase will be attained at $T=T_c$.
% \subsubsection{Antiferromagnetic Phase}

As previously mentioned, in the anti-ferromagnetic phase the metric is non-degenerate. Hence, the problem of finding shortest paths is 2D. We used numeric methods to find the shortest paths from a given point to different points on the phase transition line.

Calculation of the shortest path to the phase transition line from the inner point was done using the fast marching method\cite{kimmelComputingGeodesicPaths1998}.
% Some more text and graphs once I finish the calculations.
 
\paragraph{An example: the antiferromagnetic Mean-Field Ising Model}
%\omer{I wrote this assuming we will only publish this, and not the Ruppiner metric}

In the antiferromagnet Ising model in a uniform magnetic field it is impossible to traverse from the antiferromagnetic phase to the disordered phase without crossing a second-order phase transition. 
This makes it an ideal setting for our analysis. Exact solutions for the free energy in antiferromagnetic Ising models are known only for either $1d$ or mean field setting, and in order to explore the crossing of a phase transition we use the latter. With the framework discussed above, we next determine the least-dissipation path between two states 
$\left(T_1,h_1\right)\!\to\!\left(T_2,h_2\right)$,
with $\left(T_1,h_1\right)$ in the antiferromagnetic phase and $\left(T_2,h_2\right)$ in the disordered phase, for the mean-field antiferromagnetic model introduced in \cite{vivesUnifiedMeanfieldStudy1997}. 

In this model, the spins $s_i$ are split into sublattices $A$ and $B$, where a spin in sublattice $A$ interacts with all spins in sublattice $B$ with equal interaction strength. There are no interactions between spins in the same sublattice.
Let $m_A=\sum_{i\in A} s_i$ and $m_B = \sum_{i\in B}s_i$ denote the sublattice magnetizations. 
The Massieu potential of the system is given by
\begin{eqnarray}
    \psi \;=\ln Z&= -\frac{1}{2}\!\Big[&  \beta K\, m_A m_B \;-\; \beta h\, (m_A+m_B)\nonumber \\
      & &\left. \;+\; \,\big(\vartheta(m_A)+\vartheta(m_B)\big) \right]
\end{eqnarray}
where $\beta=1/T$, $K=-J$ with $J<0$ (antiferromagnetic coupling), and we define
\begin{equation}
    \vartheta(x) \;=\; \frac{1}{2}\Big[(1+x)\log(1+x)+(1-x)\log(1-x)\Big]
\end{equation}
We also set $\alpha\equiv \beta h$ for brevity.

As shown in the end matter, assuming Glauber rates for a single transition with a microscopic timescale $\Gamma$, the dissipation metric is given by 
\begin{widetext}
\begin{equation}
\resizebox{\columnwidth}{!}{$
\begin{aligned}
    g_{\mu\nu}
    \;= & \; \frac{\beta}{2\Gamma\big(1-\beta^2 K^2 \zeta_A \zeta_B\big)^{\!2}}
    \begin{pmatrix}
    \scriptstyle K^2 \!\left[\!\big(1+\beta^2 K^2 \zeta_A \zeta_B\big)\big(m_A^2 \zeta_B + m_B^2 \zeta_A\big) - 4\beta K\, m_A m_B \zeta_A \zeta_B \!\right]
    &
    \scriptstyle -K \!\left[\!\big(1+\beta^2 K^2 \zeta_A \zeta_B\big)\big(m_A \zeta_B+m_B \zeta_A\big) - 2\beta K (m_A+m_B)\zeta_A\zeta_B \!\right]
    \\[6pt]
    \scriptstyle -K \!\left[\!\big(1+\beta^2 K^2 \zeta_A \zeta_B\big)\big(m_A \zeta_B+m_B \zeta_A\big) - 2\beta K (m_A+m_B)\zeta_A\zeta_B \!\right]
    &
    \scriptstyle \big(1+\beta^2 K^2 \zeta_A \zeta_B\big)(\zeta_A+\zeta_B) - 4\beta K\,\zeta_A\zeta_B
    \end{pmatrix}\!
\end{aligned}
$}
\end{equation}
\end{widetext}
where $\zeta_i = 1-m_i^2$, and where $m_A$ and $m_B$ are the minimizer of $\psi$, satisfying the mean-field equations
\begin{align}
    m_A &= \tanh\!\left(\alpha - \beta K\, m_B\right) \label{eq:SCEmAB-A}\\
    m_B &= \tanh\!\left(\alpha - \beta K\, m_A\right) \label{eq:SCEmAB-B}
\end{align}
Let us discuss the implications of this metric on the two sides of the phase transition: the disordered and ordered phases. 




In the disordered phase, $m_A=m_B\equiv m$ (hence $\zeta_A=\zeta_B\equiv \zeta$). 
The metric in this case reduces to
\begin{equation}
    g_{\mu\nu} \;=\; \frac{\beta\,\zeta}{\Gamma\bigl(1+\beta K \zeta\bigr)^2}
    \begin{pmatrix}
        K^2 m^2 & -K m \\[2pt]
        -K m & 1
    \end{pmatrix}.
\end{equation}
The metric is rank-one for any value of $m$, with $\det g=0$. This is physically expected: in the disordered phase the system is uniquely defined by a single order parameter -- the total magnetization, therefore there is no dissipation associated with changing the control parameters $\beta$ and $h$ in a combination that does not change the total magnetization. Indeed, the null eigenvector aligns with directions of constant $m$. Geometrically, the least-dissipation trajectory is then the projection of the full geodesic flow onto this one-dimensional submanifold, collapsing the dynamics of the manifold onto its non-degenerate component. The problem of minimizing distances therefore reduces to a 1D minimization: at what distance from the phase boundary does the length minimize?  In this model, the minimum occurs at $\beta=0$ (see End Matter), namely at infinite temperature. This means that the optimal trajectory between two points in this phase follows the null trajectory from the initial point to $\beta=0$, traverse along $\beta=0$, then return along a null trajectory to the final point (see the brown trajectory in Fig.~\ref{fig:shortestPaths}). 




In the ordered phase, near the critical line, $m_A$ and $m_B$ can be expanded as shown in \cite{vivesUnifiedMeanfieldStudy1997}:
\begin{align}
    m_{A,B} &\approx \pm\sqrt{1-\frac{1}{\beta_c K}} \;\pm\; \sqrt{\Delta_s} \\
    \zeta_{A,B} &\approx \frac{1}{\beta_c K}\bigl(1 \pm 2\sqrt{\Delta_s}\bigr)
\end{align}
where 
\begin{align*}
    \Delta_m &= a_m\,(\alpha - \alpha_c) - b_m\,(\beta - \beta_c) \\
    \Delta_s &= a_s\,(\alpha - \alpha_c) + b_s\,(\beta - \beta_c)
\end{align*}
and $a_m,b_m,a_s,b_s>0$.

Substituting these expressions into the metric yields, to leading order,
\[
g_{\mu\nu}\;\propto\;\Delta_s^{-1}
\]
Hence, while the metric diverges as $\Delta_s\!\to\!0$, the associated thermodynamic length remains finite. 


We identify four distinct classes of minimal (shortest) paths. If both initial and final points are in the disordered phase, then the shortest path is to traverse along the constant magnetization towards $\beta=0$, then change the external magnetic field until the correct value of total magnetization is acheived, then traverse back along a constant total magnetization line to the final point. 


In the second case, the initial and final states lie in different phases. The optimal trajectory, which for simplicity we describe as initiating in the antiferromagnetic phase and ending in the disordered phase, starts from the initial point in one phase to the nearest point on the phase transition line, crosses into the disordered phase, and then ascends to infinite temperature ($\beta = 0$). Once at $\beta = 0$, the trajectory traverses at this value of $\beta$—where the thermodynamic length vanishes—until reaching the final magnetization value, and finally descends along the constant-magnetization line to the final point.

In the third case, both points are in the antiferromagnetic phase, and the direct geodesic connecting them is shorter than any path that detours through the disordered region.  Here, the minimal path lies entirely within the ordered manifold.

The fourth case occurs when both endpoints are in the antiferromagnetic phase, but it is nevertheless shorter to cross the phase transition back and forth to the disorder phase than to remain entirely within the ordered region across all the trajectory. This trajectory can be viewed as a composition of two  paths: the system transitions from the ordered phase to the disordered phase and then back again. 
Because paths within the disordered phase have zero thermodynamic length except for the part at $\beta=0$, the total length of such a trajectory is often dominated by the two crossings, and can be shorter than any trajectory that does not cross the phase transition.

An illustration of these three classes of paths is shown in Fig.~\ref{fig:shortestPaths}.
\begin{figure}[h]
    \centering
    \includegraphics[width=0.95\linewidth]{figures/ShortestPaths.png}
    \caption{Illustration of the three types of minimal paths in the mean-field antiferromagnet. The dashed black line is the phase transition between the disordered phase in the left and the antiferromagnetic in the right. The colored lines are example for optimal trajectories.}
    \label{fig:shortestPaths}
\end{figure}

The thermodynamic lengths within the antiferromagnetic phase were computed using the Fast Marching Method \cite{kimmelComputingGeodesicPaths1998}, following the approach of \cite{rotskoffDynamicRiemannianGeometry2015}. 
Minimal paths were obtained by performing gradient descent along the numerically computed distance function.

\paragraph{Conclusions} To conclude, we showed that although the dissipation metric diverges at a second order phase transition, for some universality classes crossing the phase transition can be achieved at a finite length. Surprisingly, we find that in some cases the optimal trajectory between two points that are in the same phase crosses to the other phase, then crosses back to the initial phase towards the end point. Many related problems, however, remained open: How can first order phase transitions be incorporate to such a framework? Are there models where the critical exponents vary continuously along the phase transition line (e.g. the 2D Ashkin-Teller model) such that crossing the phase transition cost infinite dissipation for some parts of the phase transition line, but finite for others? Lastly, many heat engines and refrigerators work intentionally near the phase transition. Using the suggested framework to optimize their protocol can be an interesting application.   

%\subsection{2D Ising model}
As an example for a system with a diverging Ruppiener metric but  nondiverging distance through the phase transition, let us consider the 2D Ising model on a triangular lattice with nearest neighbors interactions. This model has an exact solution due to Onsager, therefore it is possible to analytically calculate the thermodynamic distance for a trajectory as long as the magnetic field stays zero. This model served to numerically demonstrate trajectories with minimal thermodynamic length in \cite{rotskoffDynamicRiemannianGeometry2015}. Since we are considering  trajectories with a constant external field $h=0$, the distance between two states $T_1$ and $T_2$ is given by
\begin{equation}
    \mathcal{L} = \int_{T_1}^{T_2}\sqrt{g_{TT}}dT
\end{equation}
For the Ruppiener metric, 
\begin{eqnarray}
    g_{TT}&=& \frac{c(T)}{k_{\mathrm B}T^2}\nonumber \\
		&=& \frac{4}{T^2\pi}\big(K\coth2K\big)^2\!\left[K_1(q)-E_1(q)\right]\nonumber\\
		& &-\frac{1-\tanh^2\!2K}{T^2}\left[\frac{\pi}{2}+\frac{2}{\pi}\big(2\tanh^2\!2K-1\big)K_1(q)\right]\nonumber
\end{eqnarray}
where $K=\beta J$, $K_1(\cdot)$ and $E_1(\cdot)$ denote the complete elliptic integrals of the first and second kinds respectively, and $q(K)=\frac{2\sinh(2K)}{\cosh^2(2K)}$ \cite{}.

Near the critical point, the metric can be expand as
\begin{equation}
    g_{TT}(T) \;\sim\; \frac{1}{T_c^{2}}\;\frac{2}{\pi}\left(\frac{2J}{k_{\mathrm B}T_c}\right)^{\!2}\,
		\ln\!\left|1-\frac{T}{T_c}\right|, 
		\qquad T\to T_c^\pm.
\end{equation}
and therefore the integral converges even when $T_c$ is in between $T_1$ and $T_2$. Note that $g_{TT}$ diverges as $\ln|1-T/T_c|$, which implies that the integrals in both $\mathcal{L}$ (Eq. \ref{eq:L-def}) and $\mathcal{A}$ (Eq. \ref{SalomonDissipatedAvail}) converge. Moreover, it is a (local) minimal distance  trajectory. To show that, consider a trajectory that ```bypass'' the critical point from $(t,h)=(-\epsilon,0)$ to $(t,h) = (+\epsilon,0)$ along some trajectory $(t(s), h(s))$. The length along this trajectory is
\begin{eqnarray}
    |\dot{\boldsymbol{\lambda}}|=\sqrt{g_{tt}\dot t^2 + g_{hh}(\dot h + \frac{g_{th}}{g_{hh}}\dot t)^2-\frac{g_{th}^2}{g_{hh}}\dot t^2}
\end{eqnarray}
For $h=0$, $g_{ht}=0$ (from symmetry with respect to $h$
 around $h=0$), so at fixed $\dot t$, the minimum over $\dot h$ is at $\dot h=0$. Any deviation from $\dot h=0$ increases the integral by 
 \begin{eqnarray}
     \Delta|\dot{\boldsymbol{\lambda}}|\approx \frac{g_{hh}\dot h^2}{2\sqrt{g_{tt}}}.
 \end{eqnarray}
This implies that the trajectory along the $h=0$ is a local minimum of the distance function, even though it is not a solution of the geodesic equation due to the singularity of the metric.

%Note, however, that the integral over the Sivak metric does not converge due to the critical slow down.


\paragraph{Acknowledgments} O.R. acknowledges financial support from ISF Grant No. 232/23 and from the Minerva foundation with funding from the Federal German Ministry for Education and Research. We thank David Sivak for many useful discussions and for his hospitality.

\bibliography{ThermodynamicGeometry}
\end{document}