Since the earliest developments of equilibrium statistical mechanics to modern nonequilibrium statistical mechanics there have been attempts to cast the theory in a geometric form \cite{gibbs1873method,ruppeinerThermodynamicsRiemannianGeometric1979, weinholdMetricGeometryEquilibrium1975,crooksMeasuringThermodynamicLength2007,janyszekGeometricalStructureState1989,PhysRevE.111.034113,raz2016geometric,van2024geometric,brandner2020thermodynamic,RevModPhys.67.605, mandalAnalysisSlowTransitions2016,zhong2024beyond}. 

One of the most useful approaches is to regard the excess dissipation as a Riemannian metric. This idea was introduced already in the 80's \cite{salamonThermodynamicLengthDissipated1983,salamonRelationEntropyEnergy1984}, following purely geometrical concepts suggested already in the 70's \cite{ruppeinerThermodynamicsRiemannianGeometric1979, weinholdMetricGeometryEquilibrium1975}. The framework in  \cite{salamonThermodynamicLengthDissipated1983,salamonRelationEntropyEnergy1984} required two assumptions: (i)endoreversibility---the system remains in internal equilibrium throughout the process, but may be out of equilibrium with its environment; (ii) the dissipation is only a function of the system's mean response time. The latter assumption restricts the formalism’s applicability to systems with nearly constant relaxation time. Despite these limitations, the framework has proven valuable for identifying minimal-entropy-production protocols in processes such as chemical reactions and distillation \cite{Andresen2000,andresenCurrentTrendsFiniteTime2011}. 

The two limiting assumptions were lifted by using a closely related metric, which only assumes linear response theory to be applicable \cite{sivakThermodynamicMetricsOptimal2012, zulkowski2012geometry,rotskoffDynamicRiemannianGeometry2015,sivak2016thermodynamic,blaber2020skewed, rotskoffGeometricApproachOptimal2017}. In this formulation, the system’s relaxation dynamics are incorporated into the metric, hence removing the dependence on a constant response-time and eliminating the endoreversible assumption. This generalized framework extends thermodynamic geometry to a much broader class of systems---including those with widely varying time scales and microscopic systems such as biomolecular or single-molecule systems---and has found applications in optimizing computational and physical processes \cite{rotskoffGeometricApproachOptimal2017}.

A particularly challenging class of systems exhibiting widely separated time scales are those undergoing phase transitions. The application of thermodynamic geometry to such systems has drawn significant attention \cite{janyszekRiemannianGeometryThermodynamics1989, PhysRevE.51.1006}, including efforts to identify optimal driving protocols across critical regions (e.g., the 2D ferromagnetic Ising model \cite{rotskoffDynamicRiemannianGeometry2015}). In many cases of interest, e.g. antiferromagnets \cite{vivesUnifiedMeanfieldStudy1997}, crossing the phase transition cannot be avoided as the two phases are topologically separated by a phase transition surface. Yet, the divergence of thermodynamic length near phase transitions has received limited study. Notably, a divergence in the metric tensor does not necessarily imply a divergence in thermodynamic length. However, even if the length remains finite, other geometric quantities such as the curvature may diverge, signaling the critical behavior. 

In this paper, we analyze the behavior of the thermodynamic metric near second-order phase transitions. Using the Widom scaling and the dynamical scaling hypothesis, we show that in some class of models, the divergence of the metric at the phase transition does not imply divergence of dissipation when crossing the phase transition. Therefore, the thermodynamic distance between states in the different phases is finite. The class of models with this property includes several mean field models, as well as the Ising model in dimensions 3 and above. To demonstrate our findings, we apply the formalism to the mean field antiferromagnetic Ising model, where  we determine the optimal protocols connecting two points. Surprisingly, we find that in some cases the optimal protocol between two points that are in the same phase nevertheless crosses to the other phase and back.