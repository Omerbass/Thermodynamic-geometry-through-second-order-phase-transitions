
\subsubsection{Disordered phase} \label{subsec:SivakDisorderedPhase}
It is convenient to define the total and staggered magnetizations
\begin{align}
    m &= \frac{m_A+m_B}{2}\\
    s &= \frac{m_A-m_B}{2}
\end{align}

For each $m\in[-1,1]$, the corresponding critical parameters $(\beta_c,\alpha_c)$ on the phase boundary satisfy
\begin{align}
    m &= \pm\sqrt{1-\frac{1}{\beta_c K}} \label{m(beta)}\\
    \alpha_c &= \pm\Big(\beta_c K\, m + \arctanh m\Big) \label{alpha(beta,m)}
\end{align}

In the disordered phase, $m_A=m_B\equiv m$ (hence $\zeta_A=\zeta_B\equiv \zeta$). 
The metric in this case reduces to
\begin{equation} \label{eq:disorderedmetric}
    g_{\mu\nu} \;=\; \frac{\beta\,\zeta}{\Gamma\bigl(1+\beta K \zeta\bigr)^2}
    \begin{pmatrix}
        K^2 m^2 & -K m \\[2pt]
        -K m & 1
    \end{pmatrix}.
\end{equation}
As in the ruppiner case, the metric is rank-one, with the null eigenvector aligning with directions of constant $m$.
The problem of minimizing distances again reduces to a 1D minimization: at what distance from the phase boundary does the length become minimal? 
This will be calculated in the following paragraphs:

For each $m\in[-1,1]$, the corresponding critical parameters $(\beta_c,\alpha_c)$ on the phase boundary satisfy
\begin{align}
    m &= \pm\sqrt{1-\frac{1}{\beta_c K}} \label{m(beta)}\\
    \alpha_c &= \pm\Big(\beta_c K\, m + \arctanh m\Big) \label{alpha(beta,m)}
\end{align}
The eigenpairs of the metric \ref{eq:disorderedmetric} are
\begin{align}
    \lambda_1 &= 0 \quad :
    & \boldsymbol{v}_1 &= \begin{pmatrix} 1 \\ K m \end{pmatrix}
    \\
    \lambda_2 &= 1+K^2 m^2 \quad:
    & \boldsymbol{v}_2 &= \begin{pmatrix} K m \\ -1 \end{pmatrix}
\end{align}
Thus the zero mode $\boldsymbol{v}_1$ follows lines of constant $m$ in $(\beta,\alpha)$.

\paragraph{Scaling of the metric prefactor.}
From \eqref{m(beta)} we have $\zeta=1-m^2=1/(\beta_c K)$ along the phase boundary labeled by magnetization $m$, with critical parameter $\beta_c$.
Using this in the prefactor gives
\begin{equation}
    g_{\mu\nu} \;\propto\; 
    \frac{\beta}{\bigl(\beta_c+\beta\bigr)^2}\,\bigl(1+K^2 m^2\bigr).
\end{equation}

\paragraph{Scaling of the displacement along the zero mode.}
Consider two neighboring constant-$m$ lines labeled by 
$(\beta_{c1},\alpha_{c1})$ and $(\beta_{c2},\alpha_{c2})=(\beta_{c1}+\delta_\beta,\alpha_{c1}+\delta_\alpha)$.
We find their intersection with a straight line parallel to the zero mode at each point:
\begin{align}
    \begin{pmatrix}\beta\\ \alpha\end{pmatrix}
    &= a\,\boldsymbol{v}_1(\beta_{c1}) + \begin{pmatrix}\beta_{c1}\\ \alpha_{c1}\end{pmatrix}
    \label{eq:b0a0-1}\\[2pt]
    \begin{pmatrix}\beta\\ \alpha\end{pmatrix}
    &= b\,\boldsymbol{v}_1(\beta_{c2}) + \begin{pmatrix}\beta_{c1}+\delta_\beta\\ \alpha_{c1} + \dfrac{\partial \alpha_c}{\partial \beta_c}\,\delta_\beta \end{pmatrix}
    \label{eq:b0a0-2}
\end{align}
Subtracting \eqref{eq:b0a0-1} from \eqref{eq:b0a0-2} yields
\begin{equation}
    \delta_\beta 
    \begin{pmatrix} 1 \\[2pt] \dfrac{\partial \alpha_c}{\partial \beta_c} \end{pmatrix}
    =
    \begin{pmatrix}
        1 & 1 \\
        K m_2 & K m_1
    \end{pmatrix}
    \begin{pmatrix} -b \\[2pt] a \end{pmatrix}
\end{equation}
where $m_1=m(\beta_{c1})$, $m_2=m(\beta_{c2})$.
From \eqref{alpha(beta,m)} we have 
\(
\dfrac{\partial \alpha_c}{\partial \beta_c} = \dfrac{1}{m}
\)
Inverting gives
\begin{equation}
\begin{aligned}
    a \;&=\; \frac{1-m^2}{K m}\,\frac{\partial m}{\partial \beta_c}
      \;=\; \frac{1-m^2}{K m}\,\frac{2m}{(1-m^2)^2}\\
      \;&=\; \frac{2}{K(1-m^2)}
      \;=\; 2\beta_c
\end{aligned}
\end{equation}
where we used $1-m^2=1/(\beta_c K)$ and $\dfrac{\partial m}{\partial \beta_c} = \dfrac{2m}{(1-m^2)^2}$ from \eqref{m(beta)}.
Hence the displacement along the zero mode scales as 
\(
\delta\boldsymbol{\lambda}\sim (\beta+\beta_c).
\)

\paragraph{Conclusion: minimum at $\beta=0$.}
Combining the prefactor and displacement scalings,
\[
\sqrt{\delta\lambda^\mu\, g_{\mu\nu}\, \delta\lambda^\nu}
\;\propto\;
\sqrt{\beta}
\]
so the infinitesimal thermodynamic length is minimized at $\beta=0$ (infinite temperature).
Therefore, the optimal trajectory between two points in this phase follows the null direction (of constant magnetization) from the initial point to $\beta=0$, then traverse along $\beta=0$, then return along a null trajectory to the final point. 

This is opposite to our result for the Ruppiner metric, where the optimal trajectory is along the phase transition line. Meaning, taking into account the divergence in response time substantially changes the optimal trajectory in this model.