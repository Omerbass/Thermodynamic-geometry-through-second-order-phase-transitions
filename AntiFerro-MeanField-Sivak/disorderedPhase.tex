
\subsubsection{Disordered phase} \label{subsec:SivakDisorderedPhase}

In the disordered phase we have $m_A=m_B\equiv m$ (hence $\zeta_A=\zeta_B\equiv \zeta$). 
The Sivak metric in $(\beta,\alpha)$, with $\alpha=\beta h$, reduces to
\begin{equation}
    g_{\mu\nu} \;=\; \frac{\beta\,\zeta}{\Gamma\bigl(1+\beta K \zeta\bigr)^2}
    \begin{pmatrix}
        K^2 m^2 & -K m \\[2pt]
        -K m & 1
    \end{pmatrix}.
\end{equation}
Since $\det P=0$ in this phase (rows of $P$ become proportional when $m_A=m_B$), the metric is rank-one and $\det g=0$. 
The null eigenvector aligns with directions of constant $m$, so motion along that direction incurs zero thermodynamic length. 
Therefore the problem reduces to a 1D minimization: at what distance from the phase boundary does the length minimize? 
We show below that the minimum occurs at $\beta=0$ (infinite temperature).

\paragraph{Eigenstructure and the zero-length direction.}
The eigenpairs of the bracketed matrix are
\begin{align}
    \lambda_1 &= 0, 
    & \boldsymbol{v}_1 &= \begin{pmatrix} 1 \\ K m \end{pmatrix},
    \\
    \lambda_2 &= 1+K^2 m^2, 
    & \boldsymbol{v}_2 &= \begin{pmatrix} K m \\ -1 \end{pmatrix}.
\end{align}
Thus the zero mode $\boldsymbol{v}_1$ follows lines of constant $m$ in $(\beta,\alpha)$.

\paragraph{Scaling of the metric prefactor.}
From \eqref{m(beta)} we have $\zeta=1-m^2=1/(\beta_c K)$ along the phase boundary labeled by magnetization $m$, with critical parameter $\beta_c$.
Using this in the prefactor gives
\begin{equation}
    g_{\mu\nu} \;\propto\; 
    \frac{\beta}{\bigl(\beta_c+\beta\bigr)^2}\,\bigl(1+K^2 m^2\bigr).
\end{equation}

\paragraph{Scaling of the displacement along the zero mode.}
Consider two neighboring constant-$m$ lines labeled by 
$(\beta_{c1},\alpha_{c1})$ and $(\beta_{c2},\alpha_{c2})=(\beta_{c1}+\delta_\beta,\alpha_{c1}+\delta_\alpha)$.
We find their intersection with a straight line parallel to the zero mode at each point:
\begin{align}
    \begin{pmatrix}\beta\\ \alpha\end{pmatrix}
    &= a\,\boldsymbol{v}_1(\beta_{c1}) + \begin{pmatrix}\beta_{c1}\\ \alpha_{c1}\end{pmatrix}
    \label{eq:b0a0-1}\\[2pt]
    \begin{pmatrix}\beta\\ \alpha\end{pmatrix}
    &= b\,\boldsymbol{v}_1(\beta_{c2}) + \begin{pmatrix}\beta_{c1}+\delta_\beta\\ \alpha_{c1} + \dfrac{\partial \alpha_c}{\partial \beta_c}\,\delta_\beta \end{pmatrix}
    \label{eq:b0a0-2}
\end{align}
Subtracting \eqref{eq:b0a0-1} from \eqref{eq:b0a0-2} yields
\begin{equation}
    \delta_\beta 
    \begin{pmatrix} 1 \\[2pt] \dfrac{\partial \alpha_c}{\partial \beta_c} \end{pmatrix}
    =
    \begin{pmatrix}
        1 & 1 \\
        K m_2 & K m_1
    \end{pmatrix}
    \begin{pmatrix} -b \\[2pt] a \end{pmatrix}
\end{equation}
where $m_1=m(\beta_{c1})$, $m_2=m(\beta_{c2})$.
From \eqref{alpha(beta,m)} we have 
\(
\dfrac{\partial \alpha_c}{\partial \beta_c} = \dfrac{1}{m}
\)
Inverting gives
\begin{equation}
\begin{aligned}
    a \;&=\; \frac{1-m^2}{K m}\,\frac{\partial m}{\partial \beta_c}
      \;=\; \frac{1-m^2}{K m}\,\frac{2m}{(1-m^2)^2}\\
      \;&=\; \frac{2}{K(1-m^2)}
      \;=\; 2\beta_c
\end{aligned}
\end{equation}
where we used $1-m^2=1/(\beta_c K)$ and $\dfrac{\partial m}{\partial \beta_c} = \dfrac{2m}{(1-m^2)^2}$ from \eqref{m(beta)}.
Hence the displacement along the zero mode scales as 
\(
\delta\boldsymbol{\lambda}\sim (\beta+\beta_c).
\)

\paragraph{Conclusion: minimum at $\beta=0$.}
Combining the prefactor and displacement scalings,
\[
\sqrt{\delta\lambda^\mu\, g_{\mu\nu}\, \delta\lambda^\nu}
\;\propto\;
\sqrt{\beta}
\]
so the infinitesimal thermodynamic length is minimized at $\beta=0$ (infinite temperature).
Thus, within the disordered phase, the least-dissipation path approaches the high-temperature limit.

\paragraph{Geometric interpretation.}
In this regime the metric degenerates, leaving a single nonzero eigenvalue that defines a unique geodesic submanifold. 
The null direction $\boldsymbol{v}_1$ corresponds to constant-magnetization curves, which carry zero thermodynamic cost. 
Hence the effective optimization reduces to minimizing the scalar prefactor of the metric in $\beta$, rather than the full Riemannian length. 
Geometrically, the least-dissipation trajectory is the projection of the full geodesic flow onto this one-dimensional submanifold, collapsing the dynamics of the manifold onto its non-degenerate component.


% From eqs. \ref{m(beta)} and \ref{alpha(beta,m)}
% \begin{equation}
%    K d\beta_c = \frac{2m}{\left(1-m^2\right)^2}dm
% \end{equation}
% \begin{equation}
%     d\alpha_c = \frac{2}{\left(1-m^2\right)^2}dm
% \end{equation}

% and the length differential
% \begin{equation} \label{eq:LengthAlongPTline_disorderedPhase}
% \begin{split}
%     d\ell & = \sqrt{d\lambda^\mu g_{\mu\nu} d\lambda^\nu} \\
%           & =  \frac{\left|dm\right|}{ \sqrt{ 2 \Gamma K  } \, \zeta } \\
%           & =  \frac{\left|d(\arctanh \, m)\right|}{ \sqrt{ 2 \Gamma K  } } \\
% \end{split}
% \end{equation}