\subsubsection{Disordered phase} \label{subsec:SivakDisorderedPhase}
In the disordered phase one gets that $m_A=m_B\equiv m$ (and therefore also $\zeta_A=\zeta_B\equiv \zeta$). One gets
\begin{equation}
    g_{\mu\nu} = \frac{\beta\zeta}{\Gamma \left(1+\beta K \zeta\right)^2} \left(
        \begin{array}{cc}
            K^2 m^2 & - Km \\
            - Km & 1
        \end{array}
    \right)
\end{equation}
In particular (and as a direct result of $\det P=0$) $\det g=0$. 
The 0 eigenvalue corresponds to lines of constant $m$. As a result, the problem becomes a 1D minimization problem - at what distance from the phase transition line will the length be minimal? As we show below, the answer is simply at $\beta=0$.
Firstly, the eigenvectors and eigenvalues are
\begin{align}
    \lambda_1 & = 0,\quad \boldsymbol{v}_1 = \left( \begin{array}{c}
         1  \\
        Km
    \end{array} \right) \\
    \lambda_2 & = K^2 m^2+1,\quad \boldsymbol{v}_2 = \left( \begin{array}{c}
        Km \\
         -1 
    \end{array} \right) 
\end{align}
Our objective is to minimize the expression 
$ \sqrt{{\delta\lambda}^\mu g_{\mu\nu} {\delta\lambda}^\nu}$
along the $v_1$ axis. 
Noting that from \ref{m(beta)}, $\zeta = \frac{1}{K\beta_c}$, and taking into account the entire expression for $g$, and, we get that 
\begin{equation}
    g_{\mu\nu} \propto \frac{\beta}{\left(\beta_c+\beta\right)^2}\left(K^2 m^2+1\right)
\end{equation}
Now we want to find the scaling of $\delta\boldsymbol{\lambda}$. We'll take an infinitesimal step, such that $m_1$ goes to a nearby $m_2$: $(\beta_{c1}, \alpha_{c1})$ goes to $(\beta_{c2}, \alpha_{c2}) = (\beta_{c1}+\delta_\beta, \alpha_{c1}+\delta_\alpha)$  (and $\delta_\beta,\delta_\alpha$ are both small)
We want the intersection of the lines
\begin{align} \label{eq:b0,a0}
    \left( \begin{array}{c}
         \beta \\
         \alpha
    \end{array}\right) & = a \, \boldsymbol{v}_1\left(T_c\right) + \left( \begin{array}{c}
         \beta_{c1} \\
         \alpha_{c1}
    \end{array}\right) \\
    \left( \begin{array}{c}
         \beta \\
         \alpha
    \end{array}\right) & = b \, \boldsymbol{v}_1\left(T_c + \delta_T \right) + \left( \begin{array}{c}
         \beta_{c1} + \delta_T \\
         \alpha_{c1} + \frac{\partial \alpha_c}{\partial \beta_c}\delta_\beta
    \end{array}\right)
\end{align}
Subtracting the equations we get
\begin{equation}
    \delta_\beta \left( \begin{array}{c}
         1 \\
         \frac{\partial \alpha_c}{\partial \beta_c}
    \end{array}\right) = \left( \begin{array}{cc}
        1 & 1 \\
        Km_2 & Km_1
    \end{array} \right)
    \left( \begin{array}{c}
         -b  \\
         a
    \end{array}\right)
\end{equation}
From \ref{alpha(beta,m)}, we get
\begin{equation}
    \frac{\partial \alpha_c}{\partial \beta_c} = \frac{1}{m}
\end{equation}
And after inverting the matrix 
\begin{equation}
\begin{split}
    a & = \frac{1-m^2}{Km} \frac{\partial m}{\partial \beta_c} \\
      & = \frac{1-m^2}{Km} \frac{2m}{\left(1-m^2\right)^2} \\
      & = \frac{2}{K \left(1-m^2\right) } \\
      & = 2 \beta_c
\end{split}
\end{equation}
Meaning that $\delta\boldsymbol{\lambda}$ scales as $\beta+\beta_c$. Putting this all together, $ \sqrt{{\delta\lambda}^\mu g_{\mu\nu} {\delta\lambda}^\nu} \propto \sqrt{\beta}$, meaning it will be minimized at infinite temperature - when $\beta=0$.

% From eqs. \ref{m(beta)} and \ref{alpha(beta,m)}
% \begin{equation}
%    K d\beta_c = \frac{2m}{\left(1-m^2\right)^2}dm
% \end{equation}
% \begin{equation}
%     d\alpha_c = \frac{2}{\left(1-m^2\right)^2}dm
% \end{equation}

% and the length differential
% \begin{equation} \label{eq:LengthAlongPTline_disorderedPhase}
% \begin{split}
%     d\ell & = \sqrt{d\lambda^\mu g_{\mu\nu} d\lambda^\nu} \\
%           & =  \frac{\left|dm\right|}{ \sqrt{ 2 \Gamma K  } \, \zeta } \\
%           & =  \frac{\left|d(\arctanh \, m)\right|}{ \sqrt{ 2 \Gamma K  } } \\
% \end{split}
% \end{equation}