\subsubsection{Minimal Paths}

We identify three distinct classes of minimal (shortest) paths, two of which cross the phase transition line.

\paragraph*{Type A: Crossing between phases.}
In the first case, the initial and final states lie in different phases. 
The optimal trajectory proceeds from the initial point in the antiferromagnetic phase to the nearest point on the phase transition line, crosses into the disordered phase, and then ascends to infinite temperature ($\beta = 0$). 
Once at $\beta = 0$, the system moves horizontally along this line—where the thermodynamic length vanishes—until reaching the final magnetization, and finally descends along the constant-magnetization line to the destination point.

\paragraph*{Type B: Crossing and returning.}
The second case occurs when both endpoints are in the antiferromagnetic phase, but it is shorter to cross the phase transition than to remain entirely within the ordered region. 
This trajectory can be viewed as a composition of two Type~I paths: the system transitions from the ordered phase to the disordered phase and then back again. 
Because, as shown in Sec.~\ref{subsec:SivakDisorderedPhase}, paths within the disordered phase have zero thermodynamic length, the total length of such a trajectory is dominated by the two crossings.

\paragraph*{Type C: Entirely within one phase.}
In the third case, both points are again in the same phase. Either in the disordered phase or in the antiferromagnetic phase. In the disordered case, the path will be composed of moving along the constant magnetization line to infinite temperature, along infinite temperature to the new magnetization and back along the constant magnetization line.
In the antiferromagnetic case, when the direct geodesic connecting the two points is shorter than any path that detours through the disordered region, the minimal path lies entirely within the ordered phase. 

An illustration of these three classes of paths is shown in Fig.~\ref{fig:shortestPaths}.
\begin{figure}[h]
    \centering
    \includegraphics[width=0.9\linewidth]{figures/ShortestPaths_withAlpha.png}
    \caption{Illustration of the three types of minimal paths in the mean-field antiferromagnet using the Sivak metric.}
    \label{fig:shortestPaths}
\end{figure}

The thermodynamic lengths within the antiferromagnetic phase were computed using the Fast Marching Method \cite{kimmelComputingGeodesicPaths1998}, following the approach of Rotskoff in \cite{rotskoffDynamicRiemannianGeometry2015}. 
Minimal paths were obtained by performing gradient descent along the numerically computed distance function. The code is available on github at the link: \href{https://github.com/Omerbass/FMM-Geodesics}{https://github.com/Omerbass/FMM-Geodesics}.