

In the disordered phase, $m_A=m_B\equiv m$ (hence $\zeta_A=\zeta_B\equiv \zeta$). 
The metric in this case reduces to
\begin{equation}
    g_{\mu\nu} \;=\; \frac{\beta\,\zeta}{\Gamma\bigl(1+\beta K \zeta\bigr)^2}
    \begin{pmatrix}
        K^2 m^2 & -K m \\[2pt]
        -K m & 1
    \end{pmatrix}.
\end{equation}
The metric is rank-one for any value of $m$, with $\det g=0$. This is physically expected: in the disordered phase the system is uniquely defined by a single order parameter -- the total magnetization, therefore there is no dissipation associated with changing the control parameters $\beta$ and $h$ in a combination that does not change the total magnetization. Indeed, the null eigenvector aligns with directions of constant $m$. Geometrically, the least-dissipation trajectory is then the projection of the full geodesic flow onto this one-dimensional submanifold, collapsing the dynamics of the manifold onto its non-degenerate component. The problem of minimizing distances therefore reduces to a 1D minimization: at what distance from the phase boundary does the length minimize?  In this model, the minimum occurs at $\beta=0$ (see End Matter), namely at infinite temperature. This means that the optimal trajectory between two points in this phase follows the null trajectory from the initial point to $\beta=0$, traverse along $\beta=0$, then return along a null trajectory to the final point (see the brown trajectory in Fig.~\ref{fig:shortestPaths}). 


