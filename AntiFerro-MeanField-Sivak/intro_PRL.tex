In the antiferromagnet Ising model in a uniform magnetic field it is impossible to traverse from the antiferromagnetic phase to the disordered phase without crossing a second-order phase transition. 
This makes it an ideal setting for our analysis. Exact solutions for the free energy in antiferromagnetic Ising models are known only for either $1d$ or mean field setting, and in order to explore the crossing of a phase transition we use the latter. With the framework discussed above, we next determine the least-dissipation path between two states 
$\left(T_1,h_1\right)\!\to\!\left(T_2,h_2\right)$,
with $\left(T_1,h_1\right)$ in the antiferromagnetic phase and $\left(T_2,h_2\right)$ in the disordered phase, for the mean-field antiferromagnetic model introduced in \cite{vivesUnifiedMeanfieldStudy1997}. 

In this model, the spins $s_i$ are split into sublattices $A$ and $B$, where a spin in sublattice $A$ interacts with all spins in sublattice $B$ with equal interaction strength. There are no interactions between spins in the same sublattice.
Let $m_A=\sum_{i\in A} s_i$ and $m_B = \sum_{i\in B}s_i$ denote the sublattice magnetizations. 
The Massieu potential of the system is given by
\begin{eqnarray}
    \psi \;=\ln Z&= -\frac{1}{2}\!\Big[&  \beta K\, m_A m_B \;-\; \beta h\, (m_A+m_B)\nonumber \\
      & &\left. \;+\; \,\big(\vartheta(m_A)+\vartheta(m_B)\big) \right]
\end{eqnarray}
where $\beta=1/T$, $K=-J$ with $J<0$ (antiferromagnetic coupling), and we define
\begin{equation}
    \vartheta(x) \;=\; \frac{1}{2}\Big[(1+x)\log(1+x)+(1-x)\log(1-x)\Big]
\end{equation}
We also set $\alpha\equiv \beta h$ for brevity.

As shown in the end matter, assuming Glauber rates for a single transition with a microscopic timescale $\Gamma$, the dissipation metric is given by 
\begin{widetext}
\begin{equation}
\resizebox{\columnwidth}{!}{$
\begin{aligned}
    g_{\mu\nu}
    \;= & \; \frac{\beta}{2\Gamma\big(1-\beta^2 K^2 \zeta_A \zeta_B\big)^{\!2}}
    \begin{pmatrix}
    \scriptstyle K^2 \!\left[\!\big(1+\beta^2 K^2 \zeta_A \zeta_B\big)\big(m_A^2 \zeta_B + m_B^2 \zeta_A\big) - 4\beta K\, m_A m_B \zeta_A \zeta_B \!\right]
    &
    \scriptstyle -K \!\left[\!\big(1+\beta^2 K^2 \zeta_A \zeta_B\big)\big(m_A \zeta_B+m_B \zeta_A\big) - 2\beta K (m_A+m_B)\zeta_A\zeta_B \!\right]
    \\[6pt]
    \scriptstyle -K \!\left[\!\big(1+\beta^2 K^2 \zeta_A \zeta_B\big)\big(m_A \zeta_B+m_B \zeta_A\big) - 2\beta K (m_A+m_B)\zeta_A\zeta_B \!\right]
    &
    \scriptstyle \big(1+\beta^2 K^2 \zeta_A \zeta_B\big)(\zeta_A+\zeta_B) - 4\beta K\,\zeta_A\zeta_B
    \end{pmatrix}\!
\end{aligned}
$}
\end{equation}
\end{widetext}
where $\zeta_i = 1-m_i^2$, and where $m_A$ and $m_B$ are the minimizer of $\psi$, satisfying the mean-field equations
\begin{align}
    m_A &= \tanh\!\left(\alpha - \beta K\, m_B\right) \label{eq:SCEmAB-A}\\
    m_B &= \tanh\!\left(\alpha - \beta K\, m_A\right) \label{eq:SCEmAB-B}
\end{align}
Let us discuss the implications of this metric on the two sides of the phase transition: the disordered and ordered phases. 

