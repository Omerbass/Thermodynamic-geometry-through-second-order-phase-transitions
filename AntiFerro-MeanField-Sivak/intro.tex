
We now determine the least-dissipation path between two states 
$\left(T_1,h_1\right)\!\to\!\left(T_2,h_2\right)$,
with $\left(T_1,h_1\right)$ in the antiferromagnetic phase and $\left(T_2,h_2\right)$ in the disordered phase. 
In particular, we compute the Sivak metric for the MF antiferromagnetic model of \cite{vivesUnifiedMeanfieldStudy1997}.

\subsubsection{The Model}
The Massieu potential is
\begin{widetext}
\begin{equation}
    \psi \;=\ln Z=\; -\frac{1}{2}\!\left[ \beta K\, m_A m_B \;-\; \beta h\, (m_A+m_B)
    \;+\; \frac{1}{2}\,\big(\vartheta(m_A)+\vartheta(m_B)\big) \right]
\end{equation}
\end{widetext}
where $\beta=1/T$.

\subsubsection{Calculation of the Metric}
The computation proceeds in three steps:
\begin{enumerate}
    \item \textbf{Equilibrium covariances:} evaluate 
    $\langle \delta m_i(0)\,\delta m_j(0)\rangle$ ($i,j\in\{A,B\}$).
    \item \textbf{Dynamics and time integral:} incorporate linearized relaxation to obtain the integrated correlation.
    \item \textbf{Variable transform:} convert to the $\boldsymbol{\lambda}=(\beta,\beta h)\equiv(\beta,\alpha)$ representation, which is the representation of our control parameters, via the Jacobian from $(m_A,m_B)$ to $(-E,m)$.
\end{enumerate}
Note we set $\alpha \equiv \beta h$ for brevity.

Let $\zeta_i\equiv 1-m_i^2$ for $i\in\{A,B\}$. 
The inverse covariance is the Hessian of $\psi$ with respect to $(m_A,m_B)$:
\begin{equation}
    -\Sigma^{-1} \;=\; \frac{1}{2}\!
    \begin{pmatrix}
        \dfrac{1}{\zeta_A} & \beta K \\
        \beta K & \dfrac{1}{\zeta_B}
    \end{pmatrix}.
\end{equation}
Its inverse is
\begin{equation}
    \Sigma \;=\; \frac{2}{\,1-\beta^2 K^2 \zeta_A \zeta_B\,}
    \begin{pmatrix}
        \zeta_A & -\beta K\,\zeta_A\zeta_B \\
        -\beta K\,\zeta_A\zeta_B & \zeta_B
    \end{pmatrix}
\end{equation}

The relaxation dynamics \cite{klichMpembaIndexAnomalous2019} are
\begin{align}
    \dot m_A &= \Gamma\!\left[\tanh\!\left(\alpha - \beta K\, m_B\right) - m_A\right]\\
    \dot m_B &= \Gamma\!\left[\tanh\!\left(\alpha - \beta K\, m_A\right) - m_B\right]
\end{align}
with microscopic attempt rate $\Gamma$. 
Linearizing about equilibrium gives
\begin{align}
    \delta \dot m_A &= \Gamma\left(-\,\delta m_A \;-\; \beta K\,\zeta_A\,\delta m_B\right)\\
    \delta \dot m_B &= \Gamma\left(-\,\beta K\,\zeta_B\,\delta m_A \;-\; \delta m_B\right)
\end{align}
In vector form, with $y=(\delta m_A,\delta m_B)^{\!\top}$
\begin{equation}
    \dot y \;=\; -\Gamma 
    \begin{pmatrix}
        1 & \beta K\,\zeta_A \\
        \beta K\,\zeta_B & 1
    \end{pmatrix} y \;=\; -M y
\end{equation}

We require
\begin{equation}
    \beta \int_{0}^{\infty}\!\langle \delta m_i(0)\,\delta m_j(t)\rangle\,dt
    \;=\; \beta \int_0^\infty \!\langle y(0)\,y^{\!\top}(t)\rangle\,dt
\end{equation}
Since $\langle \delta m_i(0)\,\eta(t)\rangle=0$ for $t>0$, substituting $y(t)=e^{-Mt}y(0)$ yields
\begin{widetext}
\begin{equation}
\begin{aligned}
    \beta \int_0^\infty \!\langle y(0)\,y^{\!\top}(t)\rangle\,dt
    &= \beta\,\langle y(0)\,y^{\!\top}(0)\rangle \int_0^\infty \!e^{-M^{\!\top}t}\,dt
     \;=\; \beta\,\Sigma\,(M^{\!\top})^{-1} \\[3pt]
    &= \frac{2\beta}{\Gamma\big(1-\beta^2 K^2 \zeta_A \zeta_B\big)^{\!2}}
    \begin{pmatrix}
        \zeta_A\big(1+\beta^2 K^2 \zeta_A \zeta_B\big) & -\,2\beta K\,\zeta_A\zeta_B\\
        -\,2\beta K\,\zeta_A\zeta_B & \zeta_B\big(1+\beta^2 K^2 \zeta_A \zeta_B\big)
    \end{pmatrix}\!
\end{aligned}
\end{equation}
\end{widetext}

To express the Sivak metric in the control variables $\boldsymbol{\lambda}=(\beta,\alpha)$, we transform fluctuations to $(-E,m)$ via
\begin{align}
    E &= \tfrac{1}{2} K\, m_A m_B\\
    m &= \tfrac{1}{2}(m_A+m_B)
\end{align}
so the Jacobian is
\begin{equation}
    P \;=\; \frac{\partial(-E,m)}{\partial(m_A,m_B)} 
    \;=\; \frac{1}{2}
    \begin{pmatrix}
        -K m_B & -K m_A \\
        \;1 & \;1
    \end{pmatrix}
\end{equation}
The Sivak metric then reads
\begin{widetext}
\begin{equation}
\resizebox{\columnwidth}{!}{$
\begin{aligned}
    g^{\text{Sivak}}_{\mu\nu}
    \;= & \; \beta\, P\,\Sigma\,(M^{\!\top})^{-1} P^{\!\top} \\
    \;= & \; \frac{\beta}{2\Gamma\big(1-\beta^2 K^2 \zeta_A \zeta_B\big)^{\!2}}
    \begin{pmatrix}
    \scriptstyle K^2 \!\left[\!\big(1+\beta^2 K^2 \zeta_A \zeta_B\big)\big(m_A^2 \zeta_B + m_B^2 \zeta_A\big) - 4\beta K\, m_A m_B \zeta_A \zeta_B \!\right]
    &
    \scriptstyle -K \!\left[\!\big(1+\beta^2 K^2 \zeta_A \zeta_B\big)\big(m_A \zeta_B+m_B \zeta_A\big) - 2\beta K (m_A+m_B)\zeta_A\zeta_B \!\right]
    \\[6pt]
    \scriptstyle -K \!\left[\!\big(1+\beta^2 K^2 \zeta_A \zeta_B\big)\big(m_A \zeta_B+m_B \zeta_A\big) - 2\beta K (m_A+m_B)\zeta_A\zeta_B \!\right]
    &
    \scriptstyle \big(1+\beta^2 K^2 \zeta_A \zeta_B\big)(\zeta_A+\zeta_B) - 4\beta K\,\zeta_A\zeta_B
    \end{pmatrix}\!
\end{aligned}
$}
\end{equation}
\end{widetext}

\begin{figure}[h]
    \centering
    \begin{subfigure}[b]{0.45\textwidth}
        \centering
        \includegraphics[width=0.9\textwidth]{figures/sivakmetric_bb.png}
        \caption{Caloric element $g_{\beta\beta}$}
    \end{subfigure}%\hfill
    
    \begin{subfigure}[b]{0.45\textwidth}
        \centering
        \includegraphics[width=0.9\textwidth]{figures/sivakmetric_ab.png}
        \caption{Magneto–caloric element $g_{\alpha\beta}$}
    \end{subfigure}
    % \vspace{0.75ex}
    \begin{subfigure}[b]{0.45\textwidth}
        \centering
        \includegraphics[width=0.9\textwidth]{figures/sivakmetric_aa.png}
        \caption{Magnetic element $g_{\alpha\alpha}$}
    \end{subfigure}
    \caption{Elements of the Sivak metric for the MF antiferromagnet}
    \label{fig:sivakmetric_antiferro}
\end{figure}

Figures~\ref{fig:sivakmetric_antiferro} show the resulting metric components.
