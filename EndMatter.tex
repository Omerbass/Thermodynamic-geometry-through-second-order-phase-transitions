
\section{Calculating the dissipation metrix}
\begin{comment}
It is convenient to define the total and staggered magnetizations
\begin{align}
    m &= \frac{m_A+m_B}{2}\\
    s &= \frac{m_A-m_B}{2}
\end{align}

For each $m\in[-1,1]$, the corresponding critical parameters $(\beta_c,\alpha_c)$ on the phase boundary satisfy
\begin{align}
    m &= \pm\sqrt{1-\frac{1}{\beta_c K}} \label{m(beta)}\\
    \alpha_c &= \pm\Big(\beta_c K\, m + \arctanh m\Big) \label{alpha(beta,m)}
\end{align}
\end{comment}

\subsubsection{Calculation of the Metric}
The computation is done as follows: First, we calculate the equilibrium covariances, namely we evaluate 
    $\langle \delta m_i(0)\,\delta m_j(0)\rangle$ ($i,j\in\{A,B\}$).
    From these, we calculate the time integrated correlations by incorporate linearized relaxation. As these calculations are easier in $m_i$ variables, we then convert to the $\boldsymbol{\lambda}=(\beta,\beta h) \equiv (\beta, \alpha)$ representation, in which the metric is provided.

Let $\zeta_i\equiv 1-m_i^2$ for $i\in\{A,B\}$. 
The inverse covariance is the Hessian of the function $\psi$ with respect to the coordinates $(m_A,m_B)$. Using the expression for $\psi$, it is given by
\begin{equation}
    -\Sigma^{-1} \;=\; \frac{1}{2}\!
    \begin{pmatrix}
        \dfrac{1}{\zeta_A} & \beta K \\
        \beta K & \dfrac{1}{\zeta_B}
    \end{pmatrix}.
\end{equation}
Its inverse is given by
\begin{equation}
    \Sigma \;=\; \frac{2}{\,1-\beta^2 K^2 \zeta_A \zeta_B\,}
    \begin{pmatrix}
        \zeta_A & -\beta K\,\zeta_A\zeta_B \\
        -\beta K\,\zeta_A\zeta_B & \zeta_B
    \end{pmatrix}
\end{equation}

The relaxation dynamics we use \cite{klichMpembaIndexAnomalous2019}, corresponds to single spin Glauber dynamics, and are given by
\begin{align}
    \dot m_A &= \Gamma\!\left[\tanh\!\left(\alpha - \beta K\, m_B\right) - m_A\right]\\
    \dot m_B &= \Gamma\!\left[\tanh\!\left(\alpha - \beta K\, m_A\right) - m_B\right]
\end{align}
with microscopic attempt rate $\Gamma$. 
Linearizing about equilibrium gives
\begin{align}
    \delta \dot m_A &= \Gamma\left(-\,\delta m_A \;-\; \beta K\,\zeta_A\,\delta m_B\right)\\
    \delta \dot m_B &= \Gamma\left(-\,\beta K\,\zeta_B\,\delta m_A \;-\; \delta m_B\right)
\end{align}
In vector form, 
\begin{eqnarray}
    \begin{pmatrix}
        \delta \dot m_A\\
        \delta \dot m_B
    \end{pmatrix} &=& -\Gamma 
    \begin{pmatrix}
        1 & \beta K\,\zeta_A \\
        \beta K\,\zeta_B & 1
    \end{pmatrix} \begin{pmatrix}
        \delta  m_A\\
        \delta  m_B
    \end{pmatrix} \nonumber\\
    &=& -M \begin{pmatrix}
        \delta m_A\\
        \delta  m_B
    \end{pmatrix}
\end{eqnarray}

%We require
%\begin{equation}
%    \beta \int_{0}^{\infty}\!\langle \delta m_i(0)\,\delta m_j(t)\rangle\,dt
%    \;=\; \beta \int_0^\infty \!\langle y(0)\,y^{\!\top}(t)\rangle\,dt
%\end{equation}
Since $\langle \delta m_i(0)\,\eta(t)\rangle=0$ for $t>0$, substituting $\delta \vec m(t)=e^{-Mt}\delta \vec m(0)$ yields
\begin{widetext}
\begin{equation}
\begin{aligned}
    \beta \int_0^\infty \!\langle \delta m_i(0)\,\delta m_j (t)\rangle\,dt
    &= \beta\,\langle \delta m_i(0)\,\delta m_j (0)\rangle \int_0^\infty \!e^{-M^{\!\top}t}\,dt
     \;=\; \beta\,\Sigma\,(M^{\!\top})^{-1} \\[3pt]
    &= \frac{2\beta}{\Gamma\big(1-\beta^2 K^2 \zeta_A \zeta_B\big)^{\!2}}
    \begin{pmatrix}
        \zeta_A\big(1+\beta^2 K^2 \zeta_A \zeta_B\big) & -\,2\beta K\,\zeta_A\zeta_B\\
        -\,2\beta K\,\zeta_A\zeta_B & \zeta_B\big(1+\beta^2 K^2 \zeta_A \zeta_B\big)
    \end{pmatrix}\!
\end{aligned}
\end{equation}
\end{widetext}

To express the dissipation metric in the control variables $\boldsymbol{\lambda}=(\beta,\beta h)$, we transform fluctuations to $(-E,m)$ via
\begin{align}
    E &= \tfrac{1}{2} K\, m_A m_B\\
    m &= \tfrac{1}{2}(m_A+m_B)
\end{align}
so the Jacobian is
\begin{equation}
    \mathcal{J} \;=\; \frac{\partial(-E,m)}{\partial(m_A,m_B)} 
    \;=\; \frac{1}{2}
    \begin{pmatrix}
        -K m_B & -K m_A \\
        \;1 & \;1
    \end{pmatrix}
\end{equation}
The dissipation metric then reads
\begin{widetext}
\begin{equation}
\resizebox{\columnwidth}{!}{$
\begin{aligned}
    g_{\mu\nu}
    \;= & \; \beta\, \mathcal{J}\,\Sigma\,(M^{\!\top})^{-1} \mathcal{J}^{\!\top} \\
    \;= & \; \frac{\beta}{2\Gamma\big(1-\beta^2 K^2 \zeta_A \zeta_B\big)^{\!2}}
    \begin{pmatrix}
    \scriptstyle K^2 \!\left[\!\big(1+\beta^2 K^2 \zeta_A \zeta_B\big)\big(m_A^2 \zeta_B + m_B^2 \zeta_A\big) - 4\beta K\, m_A m_B \zeta_A \zeta_B \!\right]
    &
    \scriptstyle -K \!\left[\!\big(1+\beta^2 K^2 \zeta_A \zeta_B\big)\big(m_A \zeta_B+m_B \zeta_A\big) - 2\beta K (m_A+m_B)\zeta_A\zeta_B \!\right]
    \\[6pt]
    \scriptstyle -K \!\left[\!\big(1+\beta^2 K^2 \zeta_A \zeta_B\big)\big(m_A \zeta_B+m_B \zeta_A\big) - 2\beta K (m_A+m_B)\zeta_A\zeta_B \!\right]
    &
    \scriptstyle \big(1+\beta^2 K^2 \zeta_A \zeta_B\big)(\zeta_A+\zeta_B) - 4\beta K\,\zeta_A\zeta_B
    \end{pmatrix}\!
\end{aligned}
$}
\end{equation}
\end{widetext}



%Figures~\ref{fig:sivakmetric_antiferro} show the resulting metric components.
\section{Minimal Dissipation in Disordered Phase}
In the disordered phase, the magnetizations of the two sublattices are identical. Therefore, the total magnetizaiton
\begin{align}
    m &= \frac{m_A+m_B}{2} = m_A = m_B
\end{align}

For each $m\in[-1,1]$, the corresponding critical parameters $(\beta_c,\alpha_c)$ on the phase boundary satisfy (here and in what follows we use $\alpha=\beta h$ for simplicity) 
\begin{align}
    m &= \pm\sqrt{1-\frac{1}{\beta_c K}} \label{m(beta)}\\
    \alpha_c &= \pm\Big(\beta_c K\, m + \arctanh m\Big) \label{alpha(beta,m)}
\end{align}

\paragraph{Eigenstructure and the zero-length direction.}
The eigenpairs of the bracketed matrix are
\begin{align}
    \lambda_1 &= 0, 
    & \boldsymbol{v}_1 &= \begin{pmatrix} 1 \\ K m \end{pmatrix},
    \\
    \lambda_2 &= 1+K^2 m^2, 
    & \boldsymbol{v}_2 &= \begin{pmatrix} K m \\ -1 \end{pmatrix}.
\end{align}
Thus the zero mode $\boldsymbol{v}_1$ follows lines of constant $m$ in $(\beta,\alpha)$.

\paragraph{Scaling of the metric prefactor.}
From \eqref{m(beta)} we have $\zeta=1-m^2=1/(\beta_c K)$ along the phase boundary labeled by magnetization $m$, with critical parameter $\beta_c$.
Using this in the prefactor gives
\begin{equation}
    g_{\mu\nu} \;\propto\; 
    \frac{\beta}{\bigl(\beta_c+\beta\bigr)^2}\,\bigl(1+K^2 m^2\bigr).
\end{equation}

\paragraph{Scaling of the displacement along the zero mode.}
Consider two neighboring constant-$m$ lines labeled by 
$(\beta_{c1},\alpha_{c1})$ and $(\beta_{c2},\alpha_{c2})=(\beta_{c1}+\delta_\beta,\alpha_{c1}+\delta_\alpha)$.
We find their intersection with a straight line parallel to the zero mode at each point:
\begin{align}
    \begin{pmatrix}\beta\\ \alpha\end{pmatrix}
    &= a\,\boldsymbol{v}_1(\beta_{c1}) + \begin{pmatrix}\beta_{c1}\\ \alpha_{c1}\end{pmatrix}
    \label{eq:b0a0-1}\\[2pt]
    \begin{pmatrix}\beta\\ \alpha\end{pmatrix}
    &= b\,\boldsymbol{v}_1(\beta_{c2}) + \begin{pmatrix}\beta_{c1}+\delta_\beta\\ \alpha_{c1} + \dfrac{\partial \alpha_c}{\partial \beta_c}\,\delta_\beta \end{pmatrix}
    \label{eq:b0a0-2}
\end{align}
Subtracting \eqref{eq:b0a0-1} from \eqref{eq:b0a0-2} yields
\begin{equation}
    \delta_\beta 
    \begin{pmatrix} 1 \\[2pt] \dfrac{\partial \alpha_c}{\partial \beta_c} \end{pmatrix}
    =
    \begin{pmatrix}
        1 & 1 \\
        K m_2 & K m_1
    \end{pmatrix}
    \begin{pmatrix} -b \\[2pt] a \end{pmatrix}
\end{equation}
where $m_1=m(\beta_{c1})$, $m_2=m(\beta_{c2})$.
From \eqref{alpha(beta,m)} we have 
\(
\dfrac{\partial \alpha_c}{\partial \beta_c} = \dfrac{1}{m}
\)
Inverting gives
\begin{equation}
\begin{aligned}
    a \;&=\; \frac{1-m^2}{K m}\,\frac{\partial m}{\partial \beta_c}
      \;=\; \frac{1-m^2}{K m}\,\frac{2m}{(1-m^2)^2}\\
      \;&=\; \frac{2}{K(1-m^2)}
      \;=\; 2\beta_c
\end{aligned}
\end{equation}
where we used $1-m^2=1/(\beta_c K)$ and $\dfrac{\partial m}{\partial \beta_c} = \dfrac{2m}{(1-m^2)^2}$ from \eqref{m(beta)}.
Hence the displacement along the zero mode scales as 
\(
\delta\boldsymbol{\lambda}\sim (\beta+\beta_c).
\)

\paragraph{Conclusion: minimum at $\beta=0$.}
Combining the prefactor and displacement scalings,
\[
\sqrt{\delta\lambda^\mu\, g_{\mu\nu}\, \delta\lambda^\nu}
\;\propto\;
\sqrt{\beta}
\]
so the infinitesimal thermodynamic length is minimized at $\beta=0$ (infinite temperature).
Thus, within the disordered phase, the least-dissipation path approaches the high-temperature limit.
